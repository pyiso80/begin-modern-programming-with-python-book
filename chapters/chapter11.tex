\chapter{Exceptions and Exception Handling}
% syntax error or compile time error, 
% programmer error or logic error
% runtime error of external condition
    % can recover
    % cannot recover

ဖန်ရှင်တစ်ခုဟာ ကြိုတင်မျှော်မှန်းမရတဲ့ (သို့) ပုံမှန်မဟုတ်တဲ့ အခြေအနေ တစ်ရပ်ရပ်ကြောင့် သူ့တာဝန် ပြီးမြောက်အောင်မြင်အောင် ဆက်လက်လုပ်ဆောင်ဖို့ မဖြစ်နိုင်တဲ့အခါ \fEn{exception} တစ်ခုကို  \fEn{raise} လုပ်နိုင်ပါတယ်။ (မြန်မာလိုတော့ \fEn{exception} တက်အောင် လုပ်တာလို့ ပြောလေ့ရှိတယ်)။ ဒီလိုဖန်ရှင်တစ်ခုကို ခေါ်တဲ့အခါ အမြဲတမ်း အဆင်ပြေပြေ ချောချောမွေ့မွေ့နဲ့ ပုံမှန်အတိုင်း ပြီးသွားမယ် ပုံသေတွက်လို့မရတော့ဘူး။ ပြဿနာဖြစ်ခဲ့ရင် ဘာဆက်လုပ်မလဲ ထည့်သွင်းစဉ်းစားဖို့ လိုလာတယ်။
\section{Mechanism}


အောက်ပါ \fCode{fun\_c} ဟာ အကြိမ်တစ်ရာ ခေါ်ရင် (၅၀) လောက် မအောင်မြင်မဲ့ ဖန်ရှင်တစ်ခုကို \fEn{simulate} လုပ်ထားတာပါ။ လိုင်းပေါ်ကနေ ဒေတာတချို့ ဖတ်ပေးတဲ့ ဖန်ရှင်လို့ ယူဆနိုင်ပါတယ်။  အင်တာနက်လိုင်း မကောင်းတဲ့ ဒေသမှာဆိုတော့ ဒီဖန်ရှင်က မကြာခဏ ပြဿနာပေးတယ်။

%
\begin{py}
import random

def fun_c():
    print('Starting fun_c...')
    # simulate IOError, will fail 50% of the time
    if random.uniform(0.0, 1.0) <= 0.5:
        raise IOError("Failed to read!")
    print('fun_c ends!')
\end{py}
%
\fCode{fun\_c} ကို \fCode{main} ကနေ ခေါ်ပြီး အကြိမ်အနည်းငယ် \fEn{run} ကြည့်ပါ။
%
\begin{py}
def main():
    fun_c()
    print('main ends')

if __name__ == "__main__":
    main()
\end{py}
%

အဆင်ပြေတဲ့ အခါမှာ အခုလို
\begin{codetxt}
Starting main...
Starting fun_c...
fun_c ends!
main ends!
\end{codetxt}
ထွက်တယ်။ \fEn{Exception} တက်ရင်တော့ ဒီလိုမျိုး  \fEn{error} \fOpn{မက်ဆေ့ချ်}တွေ
\begin{codetxt}
Traceback (most recent call last):
  File ".../ch11/how_exceptions_works1.py", line 19, in <module>
    main()
  File ".../ch11/how_exceptions_works1.py", line 14, in main
    fun_c()
  File ".../ch11/how_exceptions_works1.py", line 8, in fun_c
    raise IOError("Failed to read!")
OSError: Failed to read!
Starting main...
Starting fun_c...
\end{codetxt}
ကျလာမှာပါ။ \fCode{main()} ကနေ \fCode{fun\_c()} ခေါ်ပြီး အဲဒီမှာ \fCode{IOError} ဖြစ်သွားတယ်လို့ ဖော်ပြထားတာ တွေ့ရတယ်။ \fCode{(most recent call last)} လို့လည်း တွေ့ရတယ်။ အောက်ဆုံးမှာ နောက်ဆုံးခေါ်ခဲ့တဲ့ ဖန်ရှင်လို့ ဆိုလိုတာ (နောက်ဆုံး ခေါ်ခဲ့တာ \fCode{fun\_c})။ ဖန်ရှင်တစ်ခုမှာ \fEn{exception} တက်တဲ့အခါ (သို့) ဖန်ရှင်တစ်ခုက \fEn{exception} ကို \fEn{raise} လိုက်တဲ့အခါ ၎င်းဖန်ရှင်ကို ခေါ်တဲ့ ဖန်ရှင်တွေအားလုံး ‘တောက်လျှောက် \fEn{fail} ဖြစ်မယ်’။ အခုဥပမာမှာ \fCode{main} ကနေ \fCode{fun\_c} ကို ခေါ်တယ်။ \fCode{fun\_c} မှာ \fEn{exception} ဖြစ်တော့ \fCode{main} လည်း ပြီးအောင်ဆက် အလုပ်မလုပ်ပေးနိုင်ဘူး။ \fEn{fail} ဖြစ်သွားတယ်။ ဆိုလိုတာက \fCode{fun\_c} မှာဆိုရင် \fEn{exception} ဖြစ်စေတဲ့နေရာ အောက်ပိုင်းက စတိတ်မန့်တွေ၊ \fCode{main} မှာဆိုရင် \fCode{fun\_c} ကိုခေါ်ထားတဲ့နေရာရဲ့ အောက်က စတိတ်မန့်တွေ ဆက်လက်အလုပ်မလုပ်တော့ဘူး (\fCode{main} အတွက် \fCode{fun\_c} ခေါ်တဲ့လိုင်းက \fEn{exception} ဖြစ်စေတဲ့နေရာ)။  အခုကိစ္စမှာ ဖန်ရှင်နှစ်ခုလုံးရဲ့ \fEn{exception} ဖြစ်တဲ့နေရာ အောက်ပိုင်းမှာ \fCode{print} စတိတ်မန့် တစ်ကြောင်းစီပဲ ရှိပါတယ်။ အခြားစတိတ်မန့်တွေ ရှိခဲ့ရင်လည်း အလုပ်လုပ်မှာ မဟုတ်ဘူး။

အကယ်၍ \fCode{main} က \fCode{fun\_a} \fCode{fun\_a} က \fCode{fun\_b} ကိုခေါ်၊ \fCode{fun\_b} ကနေမှ \fCode{fun\_c} ကို နောက်ဆုံး ခေါ်ထားရင် \fCode{fun\_c} မှာ \fEn{exception} ဖြစ်တဲ့အခါ \fCode{fun\_b} က စပြီး \fEn{fail} ဖြစ်မယ်။ ပြီးရင် သူ့ကိုခေါ်တဲ့ \fCode{fun\_a} ဆက် \fEn{fail} မယ်။ နောက်ဆုံးမှာ \fCode{fun\_a} ကိုခေါ်ထားတဲ့ \fCode{main} ဖန်ရှင် \fEn{fail} ဖြစ်ပြီး ပရိုဂရမ်တစ်ခုလုံး \fEn{crash} ဖြစ်မှာ ဖြစ်တယ်။ ရှေ့စာပိုဒ်မှာ ပြောခဲ့တဲ့ ‘တောက်လျှောက် \fEn{fail} ဖြစ်မယ်’ ဆိုတာ အခုလို ဖြစ်စဉ်ကို ဆိုလိုတာပါ။

%
\begin{py}
def main():
    print('Starting main...')
    fun_a()
    print('main ends!')

def fun_a():
    print('Starting fun_a...')
    fun_b()
    print('fun_a ends!')

def fun_b():
    print('Starting fun_b...')
    fun_c()
    print('fun_b ends!')

if __name__ == "__main__":
    main()
\end{py}
%

ဖန်ရှင်တစ်ခုက အစပြု \fEn{exception} ဖြစ်တဲ့အခါ သူ့ကိုခေါ်တဲ့ ဖန်ရှင်ကောလ် ကွင်းဆက် \fEn{(call chain)} တစ်လျှောက် ပါဝင်တဲ့ဖန်ရှင် တစ်ခုပြီးတစ်ခု \fEn{exception} ဖြစ်စေပြီး နောက်ဆုံးမှာ ပရိုဂရမ်တစ်ခုလုံး အလုပ်ဆက်မလုပ်တော့ဘဲ ရပ်သွားမှာပါ။ \fEn{Exception} ဖြစ်လို့ ပရိုဂရမ်က ရပ်သွားမယ်ဆိုရင် တကယ်ကို အဆင်မပြေဘူး။ \fEn{Exception} ဖြစ်ပေါ်စေနိုင်တဲ့ အခြေအနေတွေက ရာနဲ့ချီ ရှိနိုင်တယ်။ \fEn{Exception} ဖြစ်ခဲ့ရင်  ပရိုဂရမ်တစ်ခုလုံးကို ပြန့်မသွားဘဲ၊ မသက်ရောက်စေဘဲ ထိန်းကွပ်ပေးလို့ရတဲ့ နည်းလမ်းရှိရပါမယ်။ အဲဒါကတော့ \fEnEmp{exception handling} ပါပဲ။ 

\subsection*{Exception Handling}
စောစောက ဥပမာမှာ \fCode{fun\_c} ကိုခေါ်တဲ့အခါ ဖြစ်နိုင်တဲ့ \fEn{exception} ကို \fCode{fun\_b} မှာ အခုလို \fEn{handle} လုပ်နိုင်ပါတယ်။ 

%
\begin{py}
def fun_b():
    print('Starting fun_b...')
    try:
        fun_c()
        print("fun_c was successful!")
    except IOError as e:
        print(e)
        print("Poor connection!")
    print('fun_b ends!')
\end{py}
%
\fEn{Python} မှာ \fCode{try...except} က \fEn{exception handling} အတွက်ပါ။ \fEn{Exception} ဖြစ်နိုင်တဲ့ ဖန်ရှင်ကို ခေါ်တဲ့အခါ \fCode{try} ဘလောက်ထဲမှာ ခေါ်ရပါမယ် (ဖြစ်ခဲ့ရင် \fEn{handle} လုပ်မယ်ဆိုတဲ့ ရည်ရွယ်ချက်ရှိရင်ပေါ့)။ \fCode{except} ဘလောက်က \fEn{exception} ဖြစ်ခဲ့ရင် ဘယ်လို \fEn{handle} လုပ်မဲ့ ကိစ္စအတွက်။

%
\begin{py}
except IOError as e:
\end{py}
%
\fCode{IOError} က \fEn{handle} လုပ်မဲ့ \fEn{exception} အမျိုးအစားကို သတ်မှတ်တာ။ ဆိုလိုတာက \fCode{IOError} သီးသန့်ကိုပဲ \fEn{handle} လုပ်မယ်။ \fCode{IOError} မဟုတ်တဲ့ အခြား \fEn{exception} တွေကို \fEn{handle} မလုပ်ဘူး။ (ဒါနဲ့ ပါတ်သက်ပြီး နောက်ပိုင်းမှာ ထပ်ရှင်းပြမှာပါ)။ \fCode{e} က ဖြစ်ပေါ်တဲ့ \fCode{IOError} \fEn{exception} အတွက် ဗေရီရေဘဲလ်ပါ။ \fEn{Exception} ဖြစ်ရင် \fCode{fun\_c} က \fCode{raise} လုပ်လိုက်တဲ့ \fCode{IOError} အော့ဘ်ဂျက်ကို ဒီဗေရီရေဘဲလ်မှာ ထည့်ပေးမှာ ဖြစ်တယ်။ (\fEn{Python} မှာ \fCode{IOError}\fEn{,} \fCode{ValueError}\fEn{,} \fCode{NameError} စတဲ့ ကလပ်စ်တွေ ပါရှိပြီး ဖြစ်ပေါ်တဲ့ \fEn{exception} အမျိုးအစားအလိုက် သက်ဆိုင်ရာ အော့ဘ်ဂျက်ကို \fCode{raise} လုပ်ရတာပါ)။

စောစောက ဥပမာမှာ \fCode{fun\_b} ကို အထက်ပါအတိုင်း  \fEn{exception handling} ထည့်ပြီး စမ်းသပ်ကြည့်တဲ့အခါ \fEn{exception} ဖြစ်တဲ့အခါ အခုလို
\begin{codetxt}
Starting main...
Starting fun_a...
Starting fun_b...
Starting fun_c...
Failed to read!
Poor connection!
fun_b ends!
fun_a ends!
main ends!
\end{codetxt}
တွေ့ရမှာပါ။ \fCode{fun\_b} မှာ \fEn{exception handling} လုပ်ထားတဲ့အတွက်  \fCode{fun\_c} က \fEn{exception} ဖြစ်ခဲ့ရင် အဲ့ဒီ \fEn{exception} ဟာ \fCode{fun\_a} နဲ့ \fCode{main} ဆီကို ထပ်ဆင့် မကူးစက်သွားတော့ဘူး။ \fEn{Exception handling} ဆိုတာ ပရိုဂရမ် အခြားအစိတ်အပိုင်းတွေကို \fEn{exception} မကူးစက်သွားအောင် ကွာရန်တင်းလုပ် ထိန်းချုပ်တာလို့ ယူဆနိုင်ပါတယ်။ 

နောက်ထပ် သိထားဖို့ အရေးကြီးတာ တစ်ခုက \fEn{exception} ဖြစ်တဲ့အခါ \fCode{try} ဘလောက်ထဲမှာပါတဲ့ အောက်က စတိတ်မန့်တွေကို ကျော်ပြီး \fCode{except} ဘလောက်ထဲ ချက်ချင်း ရောက်သွားမှာပါ။ \fCode{try} ဘလောက်ဟာ ပုံမှန်အတိုင်းဆိုရင် လုပ်ဆောင်မဲ့ လမ်းကြောင်း \fEn{(normal execution flow)} အတွက်ပါ။ \fEn{Exception} ဖြစ်ခဲ့ရင်တော့ ဒီလမ်းကြောင်းအတိုင်း ဆက်အလုပ်လုပ်လို့ မရတော့ဘူး။ ပုံမှန်မဟုတ်တဲ့ အခြေအနေမှာ လုပ်ဆောင်ရမဲ့ \fCode{except} ဘလောက်ကို လွှဲပြောင်း လုပ်ဆောင်ပေးရပါမယ်။ အခု ပြထားတာက ပုံမှန်အတိုင်း သွားမဲ့လမ်းကြောင်းပါ။
%
\begin{tikzpicture}[
  remember picture,
  overlay,
  annotation/.style={
    inner sep=0pt,
    outer sep=0pt,
    outer xsep=1mm,
    fill=yellow!80!black,
    text width=5cm
  },
  >={Stealth[inset=0pt, angle=30:7pt]}
]
\filldraw[lightgray!60,opacity=0.5] ([xshift=-1ex,yshift=1em] pic cs:ch11a1) rectangle ([xshift=.25ex,yshift=-.5em] pic cs:ch11a2);
\draw[->, thin] (pic cs:ch11a2)  ++(.25ex,.5ex) -- ++(0.5,0) |- ([yshift=0.5ex] pic cs:ch11a3);
\end{tikzpicture}
%
%
\begin{py}
def fun_b():
    print('Starting fun_b...')
    try:
        ß\tikzmark{ch11a1}ßfun_c()
        print("fun_c was successful!")ß\tikzmark{ch11a2}ß
    except IOError as e:
        print(e)
        print("Poor connection!")
    print('fun_b ends!')ß\tikzmark{ch11a3}ß
\end{py}
%
ဒါက \fCode{IOError} ဖြစ်ခဲ့ရင် လုပ်ဆောင်မဲ့ပုံ (\fCode{fun\_c} ခေါ်ထားတဲ့လိုင်းကနေ \fCode{except} ကို ခုန်ပြီးရောက်သွားတာ သတိပြုပါ)။
%
\begin{tikzpicture}[
    remember picture,
    overlay,
    annotation/.style={
      inner sep=0pt,
      outer sep=0pt,
      outer xsep=1mm,
      fill=yellow!80!black,
      text width=5cm
    },
    >={Stealth[inset=0pt, angle=30:7pt]}
  ]
  
  \draw[->, thin] (pic cs:ch11b1)  ++(.25ex,.5ex) -- ++(12em,0) |- ([xshift=-.5ex,yshift=1ex] pic cs:ch11b3);
  \filldraw[lightgray!90,opacity=0.5] ([xshift=-1ex,yshift=1em] pic cs:ch11b2) rectangle ([xshift=.25ex,yshift=-.5em] pic cs:ch11b4);
  \draw[->, thin] (pic cs:ch11b4)  ++(.25ex,.5ex) -- ++(.5em,0) |- ([yshift=0.5ex]  pic cs:ch11b5);
  
\end{tikzpicture}
%
%
\begin{py}
def fun_b():
    print('Starting fun_b...')
    try:
        fun_c()ß\tikzmark{ch11b1}ß
        print("fun_c was successful!")
    except IOError as e:
        ß\tikzmark{ch11b2}ßprint(e)                  ß\tikzmark{ch11b3}ß
        print("Poor connection!")ß\tikzmark{ch11b4}ß
    print('fun_b ends!')ß\tikzmark{ch11b5}ß
\end{py}
%

ရှေ့က ဥပမာမှာ \fCode{fun\_b}  \fEn{handle} မလုပ်ဘဲ \fCode{fun\_a} က \fEn{handle} လုပ်လို့လဲရတယ်။ ဒါမှမဟုတ် \fCode{fun\_b} နဲ့ \fCode{fun\_a} မှာ \fEn{handle} မလုပ်ဘဲ \fCode{main} က လုပ်နိုင်ပါတယ်။ အောက်ပါအတိုင်း \fCode{fun\_a} မှာ \fEn{handle} လုပ်မယ်ဆိုပါစို့ 

%
\begin{py}
def fun_a():
    print('Starting fun_a...')
    try:
        fun_b()
        print("fun_b was successful!")
    except IOError as e:
        print(e)
        print("Poor connection!")
    print('fun_a ends!')


def fun_b():
    print('Starting fun_b...')
    fun_c()
    print('fun_b ends!')
\end{py}
%
\fCode{fun\_c} \fEn{exception} တက်ရင် \fEn{handle} မလုပ်ထားတဲ့ \fCode{fun\_b} လည်း  \fEn{exception} ဆက်ဖြစ်ပါမယ်။ အဲဒီ \fEn{exception} ကို \fCode{fun\_a} က \fEn{handle} လုပ်လိုက်တဲ့ အတွက် \fCode{main} ဆီကို ဆက်လက်မကူးစက် သွားတော့ပါဘူး။ \fEn{Output} အခုလို ထွက်တာ တွေ့ရမှာပါ။
%
\begin{codetxt}
Starting main...
Starting fun_a...
Starting fun_b...
Starting fun_c...
Failed to read!
Poor connection!
fun_a ends!
main ends!
\end{codetxt}
%
\fEn{Exception}  ဖြစ်တဲ့အခါ \fCode{"fun\_b ends!"} နဲ့ \fCode{"fun\_b was successful!"} အတွက် \fCode{print} စတိတ်မန့်တွေကို ကျော်သွားတာ သတိပြုကြည့်ပါ။

\subsection*{Illustration of Exception Handling}

\section{Recovery}
ဖန်ရှင်တွေ တစ်ခုပြီးတစ်ခုဆင့် ခေါ်ထားတဲ့အခါ \fEn{exception} ဖြစ်ခဲ့ရင် ဘယ်ဖန်ရှင်က \fEn{handle} လုပ်သင့်လဲ စဉ်းစားဆုံးဖြတ်ဖို့ လိုလာတယ်။ ဒီကိစ္စက ဘယ်မှာ \fEn{handle} လုပ်ရမယ် ပုံသေပြောလို့တော့ မရဘူး။ အခြေအနေနဲ့ လိုအပ်ချက်ပေါ် မူတည် ဆုံးဖြတ်ရတာမျိုး။ 
%
%
\begin{py}
def read_sensor():
    if random.uniform(0.0, 1.0) <= 0.2:
        raise IOError("Failed to read!")
    return random.randrange(1, 11)
\end{py}
%
ဒီဖန်ရှင်က \fEn{sensor device} တစ်ခုဆီကနေ ဒေတာဖတ်တာကို \fEn{simulate} လုပ်ထားတဲ့ ဖန်ရှင်ပါ။ \fEn{Sensor} ကြောင်လို့သော်လည်းကောင်း၊ \fEn{network} ကြောင့်သော်လည်းကောင်း (၂၀) ရာနှုန်း \fEn{fail} ဖြစ်တယ် ဆိုပါတော့။ 

\fEn{Sensor} တန်ဖိုး သုံးခုတစ်တွဲ ဖတ်ပြီး စောင့်ကြည့်လေ့လာရမယ် ဆိုပါစို့။ တန်ဖိုးသုံးခု အတွဲလိုက်ရအောင် ဖတ်ရမှာပါ။ \fEn{Exception} ဖြစ်လို့ သုံးခုမပြည့်သေးရင် ပြည့်တဲ့ထိ ထပ်ကြိုးစားရပါမယ်။  ဖတ်တဲ့အခါ တစ်ခါနဲ့တစ်ခါ စက္ကန့်တစ်ဝက်ခြား ဖတ်ပါတယ်။ တစ်ကယ့် လက်တွေ့မှာလည်း \fEn{sensor} ကနေ ဒေတာဖတ်တဲ့အခါ တရစပ် ဖတ်လေ့မရှိဘူး။ အချိန်အနည်းငယ် ခြားပြီးဖတ်တယ်။
%
\begin{py}
def read_3vals():
    vals = []
    while len(vals) != 3:
        try:
            vals.append(read_sensor())
        except IOError as err:
            pass
        time.sleep(0.5)
    return vals
\end{py}
%

အခုတစ်ခါ \fEn{sensor} တန်ဖိုး သုံးခုဆက်တိုက် ပြဿနာတစ်စုံတစ်ရာ မရှိဘဲ ဖတ်နိုင်တဲ့ အကြိမ်အရေအတွက်ကို စောင့်ကြည့်လေ့လာဖို့လည်း လိုအပ်လာတယ် ဆိုပါစို့။ 

%
\begin{py}
def read_3_times():
    vals = []
    for i in range(3):
        vals.append(read_sensor())
        time.sleep(0.5)
    return vals
\end{py}
%
ဒီဖန်ရှင်က ပုံမှန်ဆိုရင်တော့  \fEn{sensor} ကို သုံးကြိမ်ဖတ်မှာပါ။ ဒါပေမဲ့ \fCode{read\_sensor} မှာ \fCode{IOError} \fEn{exception} ဖြစ်ခဲ့ရင် \fEn{handle} မလုပ်ထားတဲ့အတွက်  အခု \fCode{read\_3\_times} လည်း ဆက် \fEn{fail} ဖြစ်မယ်။ ကံကောင်းလို့ သုံးခါလုံး အဆင်ပြေခဲ့ရင်တော့ ဖတ်ထားတဲ့ တန်ဖိုးသုံးခုပါတဲ့ \fCode{vals} ကို ပြန်ပေးမှာပါ။  စောစောကဖန်ရှင်နဲ့ အဓိက ကွာခြားချက်ကို သတိပြုပါ။ \fCode{read\_3vals} တန်ဖိုး သုံးခုပြည့်တဲ့ထိ ဖတ်ပေးရမှာပါ။ ဒါကြောင့် \fEn{exception handle} လုပ်ဖို့ လိုကို လိုတယ်။ ခုဖန်ရှင်က တန်ဖိုးသုံးခုကို ကြားထဲမှာ \fEn{fail} မဖြစ်ဘဲ ဆက်တိုက်  ဖတ်လို့ရပဲ ပြန်ပေးရမယ်။ ဒါကြောင့် {handle} လုပ်ဖို့ မလိုအပ်ဘူး။

\fCode{read\_3\_times} ဟာ ဖြစ်ပေါ်နိုင်တဲ့ \fEn{exception} ကို \fEn{handle} မလုပ်တဲ့ ဖန်ရှင်တစ်ခုရဲ့ သာဓကဖြစ်တယ်။ သူ့အနေနဲ့ တန်ဖိုးသုံးခု ဆက်တိုက် ဖတ်လို့ရင် ပြန်ပေးတယ်။ အကယ်၍ မရခဲ့ရင်တော့  ၎င်းကိုယ်တိုင် \fEn{fail} ဖြစ်သွားမှာပါ။ ဒီတော့ ဖန်ရှင်ကောလ် ကွင်းဆက်တစ်လျှောက်မှာ ဘယ်နေရာမှာ \fEn{exception} ကို \fEn{handle} လုပ်မလဲ။ တစ်နေရာရာမှာ လုပ်တာတော့ လုပ်ရပါမယ်။ မလုပ်ရင် ပရိုဂရမ်တစ်ခုလုံး \fEn{fail} ဖြစ်သွားမှာ။

\fCode{read\_3\_times} ကို ခေါ်တဲ့ ဖန်ရှင်က \fCode{read\_3\_times} \fEn{exception} ဖြစ်ရင် \fEn{handle} လုပ်နိုင်ပါတယ်။ ဒါမှမဟုတ် လိုအပ်ချက်အရ မလုပ်ဘဲ သူကိုယ်တိုင် \fEn{fail} အဖြစ်ခံတာလည်း ဖြစ်နိုင်တယ်။ \fEn{handle} မလုပ်ဘူး (\fEn{fail} ဆက်ဖြစ်မယ်) ဆိုရင်တော့ သူ့ကို ခေါ်ထားတဲ့ ဖန်ရှင်က \fEn{handle/fail} ဆက်လက် ဆုံးဖြတ်ရပါမယ်။ ဒီသဘောအတိုင်း တစ်ဆင့်ပြီးတစ်ဆင့် \fEn{handle} မလုပ်ဘဲ ဆက်သွားရင် နောက်ဆုံးမှာ \fEn{entry point} ဖြစ်တဲ့ \fCode{main} ဆီကို ရောက်သွားမှာပါ။ \fEn{main} ကလည်း \fEn{handle} မလုပ်ရင် \fCode{read\_sensor} က အစပြုခဲ့တဲ့ \fCode{IOError} \fEn{exception} ဟာ  ပရိုဂရမ်တစ်ခုလုံးကို ရပ်ဆိုင်းသွားစေမှာပါ။ (တကယ့်လက်တွေ့မှာတော့ ဒီလို အဖြစ်ခံလေ့ မရှိပါဘူး။ \fEn{Global exception handler} လို့ခေါ်တဲ့ ဘယ်ဖန်ရှင်ကမှ \fEn{handle} မလုပ်လို့ အပြင်ကို ပြန့်ထွက်လာတဲ့ \fEn{exception} တွေကို \fEn{handle} လုပ်တဲ့ အဆင့်တစ်ဆင့် ထည့်ထားလေ့ရှိတယ်)။

\fCode{read\_3\_times} က (၁၀) ကြိမ်မှာ ဘယ်နှစ်ကြိမ် \fEn{fail} ဖြစ်လဲ ဆန်းစစ်ပေးမဲ့ ဖန်ရှင်ကို အခုလို သတ်မှတ်နိုင်ပါတယ်။ 
%
\begin{py}
def failure_rate():
    fail = 0
    for i in range(10):
        try:
            read_3_times()
        except IOError as e:
            fail += 1
    print(fail)
    return Fraction(fail, 10)
\end{py}
%

အခု လေ့လာတွေ့ရှိချက်တွေကို အနှစ်ချုပ်ပြန်ကြည့်ရင် ဖန်ရှင်တစ်ခု \fEn{exception} ဖြစ်တဲ့အခါ ၎င်းဖန်ရှင်ကို ခေါ်ထားတဲ့ ကွင်းဆက်တစ်လျှောက်လုံးက ဖန်ရှင်တွေ \fEn{fail} ဖြစ်နိုင်ပါတယ်။ \fEn{Fail} မဖြစ်အောင် ဖန်ရှင်တစ်ခုက \fEn{exception} ကို \fEn{handle} လုပ်ရပါမယ်။ ဘယ်ဖန်ရှင်က \fEn{handle} လုပ်ရမလဲကတော့ ပုံသေမရှိဘူး။ လိုအပ်ချက်ပေါ် မူတည်ပြီး ဆုံးဖြတ်ရလေ့ရှိတယ်။

% reading dates string from a file
    % all values must okay
    % fail if any not okay
    % all values okay and also log not okay values

% you are trying to get values from network every second
    % you want 

% m -> a -> b -> c and b try to return a special value if exception 
% m -> a -> b -> c and only the main handle exception and display appropriate message
% to the user 

%
%
\begin{figure}[tbh!]
\begin{tikzpicture}[>={Stealth[inset=0pt, angle=30:7pt]}]
\foreach \i in {0,1,2,3}
{
  \pgfmathsetmacro\x{\i * (1.5 + 2)}
  \pgfmathsetmacro\y{7 - (\i * 1.25)}
  \def\myarr{{"a", "b", "c", "d"}}
  
  % How to draw rounded corners selectively
  %\filldraw[lightgray] {[rounded corners=5](\x,\y) -- ++(1.5,0)} -- ++(0,-1)  -- ++(-1.5,0){[rounded corners=5] --   cycle};
  %\begin{scope}[yshift=-1.5cm]
  %\filldraw[lightgray] (\x,\y) -- ++(1.5,0) {[rounded corners=5] -- ++(0,-1) -- ++(-1.5,0)} -- cycle;
  %\end{scope}
  
  \draw[thick, rounded corners=5, fill=lightgray!60] (\x,\y) rectangle ++(1.5,-2.5);
  % fun_a, fun_b, ..., etc
  \begin{scope}[xshift=0.75cm]
    \node at (\x,\y) [above] {$fun\_\pgfmathparse{\myarr[\i]}\pgfmathresult$};
  \end{scope}

  % fun_b(), fun_c(), etc and Bang! sign to indicate failure
  \def\myarrB{{"b", "c", "d"}}
  \begin{scope}[xshift=0.75cm,yshift=-1.25cm]
    \ifthenelse{\i < 3 }{\node at (\x,\y) {$fun\_\pgfmathparse{\myarrB[\i]}\pgfmathresult()$};
    }{}
    \ifthenelse{\i = 3 }{\node at (\x,\y) [starburst, draw=red, fill=yellow,
       starburst point height=0.4cm, font=\scriptsize, line width=1pt, inner sep=1.5pt] {\fEnBf{Bang!}};
    }{}
    \ifthenelse{\i = 2 }{\node at (\x,\y-0.8) [starburst, draw=red, fill=yellow,
       starburst point height=0.4cm, font=\scriptsize, line width=1pt, inner sep=1.5pt] {\fEnBf{Bang!}};
    }{}
  \end{scope}
  \ifthenelse{\i = 1 }{\draw[fill,color=gray] (\x+0.1,\y-1.5) rectangle (\x+1.4,\y-2.1);}{}  
  

  % draw arrows
  \ifthenelse{\i < 3 }{
    \draw[->,thin] [xshift=1.5cm,yshift=-1.25cm](\x,\y) to[out=45] ++(2,0);
  }{}
}
\draw[->,thin] (11,1.75) .. controls (9,1) and (6,1.5) .. (4.2,4);
\draw[->,thin] (3.4,3.5) .. controls (2.5,3.5) and (1.5, 4.5) .. (1.25,5.5);
\end{tikzpicture}
\caption{Exception Handling} 
\label{fig:Exception Handling}
\end{figure}


\section{\fSecCodeBf{else} နှင့် \fSecCodeBf{finally}}
\fCode{try...except} အောက်မှာ \fCode{else} ဘလောက် နဲ့ \fCode{finally}  ဘလောက် ရှိနိုင်ပါတယ်။ \fEn{Exception} မဖြစ်တဲ့အခါမှပဲ လုပ်ဆောင်ချင်တဲ့ စတိတ်မန့်တွေကို \fCode{else} ဘလောက်ထဲမှာ ထည့်နိုင်ပါတယ်။ တစ်နည်း\allowbreak အားဖြင့် \fCode{try} ဘလောက် ပြဿနာမရှိဘဲ အောင်မြင်ပြီးစီးမှသာလျှင် \fCode{else} ဘလောက်ကို လုပ်ဆောင်မှာပါ။ \fEn{Exception} ဖြစ်ခဲ့ရင်တော့ လုပ်ဆောင်ပေးမှာ မဟုတ်ပါဘူး။
%
%
\begin{py}
def process_sensor_data():
    while True:
        try:
            val = read_sensor()
        except IOError as e:
            print(e)
        else:
            use_data(val)
            notify_if_necessary()
            print("Sensor data read successfully...")

        time.sleep(1)
        print("One iteration completed...")
\end{py}
%
% ---
%
\begin{codetxt}
Sensor data read successfully...
One iteration completed...
Sensor data read successfully...
One iteration completed...
Failed to read!
One iteration completed...
Failed to read!
\end{codetxt}
ဒီ \fEn{output} ကို လေ့လာကြည့်ရင် \fCode{else} ဘလောက်ကို \fEn{exception} မတက်မှပဲ လုပ်ဆောင်ပေးတယ်ဆိုတာ မြင်နိုင်မှာပါ။  \fCode{try...except...else} အပြင် အောက်ဆုံးက နှစ်ကြောင်းနဲ့ သဘောတရား မတူတာကိုလည်း သတိပြု ကြည့်ပါ။
%
%
\begin{py}
time.sleep(1)
print("One iteration completed...")
\end{py}
%
% 
\fEn{Exception} ဖြစ်ဖြစ်၊ မဖြစ်ဖြစ် ဒီ စတိတ်မန့် နှစ်ခုကိုက လုပ်ဆောင်ပေးတယ်။ \fCode{else} အပိုင်းကိုတော့ \fEn{exception} မတက်မှပဲ လုပ်ဆောင်တယ်။ 

စောစောက ဖန်ရှင်ကို \fCode{else} မပါဘဲ အောက်ပါကဲ့သို့ ရေးမယ်ဆိုရင်လည်း ရလဒ်အားဖြင့် တူတူပါပဲ။ \fCode{else} ကို အသုံးပြုရတဲ့ အဓိက အကြောင်းအရင်းက \fEn{exception} ဖြစ်စေနိုင်တဲ့ အပိုင်းနဲ့ မဖြစ်စေနိုင်တဲ့ အပိုင်း သီးသန့်ခွဲထားဖို့အတွက်ပါ။ 
%
%
\begin{py}
def process_sensor_data():
    while True:
        try:
            val = read_sensor()
            use_data(val)
            notify_if_necessary()
            print("Sensor data read successfully...")
        except IOError as e:
            print(e)
        time.sleep(1)
        print("processed single sensor data successfully...")
\end{py}
%
အခုပုံစံမှာက \fCode{try} ထဲက ဖန်ရှင်တွေထဲက ဘယ်ဟာက \fEn{exception} ဖြစ်စေတဲ့ အရင်းအမြစ်လဲ အလွယ်\allowbreak တကူ မသိနိုင်တော့ဘူး။ စောစောက \fCode{else} နဲ့ ပုံစံမှာက \fCode{read\_sensor} က \fEn{exception} ဖြစ်နိုင်တဲ့ ဖန်ရှင်ဖြစ်ရမယ်၊ အဲဒီ \fEn{exception} ကို \fEn{handle} လုပ်ထားတယ်ဆိုတာ သိသာ မြင်သာပါတယ်။%
%
\vspace{-1\baselineskip}
\begin{multicols}{2}%
\noindent
\begin{tikzpicture}[
  remember picture,
  overlay,
  annotation/.style={
    inner sep=0pt,
    outer sep=0pt,
    outer xsep=1mm,
    fill=yellow!80!black,
    text width=5cm
  },
  >={Stealth[inset=0pt, angle=30:7pt]}
]
\node at ([xshift=1.3cm]pic cs:ch11c0) [circle, aspect=2, draw=black, fill=green!50, font=\scriptsize, line width=1pt, inner sep=1.5pt] {\huge\ding{51}};
\draw[thin,rounded corners=5,fill=lightgray] ([yshift=1.5ex]pic cs:ch11c1) rectangle ([xshift=2cm]pic cs:ch11c2);
\draw[thin,rounded corners=5,fill=darkgray] ([yshift=1.5ex]pic cs:ch11c3) rectangle ([xshift=2cm]pic cs:ch11c4);
\draw[thin,rounded corners=5,fill=lightgray] ([yshift=1.5ex]pic cs:ch11c5) rectangle ([xshift=2cm]pic cs:ch11c6);
\draw[thin,rounded corners=5,fill=lightgray] ([yshift=1.5ex]pic cs:ch11c7) rectangle ([xshift=2cm]pic cs:ch11c8);
%\draw[->, thin] ([yshift=1.5cm,xshift=2.2cm]pic cs:ch11c2)  -- ([yshift=0.5ex,xshift=2.2cm] pic cs:ch11c2);
\end{tikzpicture}
%
%
\begin{py}
try:
    ß$\ldots$ß
    ß\bfseries\fEnEmp{some\_fun()}\tikzmark{ch11c0}ß
    ß\tikzmark{ch11c1} ß
     ß$\ldots$ß
    ß\tikzmark{ch11c2}ß
except:
    ß\tikzmark{ch11c3} ß
     ß$\ldots$ß
    ß\tikzmark{ch11c4}ß
else:
    ß\tikzmark{ch11c5} ß
     ß$\ldots$ß
    ß\tikzmark{ch11c6}ß

ß\tikzmark{ch11c7} ß
 ß$\ldots$ß
ß\tikzmark{ch11c8}ß
\end{py}
\columnbreak%
%
%
%
% ---
% 
%
%
\begin{tikzpicture}[
  remember picture,
  overlay,
  annotation/.style={
    inner sep=0pt,
    outer sep=0pt,
    outer xsep=1mm,
    fill=yellow!80!black,
    text width=5cm
  },
  >={Stealth[inset=0pt, angle=30:7pt]}
]
\node at ([xshift=1.3cm]pic cs:ch11d0) [starburst, draw=red, fill=yellow,
       starburst point height=0.4cm, font=\scriptsize, line width=1pt, inner sep=1.5pt] {\fEnBf{Bang!}};
\draw[thin,rounded corners=5,fill=darkgray] ([yshift=1.5ex]pic cs:ch11d1) rectangle ([xshift=2cm]pic cs:ch11d2);
\draw[thin,rounded corners=5,fill=lightgray] ([yshift=1.5ex]pic cs:ch11d3) rectangle ([xshift=2cm]pic cs:ch11d4);
\draw[thin,rounded corners=5,fill=darkgray] ([yshift=1.5ex]pic cs:ch11d5) rectangle ([xshift=2cm]pic cs:ch11d6);
\draw[thin,rounded corners=5,fill=lightgray] ([yshift=1.5ex]pic cs:ch11d7) rectangle ([xshift=2cm]pic cs:ch11d8);
%\draw[->, thin] ([yshift=1.5cm,xshift=2.2cm]pic cs:ch11d2)  -- ([yshift=0.5ex,xshift=2.2cm] pic cs:ch11d2);
\end{tikzpicture}
%
%
\begin{py}
try:
    ß$\ldots$ß
    ß\bfseries\fEnEmp{some\_fun()}\tikzmark{ch11d0}ß
    ß\tikzmark{ch11d1} ß
     ß$\ldots$ß
    ß\tikzmark{ch11d2}ß
except:
    ß\tikzmark{ch11d3} ß
     ß$\ldots$ß
    ß\tikzmark{ch11d4}ß
else:
    ß\tikzmark{ch11d5} ß
     ß$\ldots$ß
    ß\tikzmark{ch11d6}ß

ß\tikzmark{ch11d7} ß
 ß$\ldots$ß
ß\tikzmark{ch11d8}ß
\end{py}
\end{multicols}%
\noindent \fEn{Exception} မဖြစ်တဲ့ အခါနဲ့ ဖြစ်တဲ့အခါ \fCode{try...except...else} အလုပ်လုပ်ပုံ နှိုင်းယှဉ်ပြထားတာပါ။ မီးခိုးရောင် အဖျော့ ဘလောက်တွေက လုပ်ဆောင်မဲ့ ဘလောက်တွေပါ။ မီးခိုးရင့်ရောင် ဘလောက်တွေကိုတော့ လုပ်ဆောင်မှာ မဟုတ်ပါဘူး။


\subsection*{\fSubSecCodeBf{finally}}

\fCode{except} အပြီးမှာ \fCode{finally} ဘလောက် ရှိနိုင်ပါတယ်။ \fEn{Exception} ဖြစ်သည်ဖြစ်စေ၊ မဖြစ်သည်ဖြစ်စေ \fCode{finally} အပိုင်းကို လုပ်ဆောင်ပေးမှာ ဖြစ်တယ်။ \fCode{return} လုပ်ရင်တောင်မှ \fCode{finally} လုပ်ဆောင်ပြီးမှပဲ လုပ်မှာပါ။
%
%
\begin{py}
def test_read():
    try:
        val = read_sensor()
        print("Value: " + str(val))
        return
    except IOError as e:
        print("Error while reading!")
        return
    finally:
        print("Don't skip this!!!")
\end{py}
%
ဒီဖန်ရှင်ကို ထပ်ခါထပ်ခါ \fEn{run} ပြီး စမ်းကြည့်ပါ။ \fEn{Exception} မဖြစ်တဲ့ အခါ
\begin{codetxt}
Value: 5
Don't skip this!!!
\end{codetxt}
ဖြစ်တဲ့အခါ
\begin{codetxt}
Error while reading!
Don't skip this!!!
\end{codetxt}
ကို တွေ့ရမှာပါ။ 

ဖန်ရှင် \fCode{return} မဖြစ်မီ \fCode{finally} ဘလောက်ကို လုပ်ဆောင်တာကို တွေ့ရတယ်။  \fCode{return} လုပ်ရင် ခေါ်ခဲ့တဲ့နေရာ ချက်ချင်းပြန်ရောက်တယ် ဆိုပေမဲ့ \fCode{finally} ပါရင်တော့ ချွင်းချက်အနေနဲ့ မှတ်ရပါမယ်။  \fCode{try...except} ဘလောက်တွေဟာ တကယ့်လက်တွေ့မှာ အခုဥပမာတွေလို ရိုးရှင်းမှာ မဟုတ်ဘူး။ ဒီ့ထက် အများကြီး ပိုပြီးရှုပ်ထွေး နိုင်ပါတယ်။ ကွန်ဒီရှင်နယ်လ်တွေ၊ \fEn{loop} တွေ၊ \fCode{break}\fEn{,} \fEn{early} \fCode{return} စတဲ့ဟာတွေ ရောယှက်နေတဲ့အခါ နောက်ဆုံးပိတ် လုပ်ဆောင်ပေးရမဲ့ \fEn{final steps} တချို့ ကျန်ခဲ့တာ ဖြစ်ဖို့ အလားအလာများတယ်။ ဥပမာ \fCode{test\_read} မှာ  အခြေအနေပေါ်မူတည်ပြီး \fEn{early} \fCode{return} လုပ်မယ် ဆိုပါစို့။ \fCode{finally} မသုံးဘူးဆိုရင် \fCode{return} မတိုင်ခင် \fCode{print} တွေ အခုလို ထည့်ပေးရပါမယ်။ 

%
\begin{py}
def test_read():
    try:
        val = read_sensor()
        print("Value: " + str(val))
        if val < 3:
            # ß\fEn{Do not forget here}ß
            print("Don't skip this!!!")
            return
        elif val < 5:
            use(val)
        elif val < 8:
            # ...    
        ß\fEnEmp{etc.,}ß  
        # ß\fEn{Do not forget here}ß 
        print("Don't skip this!!!")
    except IOError as e:
        print("Error while reading!")
        # ß\fEn{Do not forget here}ß 
        print("Don't skip this!!!")
        return
\end{py}
%
တစ်နေရာရာမှာ ကျန်ခဲ့ပြီး မှားနိုင်ခြေ များတယ်။ ဒီအားနည်းချက်အပြင် ပိုကြီးတဲ့ ပြဿနာက ဘယ်ဟာက မဖြစ်မနေ လုပ်ဆောင်ရမဲ့ကိစ္စ လဲဆိုတာ  ကြည့်ရုံနဲ့ ကွဲကွဲပြားပြား မမြင်နိုင်တော့ဘူး။ နေရာအတော်များများမှာ ဖြန့်ပြီးရှိနေတယ်။  \fCode{finally} သုံးခြင်းအားဖြင့် ဒီပြဿနာကို ဖြေရှင်းနိုင်ပါတယ်။ 





\begin{mytcbox}
ဒေတာဘေ့စ်နဲ့ ဆက်သွယ် ဆောင်ရွက်ခြင်း၊ ဖိုင်ဖတ်ခြင်း/ရေးခြင်း၊ နက်ဝပ်ချိတ်ဆက်ခြင်း စတဲ့ကိစ္စတွေ လုပ်ဆောင်တဲ့အခါ အသုံးပြုထားတဲ့ ဖိုင်၊ ကွန်နက်ရှင် စတဲ့ \fEn{resource} တွေကို \fEn{release} လုပ်ပေးဖို့ အရေးကြီးပါတယ်။ ဒီလို မလုပ်ရင် ကွန်ပျူတာ စနစ်ရဲ့ \fEn{CPU cycles, memory} အစရှိတဲ့ \fEn{resource} တွေ အလဟဿ ပြုန်းတီးစေတဲ့အတွက် ပြဿနာရှိပါတယ်။
\end{mytcbox}
\vspace{-1\baselineskip}
%
\begin{multicols}{2}
  \begin{tikzpicture}[
    remember picture,
    overlay,
    annotation/.style={
      inner sep=0pt,
      outer sep=0pt,
      outer xsep=1mm,
      fill=yellow!80!black,
      text width=5cm
    },
    >={Stealth[inset=0pt, angle=30:7pt]}
  ]
  \node at ([xshift=1.3cm]pic cs:ch11f0) [circle, aspect=2, draw=black, fill=green!50, font=\scriptsize, line width=1pt, inner sep=1.5pt] {\huge\ding{51}};
  \draw[thin,rounded corners=5,fill=lightgray] ([yshift=1.5ex]pic cs:ch11f1) rectangle ([xshift=2cm]pic cs:ch11f2);
  \draw[thin,rounded corners=5,fill=darkgray] ([yshift=1.5ex]pic cs:ch11f3) rectangle ([xshift=2cm]pic cs:ch11f4);
  \draw[thin,rounded corners=5,fill=lightgray] ([yshift=1.5ex]pic cs:ch11f5) rectangle ([xshift=2cm]pic cs:ch11f6);
  \draw[thin,rounded corners=5,fill=lightgray] ([yshift=1.5ex]pic cs:ch11f7) rectangle ([xshift=2cm]pic cs:ch11f8);
  %\draw[->, thin] ([yshift=1.5cm,xshift=2.2cm]pic cs:ch11f2)  -- ([yshift=0.5ex,xshift=2.2cm] pic cs:ch11f2);
  \end{tikzpicture}
  %
  %
  \begin{py}
  try:
      ß$\ldots$ß
      ß\bfseries\fEnEmp{some\_fun()}\tikzmark{ch11f0}ß
      ß\tikzmark{ch11f1} ß
       ß$\ldots$ß
      ß\tikzmark{ch11f2}ß
  except:
      ß\tikzmark{ch11f3} ß
       ß$\ldots$ß
      ß\tikzmark{ch11f4}ß
  finally:
      ß\tikzmark{ch11f5} ß
       ß$\ldots$ß
      ß\tikzmark{ch11f6}ß
  
  ß\tikzmark{ch11f7} ß
   ß$\ldots$ß
  ß\tikzmark{ch11f8}ß
  \end{py}
  \columnbreak
%
% ---
%
%
\begin{tikzpicture}[
  remember picture,
  overlay,
  annotation/.style={
    inner sep=0pt,
    outer sep=0pt,
    outer xsep=1mm,
    fill=yellow!80!black,
    text width=5cm
  },
  >={Stealth[inset=0pt, angle=30:7pt]}
]
\node at ([xshift=1.3cm]pic cs:ch11e0) [starburst, draw=red, fill=yellow,
       starburst point height=0.4cm, font=\scriptsize, line width=1pt, inner sep=1.5pt] {\fEnBf{Bang!}};
\draw[thin,rounded corners=5,fill=darkgray] ([yshift=1.5ex]pic cs:ch11e1) rectangle ([xshift=2cm]pic cs:ch11e2);
\draw[thin,rounded corners=5,fill=lightgray] ([yshift=1.5ex]pic cs:ch11e3) rectangle ([xshift=2cm]pic cs:ch11e4);
\draw[thin,rounded corners=5,fill=lightgray] ([yshift=1.5ex]pic cs:ch11e5) rectangle ([xshift=2cm]pic cs:ch11e6);
\draw[thin,rounded corners=5,fill=lightgray] ([yshift=1.5ex]pic cs:ch11e7) rectangle ([xshift=2cm]pic cs:ch11e8);
%\draw[->, thin] ([yshift=1.5cm,xshift=2.2cm]pic cs:ch11e2)  -- ([yshift=0.5ex,xshift=2.2cm] pic cs:ch11e2);
\end{tikzpicture}
%
%
\begin{py}
try:
    ß$\ldots$ß
    ß\bfseries\fEnEmp{some\_fun()}\tikzmark{ch11e0}ß
    ß\tikzmark{ch11e1} ß
     ß$\ldots$ß
    ß\tikzmark{ch11e2}ß
except:
    ß\tikzmark{ch11e3} ß
     ß$\ldots$ß
    ß\tikzmark{ch11e4}ß
finally:
    ß\tikzmark{ch11e5} ß
     ß$\ldots$ß
    ß\tikzmark{ch11e6}ß

ß\tikzmark{ch11e7} ß
 ß$\ldots$ß
ß\tikzmark{ch11e8}ß
\end{py}
%
\end{multicols}
\noindent \fEn{Exception} မဖြစ်တဲ့ အခါနဲ့ ဖြစ်တဲ့အခါ \fCode{try...except...finally} အလုပ်လုပ်ပုံ နှိုင်းယှဉ်ပြထားတာပါ။ မီးခိုးရောင် အဖျော့ ဘလောက်တွေက လုပ်ဆောင်မဲ့ ဘလောက်တွေပါ။ မီးခိုးရင့်ရောင် ဘလောက်တွေကိုတော့ လုပ်ဆောင်မှာ မဟုတ်ပါဘူး။ စောစောကဖော်ပြခဲ့သလို \fEn{early} \fCode{return} ဖြစ်မယ်ဆိုလည်း \fCode{finally} အပိုင်း လုပ်ဆောင်ပြီးမှပဲ \fCode{return} ဖြစ်မယ်။ \fCode{finally} မဟုတ်တဲ့ အခြား ဘလောက်တွေကတော့  ၎င်းဘလောက် မတိုင်မီ \fEn{early} \fCode{return} ဖြစ်သွားရင်တော့ လုပ်ဆောင်မှာ မဟုတ်ပါဘူး။


%
\begin{py}
try:

except SomeException as err:

except SomeOtherException as err:

else:

finally:
\end{py}
%

\section{Exception Classes and Handling Multiple Exception}
ပုံမှန်မဟုတ်တဲ့ အခြေအနေလို့ ပြောတဲ့အခါ တစ်မျိုးတည်း မဟုတ်နိုင်ဘူးဆိုတာ သေချာပါတယ်။ ဖိုင်တစ်ခုကို ဖွင့်ဖို့ ကြိုးစားတဲ့အခါ အဲ့ဒီဖိုင်က မရှိတာဖြစ်နိုင်သလို၊ ဖိုင်ကရှိပေမဲ့ ခွင့်ပြုချက် \fEn{(permission)} မပေးထားလို့ ဖွင့်မရတာလည်း ဖြစ်နိုင်တယ်။ ဒီလို အကြောင်း အမျိုးမျိုးကြောင့် ဖြစ်ပေါ်နိုင်တဲ့ \fEn{exception} အမျိုးမျိုးကို ဖော်ပြဖို့အတွက် \fEn{Python} မှာ \fEn{built-in} \fEn{exception} ကလပ်စ်တွေရှိပါတယ်။ 

ဒီ \fEn{exception} ကလပ်စ်အားလုံးဟာ \fCode{BaseException} ရဲ့ \fEn{subclass} တွေပါ။  \fEn{Built-in} \fEn{exception} တွေ အပြင် လိုအပ်ရင် ကိုယ်တိုင် သတ်မှတ်ချင်လည်း ရတယ်။ ကိုယ်တိုင် သတ်မှတ်တာဆိုတော့ \fEn{user-defined exception} ပေါ့။ ကိုယ်ပိုင် သတ်မှတ်မယ်ဆိုရင် \fCode{BaseException} ကနေ  တိုက်ရိုက် \fEn{inherit} မလုပ်ရဘူး။ \fCode{Exception} ကလပ်စ်ကို \fEn{inherit} လုပ်ရမှာပါ။ 

%
\begin{py}
class InvalidAgeException(Exception):
    """Raised when age is not valid"""
    pass
\end{py}
\begin{py}
class BalanceNotEnoughException(Exception):
    """Raised when account balance is not enough"""
    pass
\end{py}
%

\subsection*{Handling Multiple Exception}
ဖြစ်ပေါ်တဲ့ \fEn{exception} ပေါ်မူတည်ပြီး သင့်တော်တဲ့ နည်းလမ်းနဲ့ \fEn{handle} လုပ်ရလေ့ ရှိတယ်။ အောက်ပါ \fCode{fun\_c} က \fCode{FileNotFoundError} (သို့) \fCode{PermissionError} တက်နိုင်ပါတယ်။
%
\begin{py}
def fun_c():
    print('Starting fun_c...')
    if random.uniform(0.0, 1.0) <= 0.25:
        raise FileNotFoundError("File not found!")
    if random.uniform(0.0, 1.0) > 0.75:
        raise PermissionError("No permission!")
    print('fun_c ends!')
\end{py}
%
\fCode{try} ထဲမှာ ခေါ်ထားတဲ့ ဖန်ရှင်တွေက 

%
\begin{py}
def fun_b():
    try:
        fun_c()
    except FileNotFoundError as e:
        print("Handle FileNotFoundError")
    except PermissionError as e:
        print("Handle PermissionError")
\end{py}
%
%
\begin{py}
def fun_b1():
    try:
        fun_c()
    except (FileNotFoundError, PermissionError) as e:
        print("Handle both FileNotFoundError and PermissionError")
\end{py}
%


\begin{figure}[tbh!]
\begin{tikzpicture}[
    font=\small,
    level 1/.style = { level distance = 2cm,
                     sibling distance = 2.8cm },
    level 2/.style = { level distance = 2cm,
                     sibling distance = 2.9cm },
    >={Stealth[inset=0pt, angle=30:7pt]}]
  

\node [draw, rectangle, thick, rounded corners=3, minimum height = 1cm, minimum width=1.5cm] {\fEn{BaseException}}
    [thick, <-]
    child {node [draw, rectangle, thick, rounded corners=3, minimum height = 1cm, minimum width=1.5cm, text width=2cm,align=center] {\fEn{\hspace{0pt}SystemExit}}}
    child {node [draw, rectangle, thick, rounded corners=3, minimum height = 1cm, minimum width=1.5cm, text width=2cm,align=center] {\fEn{\hspace{0pt}Exception}}
        child{node [draw, rectangle, thick, rounded corners=3, minimum height = 1cm, text width=1.8cm,align=center] {\fEn{\hspace{0pt}Lookup\\ Error}}}
        child{node [draw, rectangle, thick, rounded corners=3, minimum height = 1cm, text width=1.8cm,align=center] {\fEn{\hspace{0pt}Arithmetic\\ Error}}
            [sibling distance=2.5cm, level distance = 2cm]
            child{node [draw, rectangle, thick, rounded corners=3, minimum height = 1cm, text width=2.2cm,align=center] {\fEn{\hspace{0pt}ZeroDivision\\ Error}}}
            child{node [draw, rectangle, thick, rounded corners=3, minimum height = 1cm, text width=1.5cm,align=center] {\fEn{\hspace{0pt}Overflow\\ Error}}}
            child{node {$\ldots$}}       
        }
        child{node [draw, rectangle, thick, rounded corners=3, minimum height = 1cm, text width=1.2cm,align=center] {\fEn{\hspace{0pt}Value\\ Error}}}
        child{node [draw, rectangle, thick, rounded corners=3, minimum height = 1cm, text width=1.2cm,align=center] {\fEn{\hspace{0pt}OS\\ Error}}
            [sibling distance=2.8cm,level distance=3cm]
            child{node [draw, rectangle, thick, rounded corners=3, minimum height = 1cm, text width=2.3cm,align=center] {\fEn{\hspace{0pt}FileNot\\ FoundError}}}
            child{node [draw, rectangle, thick, rounded corners=3, minimum height = 1cm, text width=2.2cm,align=center] {\fEn{\hspace{0pt}Permission\\ Error}}}
            child{node {$\ldots$}}
        }
        child{node {$\ldots$}}
    }
    child {node {$\ldots$}};
\end{tikzpicture}
\caption{Exception Hierarchy} 
\label{fig:exceptionhier}
\end{figure}


% https://tex.stackexchange.com/questions/35526/tikz-tree-sibling-distance
\tikzset{
  my node style/.style={
    font=\small,
    rectangle,
    draw=black,
    fill=lightgray!40,
    rounded corners,
    minimum width=1cm,
    minimum height=1cm,
    thick,
    align=center
  }
}
\forestset{
  my tree style/.style={
    for tree={
      parent anchor=south,
      child anchor=north,
      l sep+=5pt,
      my node style,
      edge={thick},
      edge path={
        \noexpand\path [draw, \forestoption{edge}] (!u.parent anchor) -- +(0,-7.5pt) -| (.child anchor)\forestoption{edge label};
      },
      if n children=3{
        for children={
          if n=2{calign with current}{}
        }
      }{},
      delay={if content={}{shape=coordinate}{}}
    }
  }
}
\centering

\begin{forest}
  my tree style
  [\fEn{BaseException}
    [\fEn{SystemExit}]
    [\fEn{KeyboardInterrupt}]
    [\fEn{Exception}
        [\fEn{Attribute}\\ \fEn{Error}]
        [\fEn{Arithmetic}\\ \fEn{Error}
            [\fEn{ZeroDivision}\\ \fEn{Error}]
            [\fEn{FloatingPoint}\\ \fEn{Error}]
            [\fEn{Overflow}\\ \fEn{Error}]
        ]
        [\fEn{EOF}\\ \fEn{Error}]
        [\fEn{Name}\\ \fEn{Error}]
        [\fEn{OS}\\ \fEn{Error}
            [[
            [\fEn{FileNotFound}\\ \fEn{Error}]
            [\fEn{Interrupted}\\ \fEn{Error}]
            [\fEn{Permission}\\ \fEn{Error}]
            [\fEn{TimeOut}\\ \fEn{Error}]
            ]]
        ]
        [\fEn{Lookup}\\ \fEn{Error}
            [\fEn{Index}\\ \fEn{Error}]
            [\fEn{Key}\\ \fEn{Error}]
        ]
        [\fEn{Type}\\ \fEn{Error}]
        [\fEn{Value}\\ \fEn{Error}]
    ]
    [\fEn{GeneratorExit}]
  ]
\end{forest}



