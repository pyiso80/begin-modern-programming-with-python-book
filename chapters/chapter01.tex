\chapter{စက်ရုပ်ကားရဲလ်ဖြင့် ပရိုဂရမ်းမင်းမိတ်ဆက်} \label{ch:ch01}
ကွန်ပျူတာတွေဟာ သက်မဲ့ စက်ပစ္စည်းတွေပါပဲ။ ကားတို့၊ လေယာဉ်တို့နဲ့ မတူတာက ကွန်ပျူတာတွေဟာ စက်ချည်းသက်သက် ဘာအစွမ်းမှ မယ်မယ်ရရ မရှိဘူး။ ဒါပေမဲ့ ဆောင်ရွက်လိုတဲ့ ကိစ္စအဝဝအတွက် ပရိုဂရမ်အမျိုးမျိုး ထည့်ပေးလိုက်တဲ့အခါမှာ သူ့ရဲ့အစွမ်းက အတိုင်းအဆမဲ့ပဲ။ နေရာမျိုးစုံ၊ နယ်ပါယ်မျိုးစုံမှာ အကူအညီပေးနိုင်တဲ့ စွယ်စုံသုံး ပစ္စည်းတစ်ခုဖြစ်သွားတယ်။ ဂီတသံစဉ်တွေကို ဖွင့်ပေးနိုင်သလို အသံလည်းသွင်းပေးနိုင်တယ်။ ရုပ်ရှင်တည်းဖြတ် လုပ်ချင်တာလား။ ပြဿနာမရှိဘူး၊ ကူညီပေးနိုင်တယ်။ နျူကလီး\allowbreak ယား ဓါတ်ပေါင်းဖိုတွေကို စီမံနိုင်သလို မောင်းသူမဲ့ ဒုံးပျံတွေကိုလည်း ပဲ့ထိန်းပေးနိုင်တယ်။ 

ကျွန်တော်တို့တွေ နိစ္စဓူဝ အသုံးပြုနေကြတဲ့ ကား၊ စမတ်ဖုန်း၊ လက်ပါတ်နာရီ၊ မိုက်ခရိုဝေဖ့်မီးဖို၊ အဝတ်လျှော်စက် စတဲ့ စက်ပစ္စည်း အမျိုးမျိုးဟာလည်း ကွန်ပျူတာတွေနဲ့ မကင်းပြန်ပါဘူး။ “ကွန်ပျူတာနည်းပညာ အကူအညီမပါတဲ့ ခေတ်မီဆန်းသစ်တီထွင်မှုဆိုတာ မရှိဘူး” လို့ ဆိုနိုင်ပါတယ်။

တစ်ချက်တစ်ချက် ရိုက်ခတ်လိုက်တဲ့ ကွန်ပျူတာနည်းပညာ လှိုင်းလုံးကြီးတွေဟာ ကမ္ဘာတစ်ဝှမ်းလုံး ပုံစံပြောင်းသွားလောက်အောင် အဟုန်ပြင်းထန်လှတယ်။ ဘီလီယံနဲ့ချီတဲ့ လူတွေ ဆိုရှယ်မီဒီယာတွေပေါ်ကနေ ရုပ်သံတွေနဲ့ ချိတ်ဆက်ပြောဆိုဆက်သွယ်လို့ ရစေတာဟာလည်း ကွန်ပျူတာစနစ်တွေပါပဲ။ \fEn{Artificial Intelligence (AI)} နည်းပညာကြောင့် သက်ရှိတွေမှာပဲတွေ့ရတဲ့ ညာဏ်ရည်မျိုးကို ကွန်ပျူတာတွေမှာလည်း တွေ့လာရပါပြီ။ သင်္ချာပုစ္ဆာတွေ ဖြေရှင်းခြင်း၊ စစ်တုရင်ထိုးခြင်း စတဲ့ကိစ္စမျိုးတွေအပြင် ပန်းချီဆွဲခြင်း၊ ကဗျာရေးစပ်ခြင်း၊ သီချင်းရေးဖွဲ့ခြင်း ကဲ့သို့ အနုပညာဖန်တီးမှုတွေကိုပါ \fEn{AI} က လုပ်ဆောင်ပေးနိုင်ပါတယ်။ နှစ်ဆယ့်တစ်ရာစုရဲ့ အထူးခြားဆုံး \fEn{AI} နည်းပညာလှိုင်းဟာ အရှိန်အဟုန်ပြင်းပြင်း ရိုတ်ခတ်ဖို့ အားယူစ ပြုနေပါပြီ။

‘ကွန်ပျူတာ’ လို့ပြောတဲ့အခါ စက်ပစ္စည်းသက်သက် မဟုတ်ဘဲ ကွန်ပျူတာမှတ်ညာဏ်ထဲက ပရိုဂရမ်တွေလည်း ပါဝင်တယ်ဆိုတာ သတိချပ်ရပါမယ်။ ကွန်ပျူတာတွေ တစ်စုံတစ်ရာ စွမ်းဆောင်နိုင်စေတဲ့  ပရိုဂရမ်တွေ ရေးတဲ့အလုပ်ကို ပရိုဂရမ်းမင်း\fEn{(Programming)} လို့ခေါ်တယ်။