\chapter{ From Classes to Objects (ကလပ်စ်များမှ အော့ဘ်ဂျက်များသို့)}

အခန်း (\fRefNo{\ref{ch:ch06objs}}) မှာ အော့ဘ်ဂျက်တချို့နဲ့ မိတ်ဆက်ပေးခဲ့တယ်။ \fCode{date}\fEn{,} \fCode{list}\fEn{,} \fCode{dict} စသည်ဖြင့်။ အော့ဘ်ဂျက်တွေကို ကလပ်စ်တစ်ခုကနေ တည်ဆောက်ယူရတာပါ။ \fCode{date} အော့ဘ်ဂျက်တွေကို \fCode{date} ကလပ်စ်၊ \fCode{Fraction} အော့ဘ်ဂျက်တွေကို \fCode{Fraction} ကလပ်စ်ကနေ ဖန်တီးယူတယ်။  အော့ဘ်ဂျက်တွေကို \fEn{class} တစ်ခုရဲ့ \fEnEmp{instance} လို့လည်းခေါ်တယ်။ \fCode{Fraction} အော့ဘ်ဂျက်ကို \fCode{Fraction} ကလပ်စ် \fEn{instance} လို့လည်း ပြောလေ့ရှိတယ်။ အော့ဘ်ဂျက်တွေဟာ \fEn{class instance} တွေဖြစ်တယ်။ (အကြမ်းဖျဉ်းအားဖြင့် အော့ဘ်ဂျက်နဲ့ \fEn{instance}  တူတယ်ပြောလို့ရပေမဲ့ အနက်အဓိပ္ပါယ်အရ အနုစိတ် ကွာဟချက်တွေ ရှိတာကိုလည်း \fEn{inheritance} အကြောင်း လေ့လာတဲ့အခါ တွေ့ရပါမယ်)။

ဒီအခန်းမှာ ကလပ်စ်အကြောင်းကို လေ့လာကြမှာပါ။ ကိုယ်ပိုင် ကလပ်စ်သတ်မှတ်ပြီး အဲ့ဒီကလပ်စ်ကနေ အော့ဘ်ဂျက်တွေ တည်ဆောက် ယူနိုင်တော့မှာဖြစ်တယ်။ အခြားသူတွေရေးထားပေးတဲ့ လိုက်ဘရီဖန်ရှင်တွေကို အသုံးပြုရာကနေ နောက်ပိုင်းမှာ ကိုယ်ပိုင် ဖန်ရှင်တွေ သတ်မှတ်လာနိုင်တာဟာ အဆင့်တစ်ဆင့် ပိုမြင့်လာသလိုပါပဲ။ ကိုယ်ပိုင်ကလပ်စ် ဒီဇိုင်းပြုလုပ် ဖန်တီးနိုင်လာတာဟာလည်း နောက်တစ်ဆင့် ထပ်မြင့်လာတယ်လို့ ဆိုရမှာပါ။ ပရိုဂရမ်တွေကို အမှားနည်းအောင်၊ ပြင်ဆင်ရလွယ်ကူအောင် စနစ်တကျ ဒီဇိုင်းလုပ် တည်ဆောက်လို့ရလာမှာ ဖြစ်ပါတယ်။

ကလပ်စ်ကို \fEn{abstraction mechanism} တစ်မျိုးလို့လည်း ရှုမြင်နိုင်တယ်။ \fEn{Abstraction} လုပ်တယ်ဆိုတာ ဘယ်လိုတည်ဆောက်ထားလဲ သိစရာမလိုဘဲ အသုံးပြုလို့ရစေတာကို ဆိုလိုတာ။ ဖန်ရှင်တွေနဲ့ \fEn{abstraction} လုပ်တာကို တွေ့ခဲ့ကြပြီးပါပြီ။ ကလပ်စ်တွေဟာလည်း \fEn{abstraction} အတွက် အထောက်\allowbreak အကူပြုတဲ့ နည်းလမ်းတစ်မျိုး \fEn{(mechanism)} ပါပဲ။ အများကြီး ပိုအစွမ်းထက်တဲ့ နည်းလမ်းလို့ ဆိုရမှာပါ။ အစွမ်းထက်လာတာနဲ့အမျှ သဘောတရားအရရော အသုံးချတဲ့အပိုင်းမှာပါ ပိုပြီးရှုပ်ထွေးခက်ခဲနိုင်ပါတယ်။ ဒီဇိုင်းပိုင်းအတွက် ကျယ်ကျယ်ပြန့်ပြန့် ဆက်လက်လေ့လာဖို့ လိုအပ်မှာဖြစ်ပြီး အတွေ့အကြုံနဲ့ပါ ပေါင်းစပ်ယူရမှာပါ။ စာချည်းဖတ်ရုံနဲ့ မရနိုင်ဘူး (စာရေးသူ ကိုယ်တွေ့ အတွေ့အကြုံအရ ကိုယ်ပိုင်အမြင်ကို ပြောခြင်းသာ)။  ဒါတွေက နောက်ပိုင်း ဆက်လက်လေ့လာဖို့အတွက်ပေါ့။ အခုတော့ အများကြီးမပြောတော့ဘဲ ကလပ်စ်တွေအကြောင်း စလိုက်ကြရအောင် $\ldots$


\section{\fSecCodeBf{Account} Class (အကောင့် ကလပ်စ်)}
ဘဏ်အကောင့်တွေကို \fCode{Account} အော့ဘ်ဂျက်နဲ့ ဖော်ပြမယ်ဆိုပါစို့။  အကောင့်နံပါတ်၊ ပိုင်ရှင်၊ လက်ကျန်\allowbreak ငွေ \fEn{(balance)} စတဲ့ အချက်အလက်တွေဟာ အပြင်က တကယ့် ဘဏ်အကောင့်တစ်ခုအတွက် အရေးပါပါတယ်။ မှတ်သား သိမ်းဆည်းထားရတယ်။  ဒါကြောင့် အပြင်က တကယ့် ဘဏ်အကောင့်ကို ထင်ဟပ်ဖော်ပြတဲ့ ဆော့ဖ်ဝဲ အော့ဘ်ဂျက်တွေမှာလည်း ဖော်ပြပါ အချက်\allowbreak အလက်တွေ ပါဝင်သင့်တယ်ဆိုရင် ကျိုးကြောင်းဆီလျော်တယ်ပဲ ယူဆရမှာပါ။ အကောင့်နဲ့ ပါတ်သက်ပြီး အခြား အရေးပါတဲ့ အချက်\allowbreak အလက်တွေ ဒီ့ထက်မက ရှိပါတယ်။ ဥပမာ အကောင့် အမျိုးအစား \fEn{(saving, current)} ၊ ဖွင့်ရက် \fEn{(open date)}၊ ဘဏ်ခွဲအမည်၊ ငွေသွင်း/ငွေထုတ်/ငွေလွှဲမှတ်တမ်း စတဲ့အချက်တွေ ရှိပါတယ်။ တကယ့်လက်တွေ့မှာ လိုအပ်မှာဖြစ်ပေမဲ့ အခုဥပမာမှာ ရိုးရိုးရှင်းရှင်းဖြစ်ဖို့ အဲဒါတွေ ထည့်မစဉ်းစားဘဲ ချန်ခဲ့ရအောင်။ ကလပ်စ်ရဲ့ သဘောတရားကို ရှင်းပြဖို့အတွက် ပထမသုံးချက်နဲ့ လုံလောက်ပါတယ်။

ကေသီ့မှာ  အကောင့်နံပါတ် \fEn{0086-6002-1111} နဲ့  လက်ကျန်ငွေ တစ်သိန်းရှိတဲ့ အကောင့်တစ်ခု ရှိတယ် ဆိုပါစို့။ \fCode{Account} ကလပ်စ်သာ ရှိမယ်ဆိုရင် အဲဒီ ကေသီ့အကောင့်ကို ကိုယ်စားပြုတဲ့ အော့ဘ်ဂျက်ကို အခုလို 
%
\begin{py}
acc1 = Account('Kathy',
               '0086-6002-1111',
               Decimal("100000.00"))
\end{py}
%
ဖန်တီးယူလို့ ရမှာပါ။ ထိုနည်းတူစွာ ငွေလက်ကျန် လေးသိန်းခွဲ၊ နံပါတ် \fEn{0086-6002-2211} နဲ့  စန္ဒီ့အကောင့် အော့ဘ်ဂျက်ကို ဒီလို
%
\begin{py}
acc2 = Account('Sandy',
               '0086-6002-2211',
               Decimal("450000.00"))
\end{py}
% 
ဖန်တီးယူနိုင်မယ်။ ဒါက \fCode{Account} ကလပ်စ်သာ ရှိခဲ့ရင် အော့ဘ်ဂျက် ဘယ်လို ဖန်တီးယူရမလဲ စဉ်းစားကြည့်တာပေါ့။  ကလပ်စ်မရှိတဲ့အတွက် အမှန်တကယ်တော့ မရသေး။ 


\fCode{date}\fEn{,} \fCode{list}\fEn{,} စတဲ့ အော့ဘ်ဂျက်တွေ အပေါ်မှာ အော်ပရေးရှင်းတွေ လုပ်ဆောင်လို့ရတာ တွေ့ခဲ့ရတယ်။ \fCode{Account} အော့ဘ်ဂျက်တွေမှာရော ဘယ်လို အော်ပရေးရှင်းတွေ လုပ်ဆောင်လို့ရသင့်လဲ။ အပြင်မှာ အကောင့်တစ်ခုကနေ ငွေထုတ်လို့ရတယ်၊ အကောင့်ထဲကို ငွေသွင်းလို့ရပါတယ်။ \fCode{Account} အော့ဘ်ဂျက်တွေမှာလည်း ဒါတွေလုပ်လို့ရသင့်တာပေါ့
%
\begin{py}
acc1.deposit(Decimal("50000.00"));
acc2.withdraw(Decimal("70000.00"));
\end{py}
%
ဒီအော်ပရေးရှင်းတွေက အော့ဘ်ဂျက်တွေအပေါ် ဘယ်လိုသက်ရောက်မှုရှိမလဲ။ ကေသီ့အကောင့် လက်\allowbreak ကျန်ငွေက တစ်သိန်းခွဲ ဖြစ်သွားသင့်တယ်။ စန္ဒီ့အကောင့်က သုံးသိန်းရှစ်သောင်း ဖြစ်သင့်တယ်။ ဒီလို အော်ပရေးရှင်းတွေဟာလည်း ကလပ်စ်ပေါ်မှာ မူတည်တယ်။ ကလပ်စ်က ထောက်ပံ့ပေးထားမှပဲ ရမယ်။ ကလပ်စ်သည်သာ အခရာလို့ ပြောရမှာပါ။

ကလပ်စ်တစ်ခုဟာ အော့ဘ်ဂျက်တွေမှာ ပါရှိရမဲ့ အချက်အလက်တွေနဲ့ ၎င်းအော့ဘ်ဂျက်တွေအပေါ်မှာ လုပ်\allowbreak ဆောင်နိုင်တဲ့ အော်ပရေးရှင်းတွေကို သတ်မှတ်ပေးပါတယ်။ ဥပမာ အနေနဲ့ \fCode{Account} ကလပ်စ် ဘယ်လိုသတ်မှတ်ရမလဲ ကြည့်ရအောင် $\ldots$ 


%
\begin{py}
# File: account.py
from decimal import *


class Account:
    def __init__(self, holder, acc_number, balance):
        self.holder = holder
        self.acc_number = acc_number
        self.balance = balance

    def deposit(self, amt):
        if amt <= Decimal(0.00):
            raise ValueError('Invalid amount for deposit!')
        self.balance += amt

    def withdraw(self, amt):
        if amt > self.balance:
            raise ValueError('Not enough balance!')
        self.balance -= amt
\end{py}
%

‘\fCode{Account} ကလပ်စ် သတ်မှတ်ပါမယ်’ လို့ ပြောဖို့အတွက် \fCode{class Account:} နဲ့ စရပါတယ်။ သူ့အောက်မှာရှိတာက \fCode{Account} ကလပ်စ်ရဲ့ ဘော်ဒီ \fEn{(body)} ပါ။ \fEn{Body} ဆိုတာ \fEn{block} ကို ပြောတာပါပဲ။ ကလပ်စ်ဘော်ဒီထဲမှာ ဖန်ရှင် သုံးခု သတ်မှတ်ထားတာ တွေ့ရမယ်။ သုံးခုလုံးမှာ ပထမ ပါရာမီတာက \fCode{self} ဖြစ်နေတာကို သတိပြုမိမှာပါ။ သူ (\fCode{self}) က လက်ရှိအော့ဘ်ဂျက် \fEn{(current object)} လို့ ဆိုလိုတာဖြစ်ပြီး သိပ်မကြာခင် သူ့အဓိပ္ပါယ်ကို ရှင်းပြပါမယ်။

 

\mintinline{text}|__init__| ဖန်ရှင်က အော့ဘ်ဂျက်မှာ ပါဝင်တဲ့ ဗေရီရေဘဲလ်တွေ သတ်မှတ်ပေးတယ်။ ၎င်းတို့ရဲ့ ကနဦး တန်ဖိုးကိုလည်း ဒီဖန်ရှင်က  \fEn{initialize} (စထည့်) လုပ်ပေးရတာပါ။ အော့ဘ်ဂျက်ရဲ့ ကနဦး အခြေအနေ \fEn{(initial state)} ကို စတင်ပေးတဲ့ ဖန်ရှင်ဖြစ်တာကြောင့် \fEn{initializer} လို့ ခေါ်လေ့ရှိတယ်။ \fCode{Account} ကလပ်စ်ကို ကြည့်ရင် ဒီဖန်ရှင်မှာ \mintinline{text}|self.holder|\fEn{,} \mintinline{text}|self.acc_number|\fEn{,} \mintinline{text}|self.balance| စတဲ့ အော့ဘ်ဂျက်ရဲ့ အချက်အလက်တွေနဲ့ သက်ဆိုင်တဲ့ ဗေရီရေဘဲလ်တွေ ကြေငြာထားတာ တွေ့ရမှာပါ။ ဒီနေရာမှာ \fEn{dot} အမှတ်အသား အဓိပ္ပါယ်ကို ‘၏/ရဲ့’ လို့ ယူဆရင် ဖတ်ရတာ အဆင်ပြေတယ်။ \fCode{self.holder} ကို ‘လက်ရှိအော့ဘ်ဂျက်ရဲ့ \fCode{holder}’၊ \fCode{self.balance} ကို ‘လက်ရှိအော့ဘ်ဂျက်ရဲ့ \fCode{bal\allowbreak ance}’ လို့ ဖတ်နိုင်တယ်။ \mintinline{text}|__init__| ရဲ့ ထူးခြားချက်က သူ့ကို တိုက်ရိုက် ခေါ်သုံးရလေ့မရှိဘူး။ အော့ဘ်ဂျက် တည်ဆောက်တဲ့ ဖြစ်စဉ်ရဲ့ နောက်ဆုံးအဆင့်မှာ \fEn{Python} က အလိုအလျှောက် ခေါ်ပေးတဲ့ ဖန်ရှင်ဖြစ်တယ်။


\fCode{Account} \fEn{instance} တစ်ခုကို ဖန်တီးယူမယ်ဆိုရင်  အခုလို ရေးရမှာပါ
%
\begin{py}
Account('Amy', '0086-6002-2233', Decimal('350_000.00'))
\end{py}
%
ဒီအခါမှာ အော့ဘ်ဂျက် တည်ဆောက်တဲ့ ဖြစ်စဉ်ကို စတင်လုပ်ဆောင်ပေးမှာ ဖြစ်တယ်။ ထည့်ပေးထားတဲ့ တန်ဖိုးတွေက \mintinline{text}|__init__| ဖန်ရှင်အတွက် ဖြစ်တယ်။ \fCode{self} နောက်က ပါရာမီတာတွေအတွက်ပဲ ထည့်ပေးရပြီး \fCode{self} အတွက် တန်ဖိုးထည့်မပေးရပါဘူး (\fCode{self} ဟာ ကလပ်စ်ရေးတဲ့သူအတွက် သီးသန့်ဖြစ်ပြီး ကလပ်စ်ဘော်ဒီထဲမှာပဲ အသုံးပြုရတာပါ)။ ဒီတန်ဖိုးတွေက  လက်ရှိအော့ဘ်ဂျက်  ဗေရီရေဘဲလ်တွေရဲ့ ကနဦး တန်ဖိုးတွေဖြစ်သွားမှာပါ
%
\begin{py}
self.holder = holder
self.acc_number = acc_number
self.balance = balance
\end{py}
%
ဒီသုံးကြောင်းအရ လက်ရှိအော့ဘ်ဂျက်ရဲ့ \fCode{holder} က \fCode{'Amy'} ဖြစ်ပါမယ်။ လက်ရှိအော့ဘ်ဂျက် \fCode{balance} နဲ့ \fCode{acc\_number} က \fCode{Decimal('350\_000.00')} နဲ့ \fCode{'0086-6002-2233'} အသီးသီး ဖြစ်ပါမယ်။ ပုံ (\fRefNo{\ref{fig:AmyAccObj1}}) မှာလို မြင်ကြည့်နိုင်ပါတယ်

\begin{figure}[tbh!]
\definecolor{myFieldFill}{HTML}{F0F6FD}
\definecolor{myFieldBorder}{HTML}{212121}
\tikzset{myFieldStyle/.style={draw=myFieldBorder,fill=myFieldFill,semithick}}
\def\fObjFigAttrValWidth{3}
\def\fObjFigAttrValHeight{1}
\newcommand{\figpctw}{0.48}
\begin{tikzpicture}
    %\draw[step=.5cm,gray,very thin] (-1.4,-1.4) grid (1.4,1.4);
    %\filldraw[color=red!20] (0,0) circle (1ex);
    \draw[myFieldStyle,yshift=\fObjFigAttrValHeight*2cm] (0,0)rectangle node{\fEn{Amy}} (\fObjFigAttrValWidth,\fObjFigAttrValHeight);
    \draw[myFieldStyle,yshift=\fObjFigAttrValHeight*1cm] (0,0)rectangle node{\fEn{0086-6002-2233}} (\fObjFigAttrValWidth,\fObjFigAttrValHeight);
    \draw[myFieldStyle,yshift=\fObjFigAttrValHeight*0cm] (0,0)rectangle node{\fEn{350,000.00}} (\fObjFigAttrValWidth,\fObjFigAttrValHeight);
   
    \draw (0,0.5*\fObjFigAttrValHeight) node[anchor=east, yshift=\fObjFigAttrValHeight*2cm]{\fEn{holder}}
          (0,0.5*\fObjFigAttrValHeight) node[anchor=east, yshift=\fObjFigAttrValHeight*1cm]{\fEn{accountNumber}}
          (0,0.5*\fObjFigAttrValHeight) node[anchor=east, yshift=\fObjFigAttrValHeight*0cm]{\fEn{balance}};
    \draw (0.5*\fObjFigAttrValWidth, -0.5*\fObjFigAttrValHeight) node{\fEnBf{(Account)}};   
\end{tikzpicture}
\caption{အေမီ့ Account အော့ဘ်ဂျက် (\fCptCodeBf{acc3})}
\label{fig:AmyAccObj1}
\end{figure}

ဒီအော့ဘ်ဂျက်ကို ရည်ညွှန်းအသုံးပြုနိုင်ဖို့ဆိုရင် ထုံးစံအတိုင်း ဗေရီရေဘဲလ်နဲ့ အဆိုင်းမန့် လုပ်ထားဖို့ လိုပါတယ်။ 
%
\begin{py}
acc3 = Account('Amy', '0086-6002-2233', Decimal('350_000.00'))
\end{py}
%
\betweenminted{\medskipamount}
%
\begin{py}
print(acc3.holder)      # Amy 
print(acc3.acc_number)  # 0086-6002-2233
print(acc3.balance)     # 350000.00
\end{py}
%
ဒီနေရာမှာ \fCode{self} အကြောင်းကို ရှင်းပြဖို့ လိုလာပြီ။ လက်ရှိအသုံးပြုနေတဲ့ အေမီ့ \fCode{Account} အော့ဘ်ဂျက်ဟာ ‘လက်ရှိအော့ဘ်ဂျက်’ \fCode{self} ပဲ ဖြစ်တယ်။ ဒါကြောင့် အခုကိစ္စမှာ \fCode{acc3} နဲ့ \fCode{self} နှစ်ခုလုံးဟာ ‘လက်ရှိအော့ဘ်ဂျက်’ (တစ်ခုတည်း) ပဲ။ \fCode{acc3.holder} နဲ့ \fCode{self.holder} နှစ်ခုလုံးက လက်ရှိအော့ဘ်ဂျက်ရဲ့ \fCode{holder} ကို ဆိုလိုတာ ဖြစ်တယ်။ \fCode{holder} က \fCode{'Amy'} ပါ။ ဒီသဘောအတိုင်း \fCode{acc3.balance} ရော \fCode{self.balance} ပါ လက်ရှိအော့ဘ်ဂျက်ရဲ့ \fCode{balance} ကို ဆိုလိုတာ။ \fCode{Decimal("350\_000.00")} ဖြစ်ပါမယ်။ လက်ရှိအော့ဘ်ဂျက်က ပြောင်းသွားရင်ရော ဘယ်လိုဖြစ်မလဲ။
%
\begin{py}
acc1 = Account('Kathy', '0086-6002-1111', Decimal("100000.00"))
\end{py}
%
အခုလက်ရှိ အော့ဘ်ဂျက်က ကေသီ့ \fCode{Account} အော့ဘ်ဂျက် ဖြစ်သွားပြီ။ ဒီအချိန်မှာ \fCode{self.holder} က \fCode{'Kathy'}။ \fCode{acc1.holder} လည်း ဒါပဲဖြစ်မယ်။ \fCode{acc\_number} နဲ့ \fCode{balance} တို့ကိုလည်း အလားတူ စဉ်းစားရမှာ ဖြစ်တယ်။ \fCode{'0086-6002-1111'} နဲ့ \fCode{Decimal("100000.00")} ဖြစ်ပါမယ်။ အခုလက်ရှိ အော့ဘ်ဂျက်ဟာ သီးခြားတည်ရှိနေတဲ့ အော့ဘ်ဂျက်တစ်ခုဖြစ်ပြီး အခုလို မြင်ကြည့်ရမှာပါ

\begin{figure}[tbh!]
\definecolor{myFieldFill}{HTML}{F0F6FD}
\definecolor{myFieldBorder}{HTML}{212121}
\tikzset{myFieldStyle/.style={draw=myFieldBorder,fill=myFieldFill,semithick}}
\def\fObjFigAttrValWidth{3}
\def\fObjFigAttrValHeight{1}
\newcommand{\figpctw}{0.48}
\begin{tikzpicture}
    %\draw[step=.5cm,gray,very thin] (-1.4,-1.4) grid (1.4,1.4);
    %\filldraw[color=red!20] (0,0) circle (1ex);
    \draw[myFieldStyle,yshift=\fObjFigAttrValHeight*2cm] (0,0)rectangle node{\fEn{Kathy}} (\fObjFigAttrValWidth,\fObjFigAttrValHeight);
    \draw[myFieldStyle,yshift=\fObjFigAttrValHeight*1cm] (0,0)rectangle node{\fEn{0086-6002-1111}} (\fObjFigAttrValWidth,\fObjFigAttrValHeight);
    \draw[myFieldStyle,yshift=\fObjFigAttrValHeight*0cm] (0,0)rectangle node{\fEn{100,000.00}} (\fObjFigAttrValWidth,\fObjFigAttrValHeight);
   
    \draw (0,0.5*\fObjFigAttrValHeight) node[anchor=east, yshift=\fObjFigAttrValHeight*2cm]{\fEn{holder}}
          (0,0.5*\fObjFigAttrValHeight) node[anchor=east, yshift=\fObjFigAttrValHeight*1cm]{\fEn{accountNumber}}
          (0,0.5*\fObjFigAttrValHeight) node[anchor=east, yshift=\fObjFigAttrValHeight*0cm]{\fEn{balance}};
    \draw (0.5*\fObjFigAttrValWidth, -0.5*\fObjFigAttrValHeight) node{\fEnBf{(Account)}};   
\end{tikzpicture}
\caption{ကေသီ့ Account အော့ဘ်ဂျက် (\fCptCodeBf{acc1})}
\label{fig:KathyAccObj1}
\end{figure}
ဆိုလိုတာက အေမီ့ \fCode{Account} အော့ဘ်ဂျက်မှာ အကောင့်ပိုင်ရှင်၊ နံပါတ်နဲ့ လက်ကျန်ငွေအတွက် သူ့ကိုယ်ပိုင် ဗေရီရေဘဲလ်သုံးခု ရှိနေမှာဖြစ်ပြီး ကေသီ့ \fCode{Account} အော့ဘ်ဂျက်ကလည်း သူ့ဟာနဲ့သူ သီးခြား သုံးခု ရှိနေမှာပါ။

အခုလောက်ဆိုရင် \fCode{self} ရဲ့ သဘောကို နားလည်လောက်ပါပြီ။ ကလပ်စ်သတ်မှတ်တဲ့အခါ အော့ဘ်ဂျက်တစ်ခုစီမှာ သီးခြားကိုယ်ပိုင် ပါရှိမဲ့ ဗေရီရေဘဲလ်တွေကို \fEn{dot} အမှတ်အသားအသုံးပြုပြီး \fCode{self} နဲ့ ရည်ညွှန်းရပါတယ်။ \fCode{deposit} နဲ့ \fCode{withdraw} ကို ဆက်ကြည့်ရအောင်။

%
\begin{py}
# File: account.py
class Account:
    ß$\ldots$ß # __init__ ß\fMM{ကို ထပ်မပြဘဲ ချန်ထားခဲ့တယ်}ß
    def deposit(self, amt):
        if amt <= Decimal(0.00):
            raise ValueError('Invalid amount for deposit!')
        self.balance += amt

    def withdraw(self, amt):
        if amt > self.balance:
            raise ValueError('Not enough balance!')
        self.balance -= amt
\end{py}
%
ဒီဖန်ရှင်တွေမှာလည်း ပထမ ပါရာမီတာက \fCode{self} ဖြစ်နေတာ တွေ့ရမှာပါ။ လက်ရှိအော့ဘ်ဂျက်ရဲ့ အခြေအ\allowbreak နေအောက်မှာ အလုပ်လုပ်ပေးမဲ့ ဖန်ရှင်ရဲ့ ပထမ ပါရာမီတာက \fCode{self} ဖြစ်ရပါမယ်။ ဒီဖန်ရှင်တွေက အော့ဘ်ဂျက်အပေါ်မှာ လုပ်ဆောင်လို့ရတဲ့ အော်ပရေးရှင်းတွေပါပဲ။ \fCode{Account} \fEn{instance} တွေအပေါ်မှာ \fCode{deposit} နဲ့ \fCode{withdraw} လုပ်ဆောင်လို့ ရမှာဖြစ်တယ်
%
\begin{py}
acc3.withdraw(Decimal('50_000.00'))
acc1.deposit(Decimal('25_000.00'))
\end{py}
%
ပထမတစ်ခုက အေမီ့ အကောင့်ကနေ  ငွေထုတ် \fEn{(withdraw)} လုပ်တာပါ။ သိပြီးဖြစ်တဲ့အတိုင်း \fCode{acc3} နဲ့ \fCode{self} ဟာ လက်ရှိအော့ဘ်ဂျက် ဖြစ်တယ်။ \fCode{withdraw} ကို ကြည့်ရင် \fCode{amt} က လက်ရှိအော့ဘ်ဂျက်ရဲ့ \fCode{balance} ထက် များနေရင် \fEn{exception raise} လုပ်ထားတယ်။ ရှိတဲ့လက်ကျန်ငွေထက် ပိုထုတ်လို့ မရသင့်ဘူး။ 
%
\begin{py}
self.balance -= amt
\end{py}
%
ကတော့ လက်ရှိအော့ဘ်ဂျက်ရဲ့ \fCode{balance} ကနေ \fCode{amt} နှုတ်လိုက်တာပါ။ အေမီ့ အကောင့်မှာ လက်ကျန်ငွေ $300,000.00$ ဖြစ်သွားမယ်။ ဒုတိယတစ်ကြောင်းက ကေသီ့ အကောင့်ကို ငွေသွင်း \fEn{(deposit)} လုပ်တာ။ ဒီတစ်ခါကျတော့ လက်ရှိအော့ဘ်ဂျက်က  ကေသီ့အကောင့်ပေါ့။ \fCode{self.balance += amt} က ကေသီ့အကောင့်နဲ့ သက်ဆိုင်တဲ့ အော့ဘ်ဂျက်ရဲ့ \fCode{balance} ကို \fCode{amt} ပမာဏ ပေါင်းပေးတာ။ ဒါကြောင့် ကေသီ့ အကောင့်လက်ကျန်ငွေ $125,000.00$ ဖြစ်သွားပါမယ်။ 
%
\begin{py}
print(acc3.balance)  # 300000.00 
print(acc1.balance)  # 125000.00
\end{py}
%

\begin{figure}[tbh!]
\definecolor{myFieldFill}{HTML}{F0F6FD}
\definecolor{myFieldBorder}{HTML}{212121}
\tikzset{myFieldStyle/.style={draw=myFieldBorder,fill=myFieldFill,semithick}}
\def\fObjFigAttrValWidth{3}
\def\fObjFigAttrValHeight{1}
\newcommand{\figpctw}{0.48}
\begin{subfigure}{{\figpctw}\textwidth} 
\begin{tikzpicture}
    %\draw[step=.5cm,gray,very thin] (-1.4,-1.4) grid (1.4,1.4);
    %\filldraw[color=red!20] (0,0) circle (1ex);
    \draw[myFieldStyle,yshift=\fObjFigAttrValHeight*2cm] (0,0)rectangle node{\fEn{Amy}} (\fObjFigAttrValWidth,\fObjFigAttrValHeight);
    \draw[myFieldStyle,yshift=\fObjFigAttrValHeight*1cm] (0,0)rectangle node{\fEn{0086-6002-2233}} (\fObjFigAttrValWidth,\fObjFigAttrValHeight);
    \draw[myFieldStyle,yshift=\fObjFigAttrValHeight*0cm] (0,0)rectangle node{\fEnBf{300,000.00}} (\fObjFigAttrValWidth,\fObjFigAttrValHeight);
   
    \draw (0,0.5*\fObjFigAttrValHeight) node[anchor=east, yshift=\fObjFigAttrValHeight*2cm]{\fEn{holder}}
          (0,0.5*\fObjFigAttrValHeight) node[anchor=east, yshift=\fObjFigAttrValHeight*1cm]{\fEn{accountNumber}}
          (0,0.5*\fObjFigAttrValHeight) node[anchor=east, yshift=\fObjFigAttrValHeight*0cm]{\fEn{balance}};
    \draw (0.5*\fObjFigAttrValWidth, -0.5*\fObjFigAttrValHeight) node{\fEnBf{(Account)}};   
\end{tikzpicture}
\caption{အေမီ့ Account အော့ဘ်ဂျက် (\fCptCodeBf{acc3})}
\end{subfigure}
\begin{subfigure}{{\figpctw}\textwidth} 
\begin{tikzpicture}
    %\draw[step=.5cm,gray,very thin] (-1.4,-1.4) grid (1.4,1.4);
    %\filldraw[color=red!20] (0,0) circle (1ex);
    \draw[myFieldStyle,yshift=\fObjFigAttrValHeight*2cm] (0,0)rectangle node{\fEn{Kathy}} (\fObjFigAttrValWidth,\fObjFigAttrValHeight);
    \draw[myFieldStyle,yshift=\fObjFigAttrValHeight*1cm] (0,0)rectangle node{\fEn{0086-6002-1111}} (\fObjFigAttrValWidth,\fObjFigAttrValHeight);
    \draw[myFieldStyle,yshift=\fObjFigAttrValHeight*0cm] (0,0)rectangle node{\fEnBf{125,000.00}} (\fObjFigAttrValWidth,\fObjFigAttrValHeight);
   
    \draw (0,0.5*\fObjFigAttrValHeight) node[anchor=east, yshift=\fObjFigAttrValHeight*2cm]{\fEn{holder}}
          (0,0.5*\fObjFigAttrValHeight) node[anchor=east, yshift=\fObjFigAttrValHeight*1cm]{\fEn{accountNumber}}
          (0,0.5*\fObjFigAttrValHeight) node[anchor=east, yshift=\fObjFigAttrValHeight*0cm]{\fEn{balance}};
    \draw (0.5*\fObjFigAttrValWidth, -0.5*\fObjFigAttrValHeight) node{\fEnBf{(Account)}};   
\end{tikzpicture}
\caption{ကေသီ့ Account အော့ဘ်ဂျက် (\fCptCodeBf{acc1})}    
\end{subfigure}
\caption{After \fCptCodeBf{withdraw} and \fCptCodeBf{deposit} on \fCptCodeBf{acc3} and \fCptCodeBf{acc1} respectively}
\end{figure}

\begin{mytcbox}
\fEnBf{Attributes, Methods and Members} 
\betweentcboxpar
\noindent အော့ဘ်ဂျက်မှာပါဝင်တဲ့ အချက်အလက်ကို \fEnEmp{attribute} လို့ခေါ်ပြီး အော့ဘ်ဂျက်ပေါ်မှာ လုပ်ဆောင်နိုင်တဲ့ အော်ပရေးရှင်းကို \fEnEmp{method} (မက်သဒ်) လို့ခေါ်တယ်။ အော့ဘ်ဂျက်တစ်ခုမှာ ပါဝင်တဲ့ \fEn{attribute} တွေနဲ့ \fEn{method} တွေ နှစ်မျိုးလုံးကိုခြုံငုံပြီး \fEnEmp{members} တွေလို့ ခေါ်ပါတယ်။ 
\end{mytcbox}


%%% Python way of access control
% https://stackoverflow.com/questions/1301346/what-is-the-meaning-of-single-and-double-underscore-before-an-object-name#:~:text=Single%20leading%20underscores%20is%20a,%2C%20__bool__%20%2C%20etc.

% Fluent Python

% https://realpython.com/python-getter-setter/
\section{Access Control}
အိမ်ရှင်တစ်ယောက်ဟာ သူ့အိမ်မှာရှိသမျှ ပစ္စည်းတွေ (သို့) အခန်းတွေ အားလုံးကိုတော့ အိမ်လာတဲ့ ဧည့်သည်ကို အသုံးပြုခွင့် မပေးချင်ပါဘူး။ အိမ်ရှင်ကိုယ်တိုင်က အားလုံးကို အသုံးပြုနိုင်ပေမဲ့ ဧည့်သည်ဖြစ်သူကတော့ အိမ်ရှင်ခွင့်ပြုတာကိုပဲ အသုံးပြုသင့်ပါတယ်။ ထိုနည်းတူစွာ ကလပ်စ်ပိုင်ရှင် (ကလပ်စ်သတ်မှတ်သူ) ဟာ သူ့ကလပ်စ် \fEn{instance} တွေမှာ ပါဝင်တဲ့ \fEn{members} တွေကို အသုံးပြုသူတွေ ‘အကန့်အသတ်မရှိ ဆန္ဒရှိသလိုသာသုံး’ ခွင့်ပြုလို့မရပါဘူး။ ကလပ်စ်ပိုင်ရှင်ဟာ ကလပ်စ် \fEn{instance} တွေ မှန်ကန်တဲ့ \fEn{state} မှာ အမြဲရှိနေအောင် ကန့်သတ် ထိန်းချုပ်ထားရပါတယ်။ အသုံးပြုသူတွေ မသိသင့်တဲ့၊ အသုံးမပြုသင့်တဲ့ အတွင်းပိုင်းတည်ဆောက်ပုံ အသေးစိတ်နဲ့သက်ဆိုင်တဲ့ \fEnEmp{implementation details} တွေကိုလည်း ဖုံးကွယ်ထားနိုင်ဖို့ နည်းလမ်းတွေ လိုအပ်ပါတယ်။ ဒီအခန်းကဏ္ဍကို \fEnEmp{access control} လို့ ခေါ်ပါတယ်။ ပရိုဂရမ်းမင်း စလေ့လာသူတွေအတွက် အခုပြောတာတွေကို မြင်နိုင်ဖို့ အခက်အခဲ ရှိနေမှာပါ။ ဥပမာတချို့နဲ့ ရှင်းပြပါမယ်။

\fCode{Account} \fEn{instance} တွေမှာ ပြဿနာတချို့ ရှိနေပါတယ်။ \fCode{withdraw} နဲ့ \fCode{deposit} မက်သဒ်တွေက ငွေရှိတာထက် ပိုထုတ်လို့မရအောင်၊ ငွေသွင်းရင်လည်း ဖြစ်သင့်တဲ့ပမာဏပဲ သွင်းအောင် ကာကွယ်ထားတာ တွေ့ရမှာပါ။ ဒါပေမဲ့ ခက်တာက အသုံးပြုသူတွေက ဒီအကာအကွယ်တွေကို ကျော်ပြီး အခုလို လုပ်လို့ရနေပါလိမ့်မယ်
%
\begin{py}
acc1.balance = Decimal('-50_000.00')    # 
acc1.acc_number = None                  #
\end{py}
%
လက်ကျန်ငွေဟာ အနှုတ်တန်ဖိုး ဖြစ်လို့မရသင့်ဘူး။ ဘဏ်အကောင့်တစ်ခုမှာ အကောင့်နံပါတ်ကလည်း ရှိကိုရှိရပါမယ်။  ဒီလိုသာ ထင်သလို တန်ဖိုး ထည့်လို့ရနေရင် ပြဿနာပဲ။ အခုလိုဆိုရင် \fCode{acc1} အော့ဘ်ဂျက်ဟာ \fEn{invalid state} (မမှန်ကန်တဲ့ အနေအထား) ဖြစ်သွားမှာပါ။ အသုံးပြုသူက သေချာစဉ်းစားပြီး မဖြစ်သင့်တာ မလုပ်နဲ့ပေါ့လို့ စောဒကတက်စရာတော့ ရှိပါတယ်။ လျှပ်စစ်မီး ပလပ်ပေါက်ထဲ သတ္တုချောင်းထိုးထည့်လို့ ဓါတ်လိုက်ရင် \fEn{user fault} လို့ပြောတာက မှားတော့မမှားပါဘူး။ ဒါပေမဲ့လည်း မီးပလပ်ပေါက် ထုတ်လုပ်သူအနေနဲ့ အတတ်နိုင်ဆုံး အဲဒီလိုလုပ်လို့မရအောင်၊ အန္တရာယ်နည်းနိုင်သမျှ နည်းအောင် ဒီဇိုင်းပြုလုပ်ထားသင့်ပါတယ်။

အော့ဘ်ဂျက် \fEn{attribute} တွေကို တိုက်ရိုက် အသုံးမပြုစေချင်တဲ့အခါ \fEn{Python} ဓလေ့ထုံးစံက \fEn{attribute} နံမည်ရှေ့မှာ \fCode{\_} \fEn{(underscore)} ထည့်ပေးပါတယ်။
%
\begin{py}
class Account:
    def __init__(self, holder, acc_number, balance):
        self._holder = holder
        self._acc_number = acc_number
        self._balance = balance
\end{py}
%
\fEn{Attribute} နံမည်တွေကို \fCode{\_holder}\fEn{,} \fCode{\_acc\_number} စသည်ဖြင့်  \fCode{\_} နဲ့စထားတာ သတိပြုပါ။ ဒီလိုနံမည်တွေဆိုရင် ကလပ်စ်အသုံးပြုသူ (အတိကျပြောရင် ကလပ်စ်ကနေ ဖန်တီးယူတဲ့ \fEn{instance} အသုံးပြုသူ) အနေနဲ့ ‘တိုက်ရိုက် အသုံးမပြုရ’ လို့ အဓိပ္ပါယ်ရတယ်။ ဒါက \fEn{Python} ပရိုဂရမ်မာ အများစု လက်ခံထားတဲ့ ဓလေ့ထုံးစံတစ်ခုသာ ဖြစ်တယ်။ အားလုံးက \fCode{\_} နဲ့စတဲ့ \fEn{attribute} (မက်သဒ်တွေလည်းပါတယ်) တွေဆိုရင် တိုက်ရိုက်ယူမသုံးသင့်ဘူးလို့ နားလည် လက်ခံထားတယ်။ ဒါပေမဲ့ အခြား \fEn{programming language} တွေလို လုံးဝသုံးလို့မရအောင် ကန့်သတ်လို့တော့ \fEn{Python} မှာ မရနိုင်ဘူး။ သုံးချင်သပဆိုရင်လည်း သုံးလို့တော့ရတယ်။ ဒါပေမဲ့ ဖြစ်လာမဲ့အကျိုးဆက်ဟာ အသုံးပြုသူရဲ့ တာဝန်သာဖြစ်တယ်။ ဒါကြောင့် ဒီလိုလုပ်လို့ ရတယ်
%
\begin{py}
acc1._balance = Decimal('-50_000.00')    # ß\fMM{မလုပ်သင့်ပါ}ß
\end{py}
%
ဒါပေမဲ့ ထုံးစံကိုဖေါက်ဖျက်တာ ဖြစ်တယ်။ မလုပ်သင့်တဲ့ အရာပေါ့။ 

ဒီလို ဘာကြောင့် မသုံးသင့်လဲ အခြားအကြောင်းအရင်း တစ်ခုလည်း ရှိပါသေးတယ်။ ကလပ်စ်ပိုင်ရှင်က သူ့ကလပ်စ် \fEn{implementation details} တွေကို အကြောင်းအမျိုးမျိုးကြောင့် ပြင်ဆင် ပြောင်းလဲရလေ့ရှိတယ်။ ဥပမာ ပိုမြန်တဲ့၊ ဒါမှမဟုတ် မမ်မိုရီ အစားသက်သာစေမဲ့ နည်းလမ်းတွေကို ပြောင်းလဲ အသုံးပြုရနိုင်တယ်။ ဘဏ်အကောင့်တွေမှာ ငွေသွင်းငွေထုတ် စာရင်းကို အစဉ်အတိုင်း သိမ်းထားဖို့ လိုတယ်ဆိုပါစို့။ \fEn{List} တစ်ခုထဲမှာ သွင်းငွေပမာဏကို အပေါင်းတန်ဖိုး၊ ထုတ်ငွေပမာဏကို အနှုတ်တန်ဖိုးနဲ့ သိမ်းထားနိုင်တယ်။ တစ်သိန်းသွင်းပြီး နှစ်သောင်းခွဲ ထပ်သွင်းတယ်။ နောက်တော့ သုံးသောင်းခွဲ ပြန်ထုတ်တယ် ဆိုပါစို့။ \fEn{List} ထဲမှာ အခုလို ရှိနေရပါမယ်
%
\begin{py}
[100000.00, 25000.00, -35000.00]
\end{py}
%
ဒီတန်ဖိုးတွေကို ပေါင်းလိုက်ရင် လက်ရှိလက်ကျန်ငွေ \fEn{balance} ကိုရမှာပါ။ ဒီလိုဆိုရင် သီးသန့်  \fEn{attribute} တစ်ခုနဲ့ \fEn{balance} ကို သိမ်းထားဖို့ မလိုအပ်တော့ဘူး။ ငွေသွင်းငွေထုတ် စာရင်းကနေ  တွက်ယူနိုင်တယ်။ အဖက်ဖက်ကနေ စဉ်းစားသုံးသပ်ပြီး ပိုင်ရှင်က သူ့ကလပ်စ်ကို အခုလို \fEn{update} လုပ်ဖို့ ဆုံးဖြတ်ချက် ချနိုင်ပါတယ်


%
\begin{py} 
from decimal import * ß\label{ch09:txntobal}ß

class Account:
    def __init__(self, holder, acc_number, balance):
        self._holder = holder
        self._acc_number = acc_number
        self._transactions = [balance]

    def deposit(self, amt):
        if amt <= Decimal(0.00):
            raise ValueError('Invalid amount for deposit!')
        self._transactions.append(amt)

    def withdraw(self, amt):
        if amt > self.balance:
            raise ValueError('Not enough balance!')
        self._transactions.append(-amt)

    @property
    def balance(self):
        return sum(self._transactions)
\end{py}
လောလောဆယ် အသေးစိတ်တွေက သိပ်အရေးမကြီးဘူး။ \fCode{\_balance} ဖြုတ်လိုက်တာကိုသာ အဓိက အာ\allowbreak ရုံထားပါ။ ဒီအခါ သူ့ကို တိုက်ရိုက် သုံးထားတဲ့ ကုဒ်အားလုံး (ဥပမာ \fCode{acc1.\_balance}) ကလပ်စ် ဗားရှင်းအသစ်ကို \fEn{update} ယူပြီးသုံးရင်  ပြဿနာတက်သွားမှာပါ။ ပိုကောင်းတဲ့ ဗားရှင်းအသစ်ကို သုံးချင်တယ်ဆိုရင် တိုက်ရိုက်သုံးထားတဲ့နေရာတွေ အားလုံးကို လိုက်ပြင်ရပါလိမ့်မယ်။ ကလပ်စ်ပိုင်ရှင်က တမင်တကာ ဖုံးကွယ်ထားတဲ့ \fEn{implementation details} တွေကို တိုက်ရိုက်မသုံးရင် ဒီလိုပြဿနာမျိုးတွေ မဖြစ်နိုင်ဘူးပေါ့။

ဒီလောက်ဆိုရင် \fEn{access control} ဘာကြောင့် အရေးကြီးသလဲ နားလည် သဘောပေါက်လောက်ပါပြီ။ ကလပ်စ် ပိုင်ရှင်ရော အသုံးပြုသူတွေ အတွက်ပါ ဂရုစိုက်စဉ်းစားရမဲ့ ကိစ္စဖြစ်ပါတယ်။ ဘယ်သူမဆို တရားဝင် အသုံးပြုခွင့် ပေးထားတဲ့ \fEn{members} ကို  ကလပ်စ်ရဲ့ \fEnEmp{interface} သို့ \fEn{API (\textit{Application Programming Interface})} လို့ခေါ်ပါတယ်။ ကလပ်စ်ပိုင်ရှင်က စဉ်းစားဆင်ခြင် သုံးသပ်ပြီး ကလပ်စ် \fEn{API} က ဘယ်လိုဖြစ်သင့်လဲ တိတိကျကျ သတ်မှတ်ပေးရတာပါ။ ဖုံးကွယ်ထားသင့်တဲ့ \fEn{implementation details} တွေကို ပေးသိ/သုံး မိတာဟာ နောင်တစ်ချိန် ကလပ်စ်ကို ပြင်ဆင်မွမ်းမံဖို့ အဟန့်အတားဖြစ်စေပါတယ်။ အသုံးပြုသူတွေကလည်း မသိ/မသုံးစေချင်တဲ့ \fEn{internal} တွေကို အသုံးပြုခြင်းဖြင့် စည်းမဖေါက်ဖို့ လိုပါတယ်။ 

\section{Composition or Has-a Relationship}
အခုတွေ့ခဲ့တဲ့ \fCode{Account} \fEn{instance} တွေက ပိုင်ရှင် နံမည်ကိုပဲ \fEn{attribute} အနေနဲ့ သိမ်းတယ်။ နံမည်အပြင် အကောင့် ပိုင်ရှင် မှတ်ပုံတင်နံပါတ် \fEn{(NRIC)}၊ မွေးသက္ကရာဇ်၊ နေရပ်လိပ်စာတို့ကိုလည်း ဘဏ်တွေက မှတ်သား သိမ်းဆည်းရပါတယ်။ ပရိုဂရမ်ထဲမှာလည်း ဒီအချက်အလက်တွေကို ထင်ဟပ်ဖော်ပြဖို့ လိုပါမယ်။ \fCode{Account} ကလပ်စ်မှာ ဒီအချက်အလက်တွေအတွက် \fEn{attribute} အသစ်တွေ ထပ်ထည့်ရမလား၊ ဒါမှမဟုတ် ပိုင်ရှင်တစ်ယောက်ချင်းကို အော့ဘ်ဂျက်အနေနဲ့ ဖော်ပြရမလား စဉ်းစားစရာ ဖြစ်လာပါတယ်။ အပြင်မှာရှိတဲ့ ဘဏ်အကောင့်ကို \fCode{Account} \fEn{instance} နဲ့ ကိုယ်စားပြုတယ်ဆိုရင် အပြင်က အကောင့်ပိုင်ရှင်ကို \fCode{Holder} \fEn{instance} နဲ့ ဖော်ပြတာဟာလည်း ကျိုးကြောင်းဆီလျော်တယ်လို့ပဲ ယူဆရမှာပါ။ အော့ဘ်ဂျက် ဖြစ်သင့်/မသင့် စဉ်းစားဆုံးဖြတ်တဲ့အခါ အခြေခံအကျဆုံး မေးခွန်းတစ်ခုကတော့ မိမိ ဖော်ပြလိုတဲ့ အရာဟာ \fEn{attributes} တွေအပြင် ဘယ်လို \fEn{methods} တွေ ပါဝင်သင့်လဲ ဆိုတာပါ။ \fEn{Attribute} တွေပဲ ရှိပြီး \fEn{method} တစ်ခုမှ မရှိရင် အော့ဘ်ဂျက်အနေနဲ့ မရှုမြင်သင့်ဘူး။ \fCode{Holder} \fEn{instance} တွေ အပေါ်မှာ လုပ်ဆောင်လို့ ရသင့်တဲ့ \fEn{method} တချို့ ခေါင်းထဲပေါ်လာဖို့ သိပ်မခက်သင့်ဘူး။ အခြားလည်း ရှိအုံးမှာပါ။
%
\begin{py}
holder1.change_address(new_address)
\end{py}
%
အပြင်မှာလည်း လိပ်စာပြောင်းတာ မကြာခဏဖြစ်လေ့ရှိပါတယ်။ \fEn{Attributes} ရော \fEn{methods} တွေပါ ရှိမယ်ဆိုရင် အော့ဘ်ဂျက်အနေနဲ့ စဉ်းစားနိုင်ပါတယ်။ ဒါဆိုရင် ကလပ်စ်သတ်မှတ် ရပါမယ်
%
\begin{py}
class Holder:
    def __init__(self, name, dob, nric, gender):
        self._name = name
        self._dob = dob
        self._nric = nric
        self._gender = gender

    # ß$\ldots$ß
\end{py}
%
ဒီလောက်နဲ့ အော့ဘ်ဂျက် ဖန်တီးယူလို့ ရပါပြီ (ကျန်တဲ့ ပါသင့်တာတွေ ခဏနေရင် ထပ်ဖြည့်ပါမယ်)။ ဒီစက်ရှင်ရဲ့ အဓိက အကြောင်းအရာကို အရင်ကြည့်ရအောင်။ ဘဏ်အကောင့်နဲ့ အကောင့်ပိုင်ရှင် ဆက်စပ်နေတာကို ဖော်ပြချင်ရင် ဒီလိုပါ
%
\begin{py}
holder1 = Holder('Kathy',
                 date(1990, 3, 10),
                 'MaRaNa (N) 1343232',
                 'F')
acc1 = Account(holder1, '0086-6002-1111', Decimal('100_000.00'))
\end{py}
%
အကောင့်ပိုင်ရှင် ကေသီ့ကို ကိုယ်စားပြုတဲ့ အော့ဘ်ဂျက် ဖန်တီးယူမယ်။ ပြီးတော့ ကေသီ့အကောင့်မှာ အဲ့ဒီအော့ဘ်ဂျက်ကို ထည့်ပေးလိုက်ရုံပဲ။ \fCode{Account} မှာ \fCode{Holder} က \fEn{attribute} တစ်ခုအနေနဲ့ ရှိနေမှာပါ။ အော့ဘ်ဂျက်နှစ်ခု အခုလို ဖွဲ့စည်းထားတာကို \fEn{composition} လို့ ခေါ်ပါတယ်။ \fEn{Composition} ဟာ ပိုင်ဆိုင်ခြင်း (သို့) ရှိခြင်း ကိုလည်း ဖော်ပြတယ်။  ‘ဘဏ်အကောင့်တစ်ခုမှာ အကောင့်ပိုင်ရှင်တစ်ယောက် ရှိတယ်’၊ ‘ကားတစ်စီးမှာ အင်ဂျင်တစ်လုံး ပါရှိတယ်’။  ဒီလိုဆက်စပ်မှုမျိုးကို ဖော်ပြတဲ့ အော့ဘ်ဂျက်နှစ်ခုဟာ \fEn{has-a relationship} ရှိတယ်လို့ ပြောလေ့ရှိတယ်။

အော့ဘ်ဂျက်တစ်ခုဟာ တစ်ခုထက်ပိုတဲ့ အခြား အော့ဘ်ဂျက်တွေနဲ့ \fEn{has-a relation} ရှိနိုင်တာပေါ့။ ဥပမာ အကောင့်ဖွင့်တဲ့ ဘဏ်ခွဲကို အော့ဘ်ဂျက်အနေနဲ့ ယူဆမယ်ဆိုရင် 
%
\begin{py}
class Branch:
    def __init__(self, name, address):
        self._name = name
        self._address = address
    # ß$\ldots$ß

class Account:
    def __init__(self, holder, acc_number, balance, branch):
        # ß$\ldots$ \fMM{အခြား \fEn{attributes} တွေရှိမယ်၊ မပြထားတော့တာ}ß
        self._branch = branch
\end{py}
%
\fCode{Account} နဲ့ သူ့ရဲ့ \fCode{Branch} ကို ချိတ်ဆက်မယ်ဆိုရင်
%
\begin{py}
# ß$\ldots$ holder1 \fMM{ရှိမယ်၊ မပြထားတော့တာ}ß
branch1 = Branch('Mandalay', '232A, 62 St, Chanayetharsi')
acc1 = Account(holder1, 
               '0086-6002-1111', 
               Decimal('100_000.00'), 
               branch1)
\end{py}
%

\subsection*{Types of Composition and Sharing}
ဘဏ်ခွဲတစ်ခုတည်းမှာ ဖွင့်ထားတဲ့ ဘဏ် အကောင့်တွေဟာ \fCode{Branch} အော့ဘ်ဂျက်တစ်ခုတည်းကိုပဲ \fEn{sharing} လုပ်ထားပြီး အသုံးပြုနိုင်ပါတယ်။
%
\begin{py}
branch1 = Branch('Mandalay', '232A, 62 St, Chanayetharsi')
acc1 = Account(holder1, 
               '0086-6002-1111', 
               Decimal('100_000.00'), 
               ß\textbf{branch1}ß)
acc2 = Account(holder2, 
               '0086-6002-1111', 
               Decimal('100_000.00'), 
               ß\textbf{branch1}ß)
\end{py}
%
\fCode{acc1} နဲ့ \fCode{acc2} နှစ်ခုလုံးက \fEn{branch1} အော့ဘ်ဂျက်တစ်ခုတည်းကိုပဲ မျှသုံးထားတာပါ။ လူတစ်ယောက်\allowbreak တည်းက အကောင့် နှစ်ခု ဖွင့်ထားရင် \fCode{Holder} အော့ဘ်ဂျက်တစ်ခုတည်းကို မျှသုံးပါမယ်။

%
\begin{py}
holder1 = Holder('Kathy',
    date(1990, 3, 10),
    'MaRaNa (N) 1343232',
    'F')
acc1 = Account(ß\textbf{holder1}ß, '0086-6002-1111', Decimal('100_000.00'))
acc5 = Account(ß\textbf{holder1}ß, '0086-6002-5522', Decimal('500_000.00'))
\end{py}
%

\fEn{Has-a relationship} ကို အမျိုးအစားတွေခွဲကြည့်ရင် \fEn{one-to-one, one-to-many, many-to-many} စသည့်ဖြင့်တွေ့ရပါတယ်။ \fCode{Car} နဲ့ \fCode{Engine} ဟာ \fEn{one-to-one} ဖြစ်ပါမယ်။ \fCode{Student} နဲ့ \fCode{Course} (သို့) \fCode{Actor} နဲ့ \fCode{Film} ဆိုရင်တော့ \fEn{many-to-many relationship} ပေါ့။ \fEn{One-to-many} ကို \fCode{Bank} နဲ့ \fCode{Branch}\fEn{,} \fCode{House} နဲ့ \fCode{Resident} တို့အကြား တွေ့ရမှာပါ။ 

\fCode{Bank} နဲ့ \fCode{Account} တို့ အကြား \fEn{one-to-many relationship} ကို ဒီအခန်းနောက်ဆုံးမှာ တွေ့ရပါမယ်။ \fEn{One-to-one} အတွက် \fCode{Car} နဲ့ \fCode{Engine}\fEn{,} \fEn{many-to-many} အတွက် \fCode{Student} နဲ့ \fCode{Course} ကို ဖော်ပြထားတာ လေ့လာကြည့်ပါ။ 

%
\begin{py}
class Engine:
    def __init__(self, horsepower):
        self._horsepower = horsepower


class Car:
    def __init__(self, model, engine):
        self._model = model
        self._engine = engine


# Creating an engine object
my_engine = Engine(horsepower=200)

# Creating a car object with the engine
my_car = Car(model="Toyota Camry", engine=my_engine)
\end{py}
%
\begin{py}
class Student:
    def __init__(self, name):
        self.name = name
        self.courses = []

    def enroll(self, course):
        self.courses.append(course)


class Course:
    def __init__(self, name):
        self.name = name
        self.students = []

    def add_student(self, student):
        self.students.append(student)


# Creating student objects
sandar = Student("Sandar")
waiyan = Student("Waiyan")

# Creating course objects
math_course = Course("Mathematics")
science_course = Course("Science")

# Enrolling students in courses
sandar.enroll(math_course)
sandar.enroll(science_course)
waiyan.enroll(math_course)

# Adding students to courses
math_course.add_student(sandar)
math_course.add_student(waiyan)
science_course.add_student(sandar)

\end{py}
%

ကျောင်းသူ/သား တစ်ယောက် တက်ရောက်ဖို့ စာရင်းပေးထားတဲ့ \fEn{course} တွေအတွက်၊  \fCode{course} တစ်ခုကို တက်ရောက်မဲ့ ကျောင်းသူ/သား တွေအတွက် \fEn{list} ကို သုံးထားတာ ဂရုပြုကြည့်ပါ။ \fEn{Has-a relationship} မှာ \fEn{many} ဘက်ခြမ်းကို ဖော်ပြဖို့အတွက်  အော့ဘ်ဂျက်တွေ တစ်စုတစ်စည်းတည်း ထားနိုင်တဲ့ \fEn{list, set, dictionary} စတာတွေ သုံးရပါတယ်။


\section{Properties}
\fEn{Access control} စည်းကမ်းအတိုင်း လိုက်နာရမယ်ဆိုရင် \fCode{Holder} ရဲ့ \fEn{attributes} တွေကို တိုက်ရိုက်မသုံးသင့်ဘူး။ ဒါဆို \fEn{attributes} တွေကို သုံးဖို့လိုအပ်ရင် ဘယ်လိုလုပ်ရမလဲ မေးစရာ ရှိပါတယ်။ ကလပ်စ်ပိုင်ရှင် အနေနဲ့ \fEn{attribute} ကို မက်သဒ်ကနေတစ်ဆင့် အသုံးပြုလို့ရအောင် လမ်းဖွင့်ပေးထားနိုင်ပါတယ်။ မဖြစ်သင့်တာ မဖြစ်အောင် မက်သဒ်မှာ စိစစ်လို့ရတယ်။ 
%
\begin{py}
class Holder:
    def __init__(self, name, dob, nric, gender):
        self._name = name
        self._dob = dob
        self._nric = nric
        self._gender = gender

    def get_name(self):
        return self._name

    def set_name(self, name):
        if name.strip() == '':
            raise ValueError('Name cannot be empty!')
        self._name = name.strip()
\end{py}
%
\fCode{set\_name} မှာ နံမည်ကို \fEn{empty string} မဖြစ်အောင် စစ်ထားတယ်။ အသုံးပြုသူအနေနဲ့ နံမည်ကို ယူကြည့်တာ၊ ပြောင်းတာ လုပ်မယ်ဆိုရင် မက်သဒ်ကိုပဲ သုံးရမှာပါ။ ဥပမာ
%
\begin{py}
print(holder1.get_name())
holder1.set_name('Amy Lynn')
holder1.set_name('    ')        # ValueError ß\fMM{ဖြစ်မယ်}ß
\end{py}
%
(\fCode{\_} နဲ့ မစရင် \fEn{API,} သုံးလို့ရတယ်)။

အခြား \fEn{attribute} တွေလည်း လိုအပ်ရင် အလားတူ မက်သဒ်တွေနဲ့  \fEn{API} ထုတ်ပေးထားလို့ရပါတယ်။ ဒီနည်းလမ်းအပြင် \fEn{Python} မှာ  \fEnEmp{managed attributes} ဆိုတာလည်း ရှိပါသေးတယ်။ \fEn{Managed attributes} တွေကို \fEnEmp{properties} လို့လည်း ခေါ်တယ်။ \fEn{Managed attribute} သုံးထားတဲ့ ဥပမာကို လေ့လာကြည့်ပါ

%
\begin{py}
class Holder:
    def __init__(self, name, dob, nric, gender):
        self._name = name
        self._dob = dob
        self._nric = nric
        self._gender = gender

    @property
    def name(self):
        return self._name

    @name.setter
    def name(self, name):
        if name.strip() == '':
            raise ValueError('Name cannot be empty!')
        self._name = name

    @property
    def dob(self):
        return self._dob

    @dob.setter
    def dob(self, value):
        self._dob = value

    # ß$\ldots$ß
\end{py}
%
\fCode{@} သင်္ကေတနဲ့ စထားတဲ့ \fCode{@property}\fEn{,} \fCode{@name.setter} စသည်ဖြင့် တွေ့ရပါတယ်။ \fEn{Property decorator} လို့ခေါ်ပါတယ်။ ဒါလေးတွေက နောက်ကွယ်မှာတော့ ဖန်ရှင်တစ်ခုပါပဲ။ \fCode{property()} ဖန်ရှင်က သူတို့ရဲ့ နောက်ကွယ်ကနေ အလုပ်လုပ်ပေးမှာပါ။  လောလောဆယ် အသေးစိတ်တွေ ခဏထားလိုက်ပြီး အသုံးပြုနည်းကို အဓိကထား ကြည့်ရအောင်။ \fEn{Decorator} ရဲ့ အောက်မှာ ရှိနေတာက ပုံမှန်မက်သဒ်တွေ ပါပဲ။ \fEn{Property} \fEn{decorator} က အဓိကဘာလုပ်ပေးလဲ။ မက်သဒ်တစ်ခုကို ပုံမှန် \fEn{attribute}  ဗေရီရေဘဲလ်လို သုံးလို့ရအောင် လုပ်ပေးတာပါ။ ဥပမာ \fCode{name} မက်သဒ်ကို ခေါ်ရင်  \fCode{holder1.name()} လို့ ရေးနေစရာ မလိုဘူး။ \fCode{holder1.name} လို့ ရေးရုံပဲ။ တကယ်တမ်း နောက်ကွယ်မှာ အလုပ် လုပ်တာကတော့ မက်သဒ်ကပဲ လုပ်သွားမှာပါ။ \fCode{name} အတွက် \fCode{@name.setter}\fEn{,}  \fCode{dob} အတွက်ဆိုရင် \fCode{@dob.setter} စသည်ဖြင့် ဖြစ်ရပါမယ်။
 
အဆိုင်းမန့် လုပ်ရင်လဲ ဒီသဘောပဲ။ \fCode{holder1.name = 'Amy Lynn'} လို့ ရေးရင် \fCode{@name.setter} \fEn{decorator} နဲ့ မက်သဒ်က အလုပ်လုပ်သွားမှာ ဖြစ်တယ်။ \fCode{holder1.name('Amy Lynn')} မက်သဒ်ခေါ်တာနဲ့ သဘောတရား တူမယ်။ ဒါကြောင့် အခုလို စမ်းကြည့်ရင်
%
\begin{py}
holder1.name = '    '     
\end{py}
%
\fCode{ValueError} \fEn{exception} ဖြစ်ပါလိမ့်မယ်။ \fEn{Property decorator} သုံးထားတဲ့အတွက် \fCode{dob} ကိုလည်း \fEn{attribute} အနေနဲ့ သုံးလို့ရပါတယ်။
%
\begin{py}
println(holder1.dob)
holder1.dob = date(1990, 3, 11)
\end{py}
%

\fEn{Python} ပရိုဂရမ်မာတွေက \fEn{managed attribute} အသုံးပြုတာကို \fEn{Pythonic} ပိုဖြစ်တဲ့ နည်းလမ်းလို့ ယူဆကြပါတယ်။ ဒါ့ကြောင့် ရိုးရိုး \fCode{get\_name}\fEn{,} \fCode{set\_name} မက်သဒ်တွေနဲ့ \fEn{attribute} သုံးလေ့မရှိဘူး။ ဘီဂင်နာအနေနဲ့  မက်သဒ်ဖြစ်သင့်တာတွေကို \fEn{attribute} အနေနဲ့ မရှုမြင်ကြည့်မိဖို့ သတိပြုသင့်ပါတယ်။ \fCode{withdraw} နဲ့ \fCode{deposit} က \fEn{attribute} မဖြစ်သင့်ဘူး။ ဒါတွေက အော့ဘ်ဂျက်အပေါ်မှာ လုပ်ဆောင်လို့ရတဲ့ အော်ပရေးရှင်းတွေ အနေနဲ့ပဲ ရှုမြင်ရမှာပါ။ \fEn{Attribute} ဆိုတာက အော့ဘ်ဂျက် ပိုင်ဆိုင်ထားတဲ့ အချက်အလက်သာ ဖြစ်တယ်။ ဒီအချက်အလက်တွေကိုမှ လိုအပ်ရင် \fEn{managed} \fEn{attribute} အနေနဲ့ အသုံးပြုခွင့် ပေးရတာဖြစ်တယ်။ 

\subsection*{Derived Attributes/Properties}
\fCode{Holder} ရဲ့ အသက်ကို ဗေရီရေဘဲလ်အနေနဲ့ ထားစရာမလိုဘဲ မွေးသက္ကရာဇ်ကနေ တွက်ယူလို့ရတယ်။ ဒီလိုမျိုး \fEn{attribute} တွေကို \fEn{derived attributes} လို့ ခေါ်လေ့ရှိတယ်။ ၎င်းတို့အတွက် ဗေရီရေဘဲလ် ရှိဖို့ မလိုအပ်တာကြောင့်  နေရာကုန် သက်သာမယ်။ အသုံးပြုတဲ့အခါမှ လိုတဲ့တန်ဖိုးကို တွက်ယူတာမို့လို့ အချိန် အကုန်အကျရှိမယ်။ တချို့ တကယ့်လက်တွေ့ အခြေအနေတွေမှာ နေရာနဲ့ အချိန် အကုန်အကျကို ဆန်းစစ်သုံးသပ်ပြီး တန်ဖိုးအနေနဲ့ သိမ်းထားမလား၊ ဒါမှမဟုတ် တွက်ချက်ယူမလား ဆုံးဖြတ်ရတတ်ပါတယ်။  \fEn{Derived attributes} တွေက နေရာသက်သာပေမဲ့ အချိန်အကုန်အကျ ရှိတယ်လို့ အ\allowbreak ခြေခံအဆင့်မှာ သိထားသင့်ပါတယ်။
%
\begin{py}
class Holder:
    def __init__(self, name, dob, nric, gender: str):
        # ...
        self._dob = dob
        # ...

    @property
    def age(self):
        return age_today(self._dob)
\end{py}
%
စာမျက်နှာ (\fRefNo{\pageref{ch09:txntobal}}) က \fCode{Account} ကလပ်စ်မှာ \fEn{balance} ကို \fEn{derived attribute} အနေနဲ့ တွေ့ရမှာပါ။ \fEn{Transaction} (ငွေအဝင်အထွက်ကို ဆိုလို) အကြိမ်အရေအတွက် များရင်များသလို \fEn{balance} တွက်ထုတ်တဲ့အခါ အချိန်အကုန်အကျလည်း များမှာဖြစ်တယ်။
%
\begin{py}
class Account:
    def __init__(self, holder, acc_number, balance):
        # ...
        self._transactions = [balance]
    # ...
    @property
    def balance(self):
        return sum(self._transactions)
\end{py}
%

\subsection*{\fSubSecCodeBf{self} အကြောင်း သိကောင်းစရာ}
\fCode{self} နဲ့ ပါတ်သက်ပြီး ထိုက်သင့်သလောက် ဖော်ပြခဲ့ပြီးပါပြီ။ ဒီလောက်သိထားရင်ကို ဘီဂင်နာအဆင့် အတွက် လုံလောက်ပါတယ်။ ဒါပေမဲ့ စိတ်ဝင်စားလို့ ပိုသိချင်တဲ့သူတွေအတွက်ရော သိထားသင့်တဲ့အတွက်ပါ တချို့ဟာလေးတွေ ထပ်ကြည့်ကြပါမယ်။  

အော့ဘ်ဂျက်တစ်ခု စဖန်တီးတဲ့အခါ \fEn{heap} လို့ခေါ်တဲ့ မမ်မိုရီ ဧရိယာထဲမှာ သူ့အတွက် လိုအပ်တဲ့နေရာ အရင်ပေးရတယ် (\mintinline{text}|__init__| အလုပ်မလုပ်ခင် ဒီကိစ္စကို အရင်လုပ်ရတာ)။ နေရာပေးပြီးသွားရင် အဲ့ဒီအော့ဘ်ဂျက်ကို ရည်ညွှန်းလို့ရမဲ့ \fEn{reference} ကို ရရှိပါတယ် (\fEn{reference} ဆိုတာ အော့ဘ်ဂျက်ကို ရည်ညွှန်းတဲ့ ဗေရီရေဘဲလ်ကို ပြောတာပါပဲ)။ ဒီအဆင့်ပြီးရင် အကြမ်းထည်အဆင့် အော့ဘ်ဂျက်တစ်ခု  ရနေပါပြီ။ ဒီအော့ဘ်ဂျက်ကို \fEn{initialization} ဆက်လုပ်ရပါမယ်။ \mintinline{text}|__init__|  ဖန်ရှင်က \fEn{initialize} လုပ်ပေးရမဲ့ အော့ဘ်ဂျက်ကို သိရပါမယ်။ ဒါ့ကြောင့် \mintinline{text}|__init__| ခေါ်တဲ့အခါ ခုနက \fEn{referene} ကို \fCode{self} နေရာမှာ \fEn{Python} က အလိုအလျှောက် ထည့်ပေးသွားမှာပါ။ 

အော့ဘ်ဂျက် အချောထည် (\fEn{properly initialized object}) တည်ဆောက်ပြီးသွားတဲ့အခါ သူ့အတွက် \fEn{reference} ကို ပြန်လည်ရရှိမှာပါ။ ဒီ \fEn{reference} နဲ့ အော့ဘ်ဂျက်ကို ဆက်လက် အသုံးပြုလို့ ရတာဖြစ်တယ်။ ဥပမာ  
%
\begin{py}
acc4 = Account('Waiyan', ...)
\end{py}
%
ဒါဟာ ပြန်ရတဲ့ \fEn{reference} ကို \fCode{acc4} နဲ့ ဖမ်းယူထားလိုက်တဲ့ သဘောပဲ။ 


%https://www.reddit.com/r/learnpython/comments/12no20f/what_is_self_attribute_exactly/
\fCode{withdraw}\fEn{,} \fCode{deposit}  မက်သဒ်တွေ ခေါ်တဲ့အခါမှာလည်း   \fCode{self} နေရာမှာ လက်ရှိအော့ဘ်ဂျက်ကို \fEn{Python} က ထည့်ပေးတာ ဖြစ်တယ်။ 
%
\begin{py}
acc4.deposit(Decimal('50_000.00'))
\end{py}
%
အခုလို ခေါ်တဲ့အခါ \fCode{acc4} ကို \fCode{deposit} မက်သဒ် \fCode{self} နေရာမှာ ထည့်ပေးမှာပါ။ ဒီလို \fEn{transformation} လုပ်ပေးတယ်လို့ ယူဆနိုင်တယ်
%
\begin{py}
acc4.deposit(ß\textbf{acc4}ß, Decimal('50_000.00'))
\end{py}
%
မှတ်ချက်။\qquad ။ အတိအကျ ဒီလိုဖြစ်တယ်လို့ မဆိုလို။ တကယ့်တကယ်က \fCode{Account.deposit(ac4, Decimal('50\_000.00'))} အဖြစ် ပြောင်းပေးတာပါ။ ဒါပေမဲ့ ဘီဂင်နာအတွက် လက်ရှိ \fEn{knowledge} နဲ့ ဒါကိုနားလည်နိုင်ဖို့က စောလွန်းသေးတယ်။


\begin{mytcbox}
\fEnBf{Instance Variables and Instance Methods} 
\betweentcboxpar
အော့ဘ်ဂျက်တစ်ခုစီက သီးခြားပိုင်ဆိုင်တဲ့ ဗေရီရေဘဲလ်တွေကို \fEnEmp{instance variable} တွေလို့လည်း ခေါ်ပါတယ်။ \fEnEmp{Instance attribute} ပြောရင်လည်း အဓိပ္ပါယ်တူတူပါပဲ။ အော့ဘ်ဂျက်တစ်ခုစီက သီးခြားပိုင်ဆိုင်တဲ့ \fEn{attribute} တွေပေါ့။ 
\betweentcboxpar
\fEnEmp{Instance method} တွေကတော့ လက်ရှိအော့ဘ်ဂျက်ရဲ့ အခြေအနေအောက်မှာ အလုပ်လုပ်တဲ့ မက်သဒ်တွေဖြစ်ပြီး  \fEn{instance variable} တွေကို အသုံးပြုလေ့ရှိတယ် (\fCode{acc1.withdraw(amt)} မှာ \fCode{withdraw} ဟာ လက်ရှိအော့ဘ်ဂျက် \fCode{acc1}  အခြေအနေအောက်မှာ အလုပ်လုပ်တယ်)။ \fEn{Instance method} ဟာ လက်ရှိအော့ဘ်ဂျက်နဲ့ တွဲဖက်လုပ်ဆောင်ရတာ ဖြစ်တာကြောင့် \fCode{self} ပါရမီတာ မပါလို့ မရဘူး။ 
\end{mytcbox}


\section{Class Attributes and Methods}
\fEnEmp{Class attribute} ဆိုတာ ကလပ်စ်တစ်ခုရဲ့ \fEn{instance} တွေအားလုံးနဲ့ ဆိုင်တဲ့ ဘုံသုံး \fEn{attribute} တွေကို ဆိုလိုတာပါ။ \fEnEmp{Class attribute} တွေကို ကလပ်စ်ဘော်ဒီ \fEn{top level} မှာ ကြေငြာရပါမယ် (မက်သဒ်တွေရဲ့ အပြင်ဘက်၊ ကလပ်စ်အောက်မှာ တိုက်ရိုက်ရှိရမှာ၊ မက်သဒ်ထဲမှာ မဖြစ်ရဘူး)။ ဘုံသုံးဆိုတဲ့အတိုင်း လက်ရှိအော့ဘ်ဂျက်နဲ့ပဲ ဆိုင်တာမဟုတ်တော့ဘူး။ ဒါကြောင့်မို့လို့ \fEn{attribute} ကြေငြာတဲ့အခါ \fCode{self.\fEnEmp{attribute\_name}} ပုံစံ မကြေငြာရဘူး။
%
\begin{py}
# File: class_attr_eg.py
class SomeClass:
    # class attributes
    MAX_ID = 9999
    cur_id = 0

    def __init__(self):
        self._id = SomeClass.cur_id + 1
        SomeClass.cur_id += 1

    @property
    def id(self):
        return self._id

\end{py}
%
\betweenminted{\medskipamount}
%
\begin{py}
# create 3 instance for testing
smc1 = SomeClass()
smc2 = SomeClass()
smc3 = SomeClass()

# ß\fMM{အားလုံး} 3 \fMM{ထွက်တယ်}ß
print(SomeClass.cur_id)      # ß\fMM{ကလပ်စ် နံမည်နဲ့ သုံးတာ}ß
print(smc1.cur_id)           # ß\fMM{ကလပ်စ် \fEn{instance} သုံးတာ}ß
print(smc2.cur_id)
print(smc3.cur_id)

print(smc3.MAX_ID)
print(SomeClass.MAX_ID)
\end{py}
%
\fCode{MAX\_ID} နဲ့ \fCode{cur\_id} က \fEn{class attribute} တွေပါ။ အသုံးပြုတဲ့အခါ \fCode{\fEnEmp{ClassName}.\fEnEmp{attribute\_name}} ပုံစံနဲ့ သုံးရမှာပါ။ ဒါပေမဲ့ ကလပ်စ် \fEn{instance} ကနေ သုံးလို့လည်း ရပါတယ်။ \fEn{Instance} တွေကို ကိုယ်စားပြုတဲ့ \fCode{smc1}\fEn{,} \fCode{smc2} စတာတွေနဲ့ အသုံးပြုလို့ရတာ တွေ့ရပါမယ်။

\mintinline{text}|__init__| ထဲမှာ လက်ရှိအော့ဘ်ဂျက်ရဲ့ \fCode{\_id} ကို  \fCode{cur\_id} အပေါင်း တစ် ထည့်ထားတာ တွေ့ရပါမယ်။ ပြီးတော့ \fCode{cur\_id} တန်ဖိုးကိုလည်း တစ်တိုးလိုက်တယ်။ ပထမဆုံး \fCode{SomeClass} \fEn{instance} ရဲ့ \fCode{id} က တစ် ဖြစ်ပါမယ်။   အော့ဘ်ဂျက်တစ်ခု ဆောက်လိုက်တိုင်း \fCode{cur\_id} တန်ဖိုးက တစ်တိုးသွားမှာပါ။ အခုဥပမာမှာ အော့ဘ်ဂျက် သုံးခုယူထားတဲ့အတွက် သုံး ဖြစ်သွားပါမယ်။ 
 
 \fEn{Attribute} တစ်ခုတည်းကိုပဲ အော့ဘ်ဂျက်အားလုံးက မျှသုံးထားတာ ဖြစ်တဲ့အတွက် \fCode{print} ထုတ်ကြည့်တဲ့အခါ ဘယ်အော့ဂျက်နဲ့ပဲဖြစ်ဖြစ် \fCode{3} ပဲထွက်တာ တွေ့ရမှာပါ။ ဒါဟာ အော့ဘ်ဂျက်တစ်ခုစီက သီး\allowbreak သန့်ပိုင်ဆိုင်တဲ့ \fEn{attribute} နဲ့ အဓိကကွာခြားချက်ပါပဲ။ ဒီအချက်ကို ကွဲကွဲပြားပြား နားလည်ဖို့ အရေးကြီးပါတယ်။ \fCode{id} ကို ထုတ်ကြည့်ပါ

%
\begin{py}
print(smc1.id)    # 1 ß\fMM{ထွက်မှာပါ}ß    
print(smc2.id)    # 2
print(smc3.id)    # 3
\end{py}
%

\begin{mytcbox}
\fEn{Class attribute} တွေကို အော့ဘ်ဂျက် (ပိုတိကျအောင်ပြောရင် အော့ဘ်ဂျက်ကို ကိုယ်စားပြုတဲ့ ဗေရီရေဘဲလ်) နဲ့ ရော ကလပ်စ်နံမည်နဲ့ပါ အသုံးပြုလို့ ရပေမဲ့  \fEn{instance attribute} တွေကိုတော့ ကလပ်စ်နံမည်နဲ့ အသုံးပြုလို့ မရနိုင်ပါဘူး။ \fCode{SomeClass} ဥပမာမှာ အခုလို စမ်းကြည့်ရင်
%
\begin{pytc}
print(SomeClass.id)
print(SomeClass._id)
\end{pytc}
% 
ပထမ တစ်ခုက \mintinline{text}|<property object at 0x10dbf4bd0>| ထွက်နေပါတယ် (နောက်ဆုံးနံပါတ်က တစ်ခါနဲ့ တစ်ခါ တူမှာမဟုတ်ပါ)။ ဒုတိယတစ်ခုက \fEn{attribute} မရှိဘူးဆိုတဲ့ အယ်ရာပြတယ်။
\end{mytcbox}


 \fCode{\fEnEmp{var\_name}.\fEnEmp{attribute\_name}} ပုံစံနဲ့ \fEn{class attribute}  အသုံးပြုတဲ့အခါ လက်ရှိအော့ဘ်ဂျက်ရဲ့ \fEn{attribute} လား၊ \fEn{class attribute} လား မကွဲပြားတော့ဘူး။ ဥပမာ
%
\begin{codetxt}
>>> dt1 = date(1990, 11, 30)
>>> print(dt1.min)
0001-01-01
\end{codetxt}
%
ဒါကို ဖတ်တဲ့သူအနေနဲ့ \fCode{min} က \fEn{class attribute} လားဆိုတာ ရုတ်တရက် မသိနိုင်ပါဘူး။ \fCode{date} ရဲ့ \fEn{documentation} မှာ ကြည့်မှ \fCode{min} ဟာ \fEn{class attribute} လို့ ဖော်ပြထားတာ တွေ့ရမှာပါ။   ကလပ်စ်နံမည်နဲ့ \fCode{date.min} ဆိုရင်တော့ \fEn{class attribute} ပဲ ဖြစ်ရပါမယ်။ 

\begin{mytcboxflt}
\noindent \fSubSec{Class attribute နှင့် \fSubSecCodeBf{self}}
\betweentcboxpar
ကလပ်စ် ဘော်ဒီထဲမှာ ၎င်းကလပ်စ် ကိုယ်တိုင်ရဲ့ \fEn{class attribute} တွေကို \fEn{self.\fEnEmp{attribute\_name}} နဲ့ ရည်ညွှန်း အသုံးပြုနိုင်ပါတယ်။ ဒါပေမဲ့ \fEn{attribute} တန်ဖိုးကို သုံးလို့ပဲ ရမှာဖြစ်ပြီး အဆိုင်းမန့်လုပ်လို့ မရပါဘူး။ တကယ်လို့ အဆိုင်းမန့် လုပ်ရင် လက်ရှိအော့ဘ်ဂျက်မှာ (\fEn{class attribute} နဲ့ နံမည်တူ) ဗေရီရေဘဲလ် အသစ်တစ်ခု ရှိသွားမှာဖြစ်တယ်။ 
%
\begin{pytc}
class SomeClass:
    # class attributes
    MAX_ID = 9999
    cur_id = 0

    def __init__(self):
        # ß\fEn{class attribute}ß
        self._id = ß\textbf{self}ß.cur_id + 1
        # ß\fMM{လက်ရှိအော့ဘ်ဂျက်မှာ} \fCode{cur\_id} \fMM{ကြေငြာသလို ဖြစ်သွားမှာ၊} \fEn{class attribute}ß 
        # ß\fMM{ကို အဆိုင်းမန့် လုပ်တာမဟုတ်ဘူး}ß    
        ß\textbf{self}ß.cur_id = self._id         
\end{pytc}
%
\end{mytcboxflt}

\subsection*{Class Methods}
\fEn{Class attribute} နဲ့ ဆက်စပ်နေတဲ့ \fEn{class method} တွေကတော့ မက်သဒ်ကို ခေါ်ဖို့အတွက် အော့ဘ်ဂျက်ရှိစရာ မလိုဘူး။ တစ်နည်းအားဖြင့် \fEn{class method} တွေက ကလပ်စ်တစ်ခုရဲ့ အခြေအနေအောက်မှာ လုပ်ဆောင်တာ။ အဲဒီ \fEn{class instance} ရဲ့ အခြေအနေအောက်မှာ လုပ်ဆောင်တာ မဟုတ်ဘူး။ မက်သဒ်ခေါ်တဲ့အခါ \fCode{\fEnEmp{ClassName}.\fEnEmp{method\_name}()} ပုံစံနဲ့ ခေါ်လို့ရတယ်။

%
\begin{py}
class Joker:
    # class attributes
    CLASS_NICK_NAME = 'NaiveIdiot'
    joker_ids = []
    last_id = 0

    # class method 
    @classmethod
    def get_info(cls):
        return {'CLS_NICK_NAME': cls.CLASS_NICK_NAME,
                'NXT_JOKER_ID': cls.last_id + 1,
                'CUR_JOKERS_IDs': cls.joker_ids}

    def __init__(self, name):
        self._id = Joker.last_id + 1
        Joker.joker_ids.append(self._id)
        Joker.last_id = self._id
        self._name = name
\end{py}
%
\betweenminted{\medskipamount}
%
\begin{py}
jk1 = Joker('Deadly Bee')
jk2 = Joker('Happy Hi')

print(Joker.get_info())
\end{py}
%
\fEn{Output:}
\begin{codetxt}
{'CLS_NICK_NAME': 'NaiveIdiot', 'NXT_JOKER_ID': 3, 'CUR_JOKERS_IDs': [1, 2]}
\end{codetxt}
\fCode{get\_info} က ကလပ်စ် မက်သဒ် ဖြစ်ပါတယ်။ \fCode{@classmethod} \fEn{decorator} က မက်သဒ် တစ်ခုကို ‘ကလပ်စ် မက်သဒ် ဖြစ်ပါတယ်လို့’ ကြေငြာတဲ့ အဓိပ္ပါယ်ပါ။ ပုံမှန် မက်သဒ်ဆိုရင် ပထမဆုံး ပါရာမီတာက လက်ရှိအော့ဘ်ဂျက်ကို ရည်ညွှန်းတဲ့ \fCode{self} ဖြစ်တယ်။ ကလပ်စ် မက်သဒ် ဆိုရင်တော့ အဲ့ဒီအစား ပထမဆုံး ပါရာမီတာက ‘လက်ရှိကလပ်စ်’ ကို ရည်ညွှန်းတဲ့ \fCode{cls} ဖြစ်ရမှာပါ။ \fCode{cls} ပါရာမီတာနဲ့ \fEn{class attribute} တွေကို အသုံးပြုလို့ ရတယ်။ \fCode{Joker} ကလပ်စ် \fCode{get\_info} မှာ \fEn{class attribute} တွေသုံးထားတာကို တွေ့ရမှာပါ။ \fEn{Class method} ခေါ်တဲ့အခါ \fCode{cls} နေရမှာ လက်ရှိအသုံးပြုနေတဲ့ ကလပ်စ် ဝင်လာမှာပါ။ \fEn{Python} က အလိုအလျှောက် ထည့်ပေးတာဖြစ်တယ်။

\begin{mytcboxflt}
\fSubSec{Built-in Decorators}    
\betweentcboxpar
\fEn{Python} မှာ \fCode{@property}\fEn{,} \fCode{@staticmethod}\fEn{,} \fCode{@classmethod} စတဲ့  \fEn{built-in} ပါပြီးသား \fEn{decorator} တွေပါရှိတယ်။ \fEn{Decorator} အမျိုးအစားအလိုက် လိုအပ်တဲ့ အသွင်ပြောင်းလဲမှု \fEn{(transformation)} တွေ \fEn{Python} က လုပ်ပေးသွားမှာ ဖြစ်တယ်။ ဥပမာ \fCode{@property} \fEn{decorator} က မက်သဒ်တွေကို  \fEn{attribute} လို သုံးလို့ရအောင် လုပ်ပေးတယ်။ \fCode{@staticmethod} နဲ့ \fCode{@classmethod}  ကလည်း မက်သဒ်တစ်ခုကို \fEn{class method} နဲ့ \fEn{static method} အနေနဲ့ အသုံးပြုလို့ရအောင် လိုအပ်တဲ့ ပြောင်းလဲမှုတွေ လုပ်ပေးမှာဖြစ်တယ်။
 \betweentcboxpar
\fEn{Decorator} တွေရဲ့  နောက်ကွယ်မှာ အလုပ်လုပ်ပေးတာက ဖန်ရှင်တွေပါပဲ။ ကိုယ်ပိုင် \fEn{decorator} သတ်မှတ်လို့လည်း ရပါတယ်။ အတွေ့အကြုံရှိ ပရိုဂရမ်မာ ဖြစ်လာတော့မှ လေ့လာကြလေ့ရှိပြီး ဒီစာအုပ်မှာတော့ ၎င်းတို့ကို သီးသန့်ဖော်ပြမှာ မဟုတ်ပါဘူး။ 
\end{mytcboxflt}

\section{Static Methods}
\fEn{Static method} တွေကတော့ ကလပ်စ်ထဲမှာ သတ်မှတ်ထားပေမဲ့ လက်ရှိအော့ဘ်ဂျက် \fCode{self} ကိုရော လက်ရှိကလပ်စ် \fCode{cls} ကိုပါ အသုံးပြုဖို့ မလိုအပ်တဲ့ မက်သဒ်တွေပါ။ \fCode{@staticmethod} \fEn{decorator} နဲ့ ကြေငြာရတယ်။ ရိုးရိုး \fEn{top level} ဖန်ရှင်လိုပဲ \fCode{cls} (သို့) \fCode{self} ပါရာမီတာ မလိုဘူး။  ငွေလွှဲ ကိစ္စအတွက် \fCode{transfer} မက်သဒ်ကို \fEn{static method} အနေနဲ့ ဥပမာ ပြထားတာကြည့်ပါ။ 
%
\begin{py}
class Account:
    def __init__(self, holder, acc_number, balance, branch):
        # ...
        self._balance = balance
        # ...
    # ...        
    @staticmethod
    def transfer(from_acc, to_acc, amt):
        if amt > from_acc._balance:
            raise ValueError('Not enough balance!')
        from_acc._balance -= amt
        to_acc._balance += amt
\end{py}
%
ဒီ မက်သဒ် ခေါ်တဲ့အခါ \fEn{class method} လိုပဲ ကလပ်စ်နံမည်နဲ့ ခေါ်လို့ရသလို \fEn{instance} နဲ့ ခေါ်ချင်လည်း ရတယ်။
%
\begin{py}
Account.transfer(acc1, acc2, Decimal('50_000.00'))
acc1.transfer(acc1, acc2, Decimal('50_000.00'))
\end{py}
%
ကလပ်စ်နံမည်နဲ့ ခေါ်တာ ပိုရှင်းပါတယ်။

မှတ်ချက်။\qquad ။ \fCode{transfer} မက်သဒ်ကို \fEn{instance method} အနေနဲ့ပဲ ရေးထားတာကို လာမဲ့ အသုံးချဥပမာမှာ တွေ့ရပါမယ်။ \fEn{Instance method} ဆိုရင် ခေါ်တဲ့အခါ
%
\begin{py}
acc1.transfer(acc2, Decimal('50_000.00'))
\end{py}
%
ဖြစ်ပါမယ်။

\fEn{Static method} ဟာ ဖိုင်တစ်ခုရဲ့ \fEn{top level} မှာ ကြေငြာတဲ့ ပုံမှန်ဖန်ရှင်တွေလိုပဲ \fCode{cls} (သို့) \fCode{self} မပါတဲ့အတွက် ကလပ်စ်ထဲအပြင်ကို ထုတ်လိုက်ရင် ရိုးရိုးဖန်ရှင် ဖြစ်သွားမှာပါ။ 
%
\begin{py}
# @staticmethod ß\fEn{decorator} \fMM{လည်း မလိုတော့ဘူး}ß
def transfer(from_acc, to_acc, amt):
    if amt > from_acc._balance:
        raise ValueError('Not enough balance!')
    from_acc._balance -= amt
    to_acc._balance += amt

class Account:
    def __init__(self, holder, acc_number, balance, branch):
        # ...
        self._balance = balance
        # ...
    # transfer ß\fMM{ကို ကလပ်စ်ထဲက ထုတ်လိုက်တယ်}ß
\end{py}
%
ဒါပေမဲ့ ငွေလွှဲတယ်ဆိုတာ ဘဏ်အကောင့်နဲ့ သက်ဆိုင်တဲ့ ကိစ္စဖြစ်တဲ့အတွက် \fCode{Account} ကလပ်စ်ထဲမှာ \fEn{static method} အနေနဲ့ ရှိတာက ပိုပြီး သဘာဝကျပါတယ်။  

\subsection*{မက်သဒ်နှင့် attribute အမျိုးအစားများ အကြား အပြန်အလှန် အသုံးပြုနိုင်မှု}
% https://stackoverflow.com/questions/23508248/why-do-we-use-staticmethod
% 
\fEn{Instance method} က \fEn{instance/class attribute} နဲ့ \fEn{instance/class/static method} တွေကို အသုံးပြုလို့ရတယ်။ \fEn{Class method} နဲ့ \fEn{static method} ကတော့ \fEn{instance attribute/method} တွေကို သုံးလို့ရမှာ မဟုတ်ပါဘူး။ အောက်ပါ ဇယားနှစ်ခုမှာ အမျိုးအစားအလိုက် အသုံးပြုလို့ ရ/မရ ခွဲခြားပြထားတာကို ကြည့်ပါ။

%
\begin{flushleft}
\vspace{1em}
\setlength{\extrarowheight}{3pt}
\begin{tabular}[h]{*{3}l l l}
    \toprule[1.5pt]
        \fTblHead{Method Type} & \fTblHead{Class Attribute}         & \fTblHead{Instance Attribute} \\       
    \midrule
    \fEn{Static}               & \fEn{Yes (class name)}             & \fEn{No}                      \\
    \fEn{Class}                & \fEn{Yes (\fCode{cls}/class name)} & \fEn{No}                       \\
    \fEn{Instance}             & \fEn{Yes (\fCode{self}/class name)}& \fEn{Yes (\fCode{self})}      \\   
    \bottomrule[1.5pt]
\end{tabular}
\label{tbl:ch09accessBetweenMethodAndAttributeType}
\captionof{table}{မက်သဒ်အမျိုးအစားနှင့်  attribute အမျိးအစား အကြား အသုံးပြုနိုင်မှု}
\end{flushleft}
%

%
\begin{flushleft}
\vspace{1em}
\setlength{\extrarowheight}{3pt}
\begin{tabular}[h]{*{3}l l l}
    \toprule[1.5pt]
        \fTblHead{Method Type} & \fTblHead{Static Method} & \fTblHead{Class Method} & \fTblHead{Instance Method} \\       
    \midrule
    \fEn{Static} & \fEn{Yes (class name)} & \fEn{Yes (class name)} & \fEn{No} \\
    \fEn{Class} &\fEn{Yes (\fCode{cls}/class name)}  & \fEn{Yes (\fCode{cls}/class name)} & \fEn{No} \\
    \fEn{Instance} & \fEn{Yes (\fCode{self}/class name)} & \fEn{Yes (\fCode{self}/class name)} & \fEn{Yes (\fCode{self})} \\   
    \bottomrule[1.5pt]
\end{tabular}
\label{tbl:ch09accessBetweenMethodTypes}
\captionof{table}{မက်သဒ်အမျိုးအစားများ အကြား အသုံးပြုနိုင်မှု}
\end{flushleft}
%
ဇယားနှစ်ခုလုံးမှာ \fEn{class name/\fCode{cls}/\fCode{self}} ဘာနဲ့ အသုံးပြုလို့ရလဲ ကိုလည်း ပြထားတယ်။ ဥပမာ \fEn{instance method} ဆိုရင် \fEn{class attribute} ကို \fEn{class name} (သို့) \fCode{self} နဲ့ သုံးလို့ရမှာပါ (ပထမ ဇယား အောက်ဆုံး \fEn{row} ပထမ \fEn{column})။ \fEn{Class method} က \fEn{static method} ကို \fCode{cls} (သို့) \fEn{class name} နဲ့ သုံးလို့ရပါမယ် (ဒုတိယ ဇယား ဒုတိယ \fEn{row} ပထမ \fEn{column})။ အခုလို ကုဒ်ရေးပြီး စမ်းသပ်စစ်ဆေး ထားတာကိုလည်း လေ့လာကြည့်ပါ။

%
\begin{py}
# File: accessibility_test.py
# can be used inside the class
top_level_var = 'top level var'


# can be called inside the class
def top_level_fun():
    pass


class AccessibilityTest:

    class_attr = 'class attr'

    def __init__(self):
        self.instance_attr = 'instance  attr'
        self.instance_attr2 = 'instance  attr2'

    @staticmethod
    def static_method():
        # print(self.instance_attr)   # cannot use
        print(AccessibilityTest.class_attr)
        # can only call static and class method types
        AccessibilityTest.static_method2()
        AccessibilityTest.class_method2()
        # self.instance_method()      # cannot use

    @staticmethod
    def static_method2():
        pass

    @classmethod
    def class_method(cls):
        # print(self.instance_attr)  # cannot use
        print(cls.class_attr)
        print(AccessibilityTest.class_attr)
        # can only call static and class method types
        AccessibilityTest.static_method()
        cls.class_method2()
        AccessibilityTest.class_method2()
        # self.instance_method()      # cannot use

    @classmethod
    def class_method2(cls):
        pass

    def instance_method(self):
        print(self.instance_attr)
        print(AccessibilityTest.class_attr)
        # can call all type of methods
        AccessibilityTest.static_method()
        AccessibilityTest.class_method()
        self.instance_method2()

    def instance_method2(self):
        print(self.instance_attr2)


# just test
avar1 = AccessibilityTest()
AccessibilityTest.static_method()
print()
AccessibilityTest.class_method()
print()
avar1.instance_method()
\end{py}
%

\section{Object-Oriented Programming}
ကလပ်စ်နဲ့ အော့ဘ်ဂျက်တွေ အကြောင်း အထိုက်အလျှောက် လေ့လာပြီးနောက် ၎င်းတို့ကို ပရိုဂရမ် တည်\allowbreak ဆောက်ရာမှာ အသုံးချတတ်အောင် ဆက်လက် လေ့လာကြရပါမယ်။ လက်တွေ့ရေး လေ့ကျင့်ဖို့ လိုပါတယ်။ အော့ဘ်ဂျက်သဘောတရား အခြေပြု ပရိုဂရမ်ရေးသား တည်ဆောက်တာကို \fEn{\textit{Object-oriented Programming} (OOP)} လို့ခေါ်ပါတယ်။ ဒီနည်းနဲ့ တည်ဆောက်ထားတဲ့ ပရိုဂရမ်တွေကို \fEn{object-oriented program/software} လို့ ဆိုနိုင်ပါတယ်။ စလေ့လာသူတွေ အတွက် အထောက်အကူဖြစ်စေမဲ့ ရိုးရှင်းတဲ့ နမူနာ \fEn{object-oriented program} လေးတစ်ခုကို လေ့လာကြည့်ပါမယ်။ 

\subsection*{A Simple Banking Application}
\fEn{S Bank} ဆိုတဲ့ ဘဏ်အတွက် ပရိုဂရမ်တစ်ခု ရေးပေးရမယ်လို့ စိတ်ကူးကြည့်ပါ။ ဘဏ်အကောင့် အသစ်ဖွင့်တဲ့အခါ အောက်ပါ အချက်အလက်တွေ လိုအပ်တယ်။ \fEn{S Bank} က လောလောဆယ် ရန်ကုန်၊ မန္တလေးနဲ့ နေပြည်တော် ဘဏ်ခွဲ သုံးခုပဲ ရှိတယ်။ 
%
\begin{minted}[frame=lines, framerule=0pt,escapeinside=ßß]{text}
ß\fEnBf{Open Account}ß
ß\fEn{Name: Kathy}ß
ß\fEn{NRIC: 12/ThaGaKa (N) 283772}ß
ß\fEn{Gender (F/M): F}ß
ß\fEn{Date of Birth: 12/12/1990}ß
ß\fEn{Address Line 1: 45A, Marlar St., 16 Ward}ß
ß\fEn{Address Line 2: South Okkalapa}ß
ß\fEn{City: Yangon}ß
ß\fEn{State/Division: Yanon}ß
ß\fEn{Initial deposit: 350000.00}ß
ß\fEn{Branch: Yangon}ß
ß\fEn{Please confirm (Y/N): Y}ß

ß\fEn{Account No: 100000000012}ß
ß\fEn{Successfully created}ß
\end{minted}
%
အကောင့်နံပါတ်က ဂဏန်း (၁၂) လုံး ဖြစ်ရမယ်။ \fEn{100000000001} က ပထမဆုံးဖွင့်တဲ့ အကောင့်နံပါတ်  ဖြစ်ပြီး အကောင့်သစ် တစ်ခုတိုင်းအတွက် နံပါတ်စဉ် တိုး သွားရမှာပါ။

အကောင့်ထဲကို ငွေသွင်းမယ်ဆိုရင် အကောင့်နံပါတ်နဲ့ သွင်းမဲ့ ငွေပမာဏ ထည့်ပေးရပါမယ်။ အကောင့်နံပါတ် ထည့်ပေးတာ မှားခဲ့ရင် အခြားအကောင့်တစ်ခုထဲကို ငွေသွင်းမိနိုင်တယ်။ အကောင့်ပိုင်ရှင် နံမည်ကို ပြပေးပြီး ကွန်ဖန်းလုပ်ရင် အမှားဖြစ်နိုင်ခြေ ပိုပြီးနည်းသွားမှာပါ။ ငွေထုတ်ရင်လည်း ဒီလိုပဲ ကွန်ဖန်းလုပ်သင့်တယ်။

%
\begin{minted}[frame=lines, framerule=0pt,escapeinside=ßß]{text}
ß\fEnBf{Deposit}ß
ß\fEn{Account No: 100000000012}ß
ß\fEn{Amount: 250000}ß
ß\fEn{Please check and confirm the following details:}ß
    ß\fEn{Holder: Kathy}ß
    ß\fEn{Account No: 100000000012}ß
    ß\fEn{Amount to deposit: 250000}ß
ß\fEn{Please confirm (Y/N): Y}ß 
ß$\ldots$ß
ß\fEn{Account transaction successfully completed.}ß
\end{minted}
%

%
\begin{minted}[frame=lines, framerule=0pt,escapeinside=ßß]{text}
ß\fEnBf{Withdrawal}ß
ß\fEn{Account No: 100000000012}ß
ß\fEn{Amount: 50000}ß
ß\fEn{Please check and confirm the following details:}ß
    ß\fEn{Holder: Kathy}ß
    ß\fEn{Account No: 100000000012}ß
    ß\fEn{Amount to withdraw: 50000}ß
ß\fEn{Please confirm (Y/N): Y}ß 
ß$\ldots$ß
ß\fEn{Account transaction successfully completed.}ß
\end{minted}
%

ငွေလွှဲမယ်ဆိုရင် လွှဲမဲ့အကောင့်၊ လက်ခံမဲ့အကောင့် နှစ်ခုလိုမယ်။ လွှဲမဲ့အကောင့်နဲ့ လက်ခံအကောင့် တစ်ခုတည်း ဖြစ်မနေသင့်ပါဘူး။ 
%
\begin{minted}[frame=lines, framerule=0pt,escapeinside=ßß]{text}
ß\fEnBf{Transfer}ß
ß\fEn{From Account No: 100000000012}ß
ß\fEn{To Account No: 100000000015}ß
ß\fEn{Amount: 50000}ß
ß\fEn{Please check and confirm the following details:}ß
ß\fEn{From}ß
    ß\fEn{Holder: Kathy}ß
    ß\fEn{Account No: 100000000012}ß
ß\fEn{To}ß
    ß\fEn{Holder: Sandar}ß
    ß\fEn{Account No: 100000000015}ß
ß\fEn{Amount: 50000}ß
ß\fEn{Please confirm (Y/N): Y}ß 
ß$\ldots$ß
ß\fEn{Account transaction successfully completed.}ß
\end{minted}
%

ဒီပရိုဂရမ်လေးက အပြင်မှာ တကယ်သုံးလို့မရပေမဲ့ တတ်နိုင်သမျှ ယုတ္တိကျအောင်၊ အပြင်နဲ့ နီးစပ်အောင် စဉ်းစားပြီး ရေးသင့်ပါတယ်။ ဥပမာ အကောင့်နံပါတ်မှားရင် ဘယ်လိုပြပေးသင့်သလဲ၊ ထုတ်မဲ့ ငွေပမာဏက လက်ကျန်ငွေထက် ပိုများနေတဲ့အခါ ဘယ်လိုဖြစ်သင့်လဲ စတာတွေကို သေသေချာချာ စဉ်းစားပေးသင့်တယ်။ အခြေခံအဆင့် ဥပမာမို့လို့ အလွန်အကျွံ ရှုပ်ထွေးလို့တော့လည်း မရသေးဘူးပေါ့။ ဘီဂင်နာအဆင့်နဲ့ သင့်တော်လောက်မဲ့ ပုံစံမျိုးလေးနဲ့ တစ်ပိုင်းချင်း တည်ဆောက်သွားရအောင်။

ပရိုဂရမ်တစ်ခု စတည်ဆောက်ဖို့ ကြိုးစားတဲ့အခါ ဘယ်ကစရမယ်မှန်းမသိ အစရှာလို့မရ ဖြစ်နေတတ်ပါတယ်။ ဒီလိုအခြေအနေမျိုးမှာ \fEn{problem decomposition} ကို အသုံးချရမှာပါ။ ဖြေရှင်းမဲ့ ကိစ္စကို အပိုင်းတွေ ခွဲကြည့်ပြီး ဘယ်တစ်ခုကို အရင်ဆုံး တည်ဆောက်ရမလဲ ရွေးချယ်ရပါမယ်။ ဖော်ပြထားချက်တွေအရ ဘဏ်ပရိုဂရမ်က အကောင့်ဖွင့်တာ၊ ငွေသွင်း၊ ငွေထုတ်နဲ့ ငွေလွှဲ ကိစ္စတွေ ဆောင်ရွက်ပေးရမှာပါ။ အကောင့်ဖွင့်တဲ့ အပိုင်းကို ပထမဆုံး တည်ဆောက်ပါမယ်။ ဘာလို့ ဒါနဲ့စတင်မယ်လို့ ဆုံးဖြတ်တာလဲ ခိုင်လုံတဲ့ အကြောင်းပြချက် သိပ်တော့ကြီးကြီးမားမားရယ် မရှိပါဘူး။ (ငွေသွင်း/ထုတ် တွေ လုပ်နိုင်ဖို့က အကောင့်အရင် ဖွင့်ထားမှ လုပ်လို့ရမှာဖြစ်လို့ ဒီကနေ စသင့်တယ်လို့တော့ မြင်မိတယ်)။ ဒါပေမဲ့ ငွေသွင်းတဲ့အပိုင်းနဲ့ စမယ်ဆိုရင်လည်း ရတာပါပဲ။ 

\fEn{Object-oriented Programming} နည်းနဲ့ ပရိုဂရမ်ရေးတဲ့အခါ ဖော်ပြထားချက်ထဲမှာ ပါဝင်တဲ့အကြောင်းအရာတွေကို လေ့လာသုံးသပ်ပြီး ဘယ်အရာတွေကို အော့ဘ်ဂျက်တွေအနေနဲ့ ဖော်ပြရမလဲ ဆုံးဖြတ်ရတယ်။ အကောင့်၊ အကောင့် ပိုင်ရှင်၊ လိပ်စာ၊ ဘဏ် နဲ့ ဘဏ်ခွဲ တို့ကို အော့ဘ်ဂျက်အနေနဲ့ ရှုမြင်နိုင်တယ်။ ဒါ့ကြောင့် \fCode{Account}\fEn{,} \fCode{Holder}\fEn{,} \fCode{Address}\fEn{,} \fCode{Bank} နဲ့  \fCode{Branch} ကလပ်စ်တွေ သတ်မှတ်ပါမယ်။ \fCode{Account} ကို အရင်ကြည့်ရအောင် $\ldots$

\subsection*{\fSubSecCodeBf{Account} Class}
\fCode{Account} ကလပ်စ်ကို ရှေ့ပိုင်းက ဥပမာတွေမှာလည်း တွေ့ခဲ့ကြပြီးပါပြီ။ ဘဏ်ပရိုဂရမ်အတွက် လိုအပ်တာ တချို့ကို ထပ်ဖြည့်ထားပါတယ်။
%
\begin{py}
# File: bank_app.py
class Account:
    _next_acc_no = 100_000_000_001

    def __init__(self, holder, balance, branch):
        self._holder = holder
        self._acc_number = Account._next_acc_no
        Account._next_acc_no += 1
        self._balance = balance
        self._branch = branch

    def deposit(self, amt):
        if amt <= Decimal(0.00):
            raise ValueError('Invalid amount!')
        self._balance += amt

    def withdraw(self, amt):
        if amt <= Decimal(0.00):
            raise ValueError('Invalid amount!')
        if amt > self._balance:
            raise ValueError('Not enough balance!')
        self._balance -= amt

    def transfer(self, to_acc, amt):
        if self.acc_number == to_acc._acc_number:
            raise ValueError('From-account and to-account should not be '
                             'the same!')
        if amt <= Decimal(0.00):
            raise ValueError('Invalid amount!')
        if amt > self._balance:
            raise ValueError('Not enough balance!')
        self._balance -= amt
        to_acc._balance += amt

    @property
    def holder(self):
        return self._holder

    @property
    def acc_number(self):
        return self._acc_number

    @property
    def balance(self):
        return self._balance

    @property
    def branch(self):
        return self._branch
\end{py}
%
ရှေ့မှာဖော်ပြခဲ့တဲ့ လိုအပ်ချက်အရ အကောင့်နံပါတ်မှာ ဂဏန်း (၁၂) လုံး ရှိရမယ်။ ပထမဆုံး ဖွင့်တဲ့အကောင့် နံပါတ်က \fEn{100000000001} ဖြစ်ရမယ်။ နောက်ထပ် အကောင့်သစ်တစ်ခု ဖွင့်တိုင်း နံပါတ်စဉ် တစ်ခုချင်း တိုးသွားရမှာပါ။ ဒီလိုအပ်ချက်အတွက် \fEn{class attribute} ကို အသုံးပြုနိုင်ပါတယ်
%
\begin{py}
_next_acc_no = 100_000_000_001
\end{py}
%
\fEn{Initializer} ထဲမှာ ဒီ \fEn{attribute} ကို ဘယ်လို သုံးထားလဲ လေ့လာကြည့်ပါ။ လက်ရှိအော့ဘ်ဂျက်ရဲ့ အကောင့်နံပါတ်အဖြစ် သုံးပြီးတာနဲ့ နောက်ဖန်တီးမဲ့ အော့ဘ်ဂျက်အတွက် အသင့်ဖြစ်အောင် တစ်တိုးပေးထားတာ တွေ့ရမှာပါ။ မက်သဒ်တွေနဲ့ \fEn{instance attribute} တွေကတော့ ရှေ့က ဥပမာတွေမှာ တွေ့ပြီးဖြစ်တဲ့အတွက် သိပ်ထွေထွေထူးထူး ရှင်းပြဖို့ မလိုလောက်တော့ဘူး။ \fCode{transfer} မက်သဒ်ကတော့ နည်းနည်းထူးခြားတာလေးတွေ ပါနေတယ်
%
\begin{py}
def transfer(self, to_acc, amt):
    # ß$\ldots$ \fMM{(တချို့လိုင်းတွေ မပြထား)}ß
    self._balance -= amt
    to_acc._balance += amt # ß\fMM{မသုံးစေချင်တာကို ယူသုံးထားတာ ဖြစ်မနေဘူးလား}ß
\end{py}
%
\fCode{self.\_balance} က ပြဿနာမရှိပေမဲ့ \fCode{to\_acc} ရဲ့  \fCode{\_balance} ကို တိုက်ရိုက် ယူသုံးထားတာ \fEn{access control} စည်းကမ်းဖောက်သလို ဖြစ်မနေဘူးလား မေးစရာရှိပါတယ်။ အခု ဒီလိုမျိုး သုံးထားတာက \fCode{Account} ကလပ်စ်ထဲမှာပါ။ \fCode{to\_acc} ကလည်း အခြား \fCode{Account} အော့ဘ်ဂျက် တစ်ခုဖြစ်တယ်။ လက်ရှိ ကလပ်စ်သတ်မှတ်သူဟာ အဲ့ဒီကလပ်စ်အမျိုးအစား အော့ဘ်ဂျက်တွေရဲ့ \fEn{attribute} တွေကို တိုက်ရိုက်သုံးတာက ပြဿနာမရှိဘူးလို့ ယူဆရပါမယ်။ အကြောင်းအရင်းက ဒီလိုပါ။ ပိုင်ရှင်ဟာ သူ့ကလပ်စ် \fEn{attribute} (ပိုတိကျအောင် ပြောရင် သူ့ကလပ်စ်ကနေ ဖန်တီးယူလိုက်တဲ့ အော့ဘ်ဂျက်ရဲ့ \fEn{attribute}) တွေကို အခြားသူတွေကိုပဲ တိုက်ရိုက် အသုံးမပြုစေချင်တာ။ သူကိုယ်တိုင်အနေနဲ့တော့ သုံးလို့ရသင့်ပါတယ်။ ဘာဖြစ်လို့လဲဆိုတော့ သူကိုယ်တိုင်က ပိုင်ရှင် ဖြစ်တဲ့အတွက် ကလပ်စ်ကို ပြင်ဆင်ပြောင်းလဲတဲ့အခါ ဒီလိုတိုက်ရိုက်ယူသုံးထားတဲ့အပေါ် ပြဿနာ ရှိ/မရှိ သူ့အနေနဲ့ သုံးသပ်နိုင်ပါတယ် (သူရေးထားတဲ့ ကလပ်စ်ပဲလေ၊ ဘာလုပ်ရင် ဘာဖြစ်မယ် သူသိရမှာပေါ့)။ ဒီအတွက် တိုက်ရိုက်ယူသုံးထားတဲ့ နေရာတွေကိုလည်း လိုအပ်သလို အလိုက်သင့် ပြင်ဆင်ပြောင်းလဲ ပေးနိုင်ပါတယ်။ ဒီလောက်ဆိုရင် ကလပ်စ်ပိုင်ရှင် အနေနဲ့ အဲ့ဒီကလပ်စ် အော့ဘ်ဂျက်တွေရဲ့ \fEn{attribute} တွေကို တိုက်ရိုက် အသုံးပြုလို့  ရသင့်တယ်ဆိုတဲ့ အကြောင်းပြချက်ဟာ ကျိုးကြောင်းဆီလျော် ရှိတယ်လို့ စာဖတ်သူ လက်ခံနိုင်မယ် မျှော်လင့်ပါတယ်။

ကလပ်စ်တစ်ခုမှာ ပါဝင်သင့်တဲ့ မက်သဒ်နဲ့ \fEn{attribute} တွေကို တစ်ခါတည်းနဲ့ အားလုံးပြီးပြည့်စုံအောင် ကြိုတင်စဉ်းစားထားဖို့ မလိုအပ်ပါဘူး။ စဉ်းစားလို့ရတာတွေနဲ့ပဲ စသတ်မှတ်ထားနိုင်တယ်။ ပရိုဂရမ်တည်ဆောက်ရင်းနဲ့ လိုအပ်တာတွေ ထပ်ပြီး တွေ့လာ၊ မြင်လာတဲ့ အခါကျမှ ပြင်ဆင်ဖြည့်စွက်လို့ရပါတယ်။ ကလပ်စ်တစ်ခုမှာပဲ လည်နေပြီး ရှေ့မဆက်နိုင်တာဟာ စလေ့လာသူ ဖြစ်လေ့ရှိတဲ့ အမှားတစ်ခုပါ။ 
\subsection*{\fSubSecCodeBf{Address} and \fSubSecCodeBf{Holder} Classes}
\fCode{Address} နဲ့ \fCode{Holder} ကို အခုလို သတ်မှတ်ထားပါတယ်။ \fCode{Account} နဲ့ နှိုင်းယှဉ်ရင် အတော်လေး ရိုးရှင်းတာ တွေ့ရမှာပါ။

%
\begin{py}
class Address:
    def __init__(self, line_1, line_2, city, division):
        self._address_line_1 = line_1
        self._address_line_2 = line_2
        self._city = city
        self._division = division

    @property
    def address_line_1(self):
        return self._address_line_1

    @property
    def address_line_2(self):
        return self._address_line_2

    @property
    def city(self):
        return self._city

    @property
    def division(self):
        return self._division
\end{py}
%
\betweenminted{\medskipamount}
%
\begin{py}
class Holder:
    def __init__(self, name, dob, nric, gender, address):
        if name.strip() == '':
            raise ValueError('Name cannot be empty!')
        if age_today(dob) < 18:
            raise ValueError('Too young to open a bank account!')
        if nric.strip() == '':
            raise ValueError('Name cannot be empty!')
        if not (gender.upper() == 'F'
                or gender.upper() == 'M'):
            raise ValueError('Unknown gender!')
        self._name = name
        self._dob = dob
        self._nric = nric
        self._gender = gender.upper()
        self._address = address

    @property
    def age(self):
        return age_today(self._dob)

    @property
    def name(self):
        return self._name

    @name.setter
    def name(self, name):
        if name.strip() == '':
            raise ValueError('Name cannot be empty!')
        self._name = name

    @property
    def dob(self):
        return self._dob

    @dob.setter
    def dob(self, value):
        self._dob = value

    @property
    def gender(self):
        return 'FEMALE' if self._gender == 'F' else 'MALE'
\end{py}
%
\fCode{Holder} ရဲ့ \fCode{age} \fEn{property} မှာ \fCode{age\_today} ဖန်ရှင် သုံးထားတယ်။  \fEn{Insurance premium} ပရိုဂရမ်မှာလည်း ဒီဖန်ရှင်ကို သုံးထားတယ်။ ပရိုဂရမ်တွေမှာ ဒီလိုမျိုး မကြာခဏ ပြန်အသုံးပြုရတဲ့ ဖန်ရှင်တွေကို မော်ဒျူး (သို့) လိုက်ဘရီတစ်ခုအနေနဲ့ ထုတ်ထားလို့ရတယ်။ လိုတဲ့အခါ \fEn{import} လုပ်ပြီး သုံးရုံပဲ။ ပြန်ရေးဖို့ မလိုတော့ဘူး။ မော်ဒျူးအကြောင်းကို နောက်အခန်းတွေမှာ ဆက်လက်လေ့လာကြမှာပါ။ လောလောဆယ်တော့ ဘဏ်ပရိုဂရမ်ဖိုင် \fEn{(\fEnSnd{bank\_app.py})} ရဲ့ \fEn{top level} မှာ ပြန်ရေးပါမယ်။ (\fEn{Copy and paste} လုပ်လည်းရတယ်)။
%
\begin{py}
# File: bank_app.py
# ß\fEn{top level} \fMM{ဖန်ရှင်ပါ၊ ကုဒ်ဖိုင်ရဲ့ အောက်မှာ တိုက်ရိုက်ရှိပါတယ်၊ ကလပ်စ်ထဲမှာမဟုတ်}ß
def age_today(dob):
    today = date.today()
    this_bd = dob.replace(year=today.year)
    if today - dob >= this_bd - dob:
        return today.year - dob.year
    else:
        return today.year - dob.year - 1
\end{py}
%

\subsection*{\fSubSecCodeBf{Branch} and \fSubSecCodeBf{Bank} Classes}
\fCode{Branch} ကလပ်စ်ကတော့ ရှင်းပါတယ်။ နံမည်နဲ့ လိပ်စာပဲ ပါတယ်။ ပရိုဂရမ် လိုအပ်ချက်အရ နောက်ပိုင်း ထပ်ဖြည့်ရတာတွေတော့ ရှိလာနိုင်ပါတယ်။ ဥပမာ ဘဏ်ခွဲ စဖွင့်တဲ့ရက်၊ ဘဏ်ခွဲမှာ ရရှိနိုင်တဲ့ ဝန်ဆောင်မှုတွေ စသည်ဖြင့်။ အခုတော့ ဒါတွေ မလိုသေးပါဘူး။ ဖော်ပြထားချက်ကို ပြန်ကြည့်ရင် ဘဏ်ခွဲလိပ်စာကိုလည်း သုံးဖို့ မလိုသေးတာ တွေ့ရမှာပါ။ အခုပရိုဂရမ် လိုအပ်ချက်အရဆိုရင် ဘဏ်ခွဲကို သီးခြား အော့ဘ်ဂျက်အနေနဲ့ မယူဆလည်း ရတယ်။ ဘဏ်ခွဲနံမည်ကို စာသားအနေနဲ့ သိမ်းထားရင် ရပြီ။ ဒါပေမဲ့ ဘဏ်ခွဲဟာ အပြင်မှာတကယ်ရှိတဲ့ အရာဖြစ်တယ်။ နံမည်အပြင် ဘဏ်ခွဲနဲ့ပါတ်သက်တဲ့ အခြား အချက်အလက်တွေကိုလည်း နောက်ပိုင်းမှာ လိုအပ်လာနိုင်တယ်လို့ မျှော်မှန်းရပါတယ်။ ဒါကြောင့် အော့ဘ်ဂျက်ဖြစ်သင့်တယ်လို့ ယူဆပါမယ်။ 
%
\begin{py}
class Branch:
    def __init__(self, name, address):
        self._name = name
        self._address = address

    @property
    def name(self):
        return self._name
\end{py}
% 
မှတ်ချက်။\qquad ။ ဒီနေရာမှာ  ဘဏ်ခွဲသည် အော့ဘ်ဂျက် ဖြစ်ကိုဖြစ်ရမယ်လို့ တစ်ထစ်ချမယူဆဘဲ လိုအပ်\allowbreak ချက်ပေါ်မူတည်ပြီး ဆုံးဖြတ်နိုင်တယ် ဆိုတာကို အဓိကပြောချင်တာပါ။ မလိုအပ်ဘူး ယူဆရင် ဘဏ်ခွဲနံမည်ကို စာသားအနေနဲ့ပဲ ထားပြီး နောင်လိုအပ်လာတော့မှ ကလပ်စ် သတ်မှတ်လည်း ရတယ်။ 

\fCode{Bank} ကလပ်စ်ကို ကြည့်ရအောင်။ ဘဏ်မှာ အကောင့်တွေ ရှိပါမယ်။ အကောင့် ဖွင့်ထားတဲ့ ကာစတမ်မာတွေ ရှိပါမယ်။ ဘဏ်ခွဲ သုံးခုရှိမယ်။ ဆက်စပ်မှုက \fEn{has-a relationship} ပါ။ 
%
\begin{py}
class Bank:
    def __init__(self, name):
        self._name = name
        self._accounts = []
        self._acc_holders = []
        self._branches = [Branch('Yangon', Address('50A, Theinphyu St.',
                                                   'Botataung Township',
                                                   'Yangon', 'Yangon')),
                          Branch('Mandalay', Address('65B, 63 St.',
                                                     'Chanayetharsi',
                                                     'Mandalay',
                                                     'Mandalay')),
                          Branch('Naypyidaw', Address('332C, Main Rd.',
                                                      'Ottarathiri Township',
                                                      'Naypyidaw',
                                                      'Naypyidaw Union'))]

    def open_acc(self, holder, init_bal, branch_name):
        branch = self.retrieve_branch(branch_name)
        account = Account(holder, init_bal, branch)
        self._accounts.append(account)
        self._acc_holders.append(holder)
        return account

    def withdraw(self, acc_no, amt):
        acc = self.retrieve_account(acc_no)
        acc.withdraw(amt)

    def deposit(self, acc_no, amt):
        acc = self.retrieve_account(acc_no)
        acc.deposit(amt)

    def transfer(self, from_acc_no, to_acc_no, amt):
        from_acc = self.retrieve_account(from_acc_no)
        to_acc = self.retrieve_account(to_acc_no)
        from_acc.transfer(to_acc, amt)

    def retrieve_account(self, acc_no):
        for acc in self._accounts:
            if acc.acc_number == acc_no:
                return acc
        raise ValueError(f'Account with {acc_no} not found!')

    def retrieve_branch(self, name):
        for branch in self._branches:
            if branch.name == name:
                return branch
        raise ValueError(f'{name} branch not found!')
\end{py}
%
ဘဏ်ခွဲသုံးခုကို ကြိုပြီး ထည့်ထားတယ်။ လက်ရှိမှာ သုံးခုပဲရှိပေမဲ့ ဘဏ်ခွဲ အသစ်တွေ ထပ်ဖွင့်နိုင်တယ်။ ပရိုဂရမ် နောက်ပိုင်း ဗားရှင်းတွေမှာ ဘဏ်ခွဲအသစ် ထပ်ထည့်နိုင်ဖို့ လိုအပ်ချက်ရှိလာမယ်လို့ မျှော်မှန်းရပါတယ်။ အားလုံး သိပြီးဖြစ်တဲ့အတိုင်း ပရိုဂရမ်တွေက သူတို့ကို အသုံးပြုနေတဲ့ ကာလတစ်လျှောက် အမြဲပြင်ဆင်ဖြည့်စွက်တာ၊ ပြုပြင်ထိန်းသိမ်းတာတွေ လုပ်နေရမှာပါ။ ဗားရှင်းအသစ်တွေ မကြာခဏ \fEn{update} လုပ်ကြရပါတယ်။ တကယ့် အသုံးချ ဆော့ဖ်ဝဲတွေ တည်ဆောက်တဲ့အခါ ပြုပြင်ထိန်းသိမ်းရ လွယ်ကူအောင်၊ ဖီချာအသစ်တွေ ထပ်ထည့်ဖို့ မခက်ခဲအောင် ဒီဇိုင်းပြုလုပ်ထားရပါတယ်။ အခြေခံ ပရိုဂရမ်းမင်း တတ်ကျွမ်းပြီးတဲ့နောက်မှာ ဒီလို ဒီဇိုင်းပိုင်းဆိုင်ရာတွေ နားလည်အသုံးချတတ်လာအောင် ဆက်လက် လေ့\allowbreak လာဖို့ လိုအပ်ပါလိမ့်မယ်။

\fCode{Account} မှာ ပါတဲ့ မက်သဒ်တွေကို \fCode{Bank} မှာလည်း ပြန်တွေ့ရပါတယ်။ \fCode{transfer} ကို ကြည့်ရင် အခုလိုရေးထားတာ တွေ့ရတယ်
%
\begin{py}
def transfer(self, from_acc_no, to_acc_no, amt):
    from_acc = self.retrieve_account(from_acc_no)
    to_acc = self.retrieve_account(to_acc_no)
    from_acc.transfer(to_acc, amt)
\end{py}
%
အခု \fCode{Bank} ကလပ်စ် \fCode{transfer} က အကောင့် နံပါတ်နှစ်ခုနဲ့ ငွေပမာဏကို လက်ခံထားတယ်။ အဲ့ဒီနံပါတ်နဲ့ အကောင့်နှစ်ခုကို ဘဏ်ရဲ့ အကောင့်တွေအားလုံးထဲကနေ ဆွဲထုတ်တယ် (\fCode{retrieve\_account} မက်သဒ်ကို ကြည့်ပါ)။ ရလာတဲ့ အော့ဘ်ဂျက်ရဲ့ \fCode{transfer} ကို တစ်ဆင့်ပြန်ခေါ်ထားတယ်။ ဘာလို့ ဒီလို အဆင့်ဆင့် လုပ်နေရလဲ မေးစရာရှိလာပါတယ်။ တစ်ဆင့်တည်းနဲ့ရော ရအောင်လုပ်လို့ မရနိုင်ဘူးလား။ ဘာလို့ အခုလို နှစ်ဆင့်လုပ်သင့်လဲ \fEn{design principle} တချို့နဲ့ ရှင်းပြလို့ ရနိုင်ပေမဲ့ အခြေခံအဆင့်အတွက် အချိန်မကျသေးဘူးလို့ ယူဆရပါတယ်။ လက်ရှိအဆင့်မှာ အခုပြထားတဲ့ နည်းလမ်းအပြင် မိမိစဉ်းစားလို့ရတဲ့ အခြားနည်းလမ်းနဲ့ ပြောင်းလဲ စမ်းသပ်ရေးကြည့်ပါ။ ဘယ်နည်းလမ်းက ရေးရတာ (သို့) နားလည်းရတာ ပိုလွယ်လဲ၊ ပိုရိုးရှင်းလဲ စဉ်းစားချင့်ချိန်ကြည့်ခြင်းအားဖြင့် အကျိုးရှိနိုင်တယ်။ ပိုပြီးတော့ ထိုးထွင်းသိမြင်လာဖို့လည်း အထောက်အကူ ဖြစ်နိုင်ပါတယ်။

မှတ်ချက်။\qquad ။ \fCode{Bank} ဟာ ၎င်းရဲ့ အကောင့် အော့ဘ်ဂျက်တွေ အားလုံးကို စုစည်းသိမ်းထားတယ်။ ဒါကြောင့် ငအကောင့်နံပါတ်နဲ့ အကောင့်တွေကို ရှာတဲ့အလုပ်ကို \fCode{Bank} ကပဲ လုပ်ဆောင်ပေးတာ သဘာဝ\allowbreak ကျတယ်။ ငွေလွှဲတဲ့အခါ အကောင့်နှစ်ခုရဲ့ လက်ကျန်ငွေကို ပြောင်းလဲစေမှာပါ။ လက်ကျန်ငွေ \fEn{balance} က \fCode{Account} နဲ့ဆိုင်တဲ့ အချက်အလက် ဖြစ်တာကြောင့် \fCode{transfer} ဟာ \fCode{Account} မှာပဲရှိသင့်တယ်လို့ ယူဆနိုင်တယ် (တစ်နည်းအားဖြင့် ဒေတာကို ပိုင်ဆိုင်ထားတဲ့ ကလပ်စ်နဲ့ ၎င်းဒေတာအပေါ် မှီခိုလုပ်ဆောင်ရမဲ့ မက်သဒ်ကို တပေါင်းတစည်းတည်း ရှိစေချင်တာ)။  ငွေလွှဲကိစ္စဟာ ဘဏ်နဲ့ ၎င်းပိုင်ဆိုင်ထားတဲ့ အကောင့်နှစ်ခု ပူးပေါင်းလုပ်ဆောင်ရတဲ့ အလုပ်လို့ ရှုမြင်ကြည့်နိုင်ပါတယ်။

\subsection*{\fSubSecCodeBf{BankApp} Class}

\fEn{Input/Output} နဲ့ ဆိုင်တာတွေကို လုပ်ငန်းပိုင်းဆိုင်ရာတွေနဲ့ ခွဲခြားရေးသင့်တယ်လို့ အတွေ့အကြုံရှိ ပရိုဂရမ်တည်ဆောက်သူတွေ အားလုံးက လက်ခံထားကြပါတယ်။ လုပ်ငန်းပိုင်းဆိုင်ရာကို \fEn{domain} လို့ ခေါ်ပါတယ်။ ဘဏ်လုပ်ငန်းမှာဆိုရင် အကောင့်ဖွင့်ခြင်း၊ ငွေ သွင်း/လွှဲ/ထုတ် လုပ်ခြင်း၊ အတိုးပေးခြင်း၊ ချေးငွေထုတ်ပေးခြင်း စတာတွေဟာ \fEn{banking domain} နဲ့ သက်ဆိုင်တဲ့ အရာတွေပါ။ ရှေ့မှာဖော်ပြခဲတဲ့ ကလပ်စ်တွေဟာ \fEn{banking domain} နဲ့သက်ဆိုင်ပါတယ်။ 

\fEn{Input/Output} ဆိုတာ ပရိုဂရမ်နဲ့ ပြင်ပလောက ဆက်သွယ်လုပ်ဆောင်တဲ့ ကိစ္စတွေလို့ အကြမ်းဖျဉ်း ပြောလို့ရပါတယ်။ ကီးဘုဒ်ကနေ ရိုက်ထည့်လိုက်တာကို ပရိုဂရမ်က ဖတ်တာဟာ \fEn{input} ဖြစ်တယ်။ မော်နီတာမှာ စာသား/ရုပ်ပုံ ပြပေးတာဟာ \fEn{output} ဥပမာ တစ်ခုဖြစ်တယ်။ \fEn{Input/output} အပိုင်းမှာ ပါဝင်တဲ့ အခြား ဥပမာတွေလည်း အများကြီးရှိပါတယ်။

\fCode{BankApp} က \fEn{input/output} ကို အဓိက တာဝန်ယူလုပ်ဆောင်ပေးတဲ့ ကလပ်စ်ပါ။ ဒီကလပ်စ်ကနေ အော့ဘ်ဂျက်တွေ ဖန်တီးယူဖို့ မလိုတဲ့အတွက် \fEn{initializer} နဲ့ \fEn{instance attribute/method} တွေ မပါတော့ဘူး။ ကလပ်စ်အနေနဲ့ပဲ ထားပြီး ဘာလို့ အော့ဘ်ဂျက်ဖန်တီးဖို့ မလိုအပ်တာလဲ။ \fCode{BankApp} ဟာ ဘဏ်ပရိုဂရမ်တစ်ခုလုံးကို ကိုယ်စားပြုတယ်။ ပရိုဂရမ်ကတစ်ခုပဲ ရှိမှာပါ။ ဒီတော့ အော့ဘ်ဂျက်တစ်ခုချင်းစီ ပိုင်ဆိုင်ထားရမဲ့ \fEn{attribute} တွေလည်း သူ့မှာ မရှိဘူး။ ဒါကြောင့် ကလပ်စ်အနေနဲ့ပဲ ထားပြီး အသုံးပြုလို့ရတယ်။

\fEn{Class attribute} နှစ်ခု တွေ့ရတယ်။ \fEn{‘S Bank’} အတွက် \fCode{Bank} အော့ဘ်ဂျက်တစ်ခုနဲ့ ဘဏ်ခွဲနံမည်တွေ ပါတဲ့ \fEn{list} တစ်ခုတို့ ဖြစ်တယ်။ ဘဏ်ခွဲနံမည် \fEn{list} ကို \fCode{read\_branch} မှာ သုံးထားတယ်။ ဒီမက်သဒ်က ဘဏ်ခွဲနံမည် မှားနေရင် ပြန်ထည့်ခိုင်းမှာပါ။
%
\begin{py}
class BankApp:

    _bank = Bank('S Bank')
    _branches = ['Yangon', 'Mandalay', 'Naypyidaw']

    @classmethod
    def open_account(cls):
        name = read_nonempty_text('Name: ')
        nric = read_nonempty_text('NRIC: ')
        gender = read_gender('Gender (F/M): ')
        dob = read_date('Date of Birth: ')

        address_line1 = read_nonempty_text('Address Line 1: ')
        address_line2 = read_nonempty_text('Address Line 2: ')
        city = read_nonempty_text('City: ')
        division = read_nonempty_text('State/Division: ')

        init_deposit = read_decimal('Initial deposit: ')

        address = Address(address_line1, address_line2, city, division)
        holder = Holder(name, dob, nric, gender, address)
        branch = cls.read_branch()
        has_confirmed = read_confirmation('Please confirm (Y/N): ')
        if has_confirmed:
            try:
                acc = cls._bank.open_acc(holder, init_deposit, branch)
                print(f'A new account with account number {acc.acc_number} '
                      f'successfully created')
            except ValueError as err:
                print(err)
            except RuntimeError as err:
                print(err)

    @classmethod
    def read_branch(cls):
        while True:
            branch_name = read_nonempty_text('Branch: ')
            if branch_name in cls._branches:
                return branch_name
            print(f'{branch_name} not found!')

\end{py}
%
\fCode{open\_account} က \fEn{class method} ပါ။ \fCode{read\_nonempty\_text} က စာသား \fEn{input} ဖတ်ပေးတဲ့ ဖန်ရှင်ဖြစ်တယ်။ \fEnSnd{bank\_app.py} ဖိုင် \fEn{top level} မှာ သတ်မှတ်ထားတာပါ။ အခြား \fCode{read\_decimal}\fEn{,} \fCode{read\_gender} စတဲ့ \fEn{input} ဖတ်ပေးတဲ့ ဖန်ရှင်တွေကိုလည်း \fEn{top level} မှာပဲ သတ်မှတ်ထားတယ်။ ဘဏ်ခွဲ နံမည်ဖတ်တဲ့ \fCode{read\_branch} ကိုတော့ \fCode{BankApp} \fEn{class method} အနေနဲ့ သတ်မှတ်တယ်။ ဘဏ်ခွဲ
နံမည်မှားရင် ပြန်ထည့်ပေးခိုင်း ရပါမယ်။ ဘဏ်ခွဲနံမည်ဖတ်တာက ဘဏ်ပရိုဂရမ်နဲ့ပဲ ဆက်စပ်အသုံးပြုရမှာပါ။ ဒါ့အပြင် ဘဏ်ခွဲနံမည်တွေကိုလည်း \fCode{BankApp} \fEn{class attribute} အနေနဲ့ ထားသင့်တယ်။  \fCode{BankApp} ကလပ်စ်ဟာ ဆက်စပ်မှုရှိတဲ့ ဖန်ရှင်တွေနဲ့ ဒေတာတွေကို စုစည်းပေးထားတာပါ။ ဒီကလပ်စ်ထဲမှာ ပါရှိတာတွေကို \fEn{top level} ကို ထုတ်ရင်လည်း ရပေမဲ့ အခုလိုဖွဲ့စည်းထားခြင်းအားဖြင့် ဘဏ်ပရိုဂရမ်နဲ့ပဲ ဆက်စပ်အသုံးပြုရတဲ့ အရာတွေကို ပိုပြီး နီးနီးကပ်ကပ် တပေါင်းတစည်းတည်း ရှိစေတယ်။ \fEn{Top lelvel} ဖန်ရှင်တွေကတော့ အခြား ပရိုဂရမ်တွေမှာလည်း အသုံးတည့်မယ်။ နောက်ပိုင်းမှာ မော်ဒျူးတစ်ခုအနေနဲ့ ထုတ်ထားရင် ပိုကောင်းတယ်။

\subsection*{\fSubSecCodeBf{deposit}, \fSubSecCodeBf{withdraw}, \fSubSecCodeBf{transfer}}
\fEn{Input/output} အတွက် \fCode{deposit}\fEn{,} \fCode{withdraw} နဲ့ \fCode{transfer} တို့ဟာလည်း \fEn{class method} တွေပါ။ \fCode{open\_account} နားလည်ရင် ဒီမက်သဒ်တွေကိုလည်း နားလည်မှာပါ။ သဘောပေါက်အောင် လေ့လာကြည့်ဖို့ တိုက်တွန်းပါတယ်။
%
\begin{py}
@classmethod
def withdraw(cls):
    acc_no = read_nonempty_text('Account no: ')
    try:
        acc = cls._bank.retrieve_account(int(acc_no))
    except ValueError as err:
        print(err)
        return
    amt = read_decimal('Amount: ')
    print('Please check and confirm the following details:')
    print(f'Holder: {acc.holder.name}')
    print(f'Account no: {acc.acc_number}')
    print(f'Amount to withdraw: {amt}')
    has_confirmed = read_confirmation('Please confirm (Y/N): ')
    if has_confirmed:
        try:
            cls._bank.withdraw(int(acc_no), amt)
            print('Operation successfully completed.')
        except ValueError as err:
            print(err)
\end{py}
%
\betweenminted{\medskipamount}
%
\begin{py}
@classmethod
def deposit(cls):
    acc_no = read_nonempty_text('Account no: ')
    try:
        acc = cls._bank.retrieve_account(int(acc_no))
    except ValueError as err:
        print(err)
        return
    amt = read_decimal('Amount: ')
    print('Please check and confirm the following details:')
    print(f'Holder: {acc.holder.name}')
    print(f'Account no: {acc.acc_number}')
    print(f'Amount to deposit: {amt}')
    has_confirmed = read_confirmation('Please confirm (Y/N): ')
    if has_confirmed:
        try:
            cls._bank.deposit(int(acc_no), amt)
            print('Operation successfully completed.')
        except ValueError as err:
            print(err)
\end{py}
%
\betweenminted{\medskipamount}
%
\begin{py}
@classmethod
def transfer(cls):
    from_acc_no = read_nonempty_text('From account number: ')
    to_acc_no = read_nonempty_text('To account number: ')
    try:
        from_acc = cls._bank.retrieve_account(int(from_acc_no))
        to_acc = cls._bank.retrieve_account(int(to_acc_no))
    except ValueError as err:
        print(err)
        return
    amt = read_decimal('Amount: ')
    print('Please check and confirm the following details:')
    print('From:')
    print(f'    Holder: {from_acc.holder.name}')
    print(f'    Account no: {from_acc.acc_number}')
    print('To:')
    print(f'    Holder: {to_acc.holder.name}')
    print(f'    Account no: {to_acc.acc_number}')
    print(f'Amount to deposit: {amt}')
    has_confirmed = read_confirmation('Please confirm (Y/N): ')
    if has_confirmed:
        try:
            cls._bank.transfer(int(from_acc.acc_number),
                                int(to_acc.acc_number),
                                amt)
            print('Operation successfully completed.')
        except ValueError as err:
            print(err)
\end{py}
%

\subsection*{Application Loop}
အခုနောက်ဆုံးမှာ အကောင့်ဖွင့်၊ ငွေသွင်း၊ ငွေထုတ်၊ ငွေလွှဲ ဘယ် \fEn{transaction} လုပ်မလဲ ရွေးလို့ရအောင် \fEn{application loop} ကို ရေးရပါမယ်။
%
\begin{py}
@classmethod
def app_loop(cls):
    print('O (to open a new account)')
    print('D (to deposit)')
    print('W (to withdraw)')
    print('T (to transfer)')
    while True:
        menu_choice = input('Please enter O/D/W/T to start '
                            'a transaction: ')
        if menu_choice == 'O':
            cls.open_account()
        elif menu_choice == 'D':
            cls.deposit()
        elif menu_choice == 'W':
            cls.withdraw()
        elif menu_choice == 'T':
            cls.transfer()
        else:
            print(f'{menu_choice} is not valid!')
\end{py}
%
ပရိုဂရမ်က ဒီ \fEn{loop} ကနေ စအလုပ်လုပ်ရမှာဆိုတော့ အခုလို ခေါ်ပေးရုံပါပဲ
%
\begin{py}
BankApp.app_loop()
\end{py}
%
ဒီလိုပြပေးပါလိမ့်မယ်။ \fEn{O/D/W/T} ထဲက တစ်ခုရွေးရင် လုပ်ဆောင်မဲ့ ကိစ္စအလိုက် လိုအပ်တဲ့ \fEn{input} တွေ ဆက်တောင်းမှာပါ။ နောက်ဆုံးမှာ \fEn{Y/N} ကွန်ဖန်းလုပ်ရပါမယ်။  
\begin{codetxt}
ß\fEn{O (to open a new account)}ß
ß\fEn{D (to open a new account)}ß
ß\fEn{W (to open a new account)}ß
ß\fEn{T (to open a new account)}ß
ß\fEn{Please enter O/D/W/T to start a transaction:}ß 
\end{codetxt}

\subsection*{Top Level Common Functions}
\fEn{Top level} ဖန်ရှင်တွေကို လေ့လာလို့ရအောင် ဖော်ပြပေးထားပါတယ်။  အလားတူ ဖန်ရှင်တွေကို  အခန်း (\fRefNo{\ref{ch:ch08Funs}}) မှာ အသေးစိတ် ရှင်းပြထားတယ်။
%
\begin{py}
FEMALE = 'F'
MALE = 'M'

MONTHS = {'Jan': 1, 'Feb': 2, 'Mar': 3, 'Apr': 4,
          'May': 5, 'Jun': 6, 'Jul': 7, 'Aug': 8,
          'Sep': 9, 'Oct': 10, 'Nov': 11, 'Dec': 12}


def age_today(dob):
    today = date.today()
    this_bd = dob.replace(year=today.year)
    if today - dob >= this_bd - dob:
        return today.year - dob.year
    else:
        return today.year - dob.year - 1


def parse_date_str(dt_str):
    parts = dt_str.replace('/', '-').split('-')
    if parts[0].isdigit():
        return date(int(parts[2]), int(parts[1]), int(parts[0]))
    else:
        return date(int(parts[2]), MONTHS[parts[0]], int(parts[1]))


def read_date(prompt):
    while True:
        try:
            return parse_date_str(input(prompt))
        except Exception as err:
            print("Incorrect date format. Please enter the date again.")
            print('Error is probably: ' + str(err))


def read_gender(prompt):
    while True:
        gender = input(prompt)
        if gender in ['F', 'M']:
            return gender
        else:
            print('Invalid gender! Accept only F or M!')


def read_confirmation(prompt):
    while True:
        choice = input(prompt)
        if choice == 'Y':
            return True
        elif choice == 'N':
            return False
        else:
            print('Invalid gender! Accept only Y or N!')


def read_decimal(prompt):
    while True:
        value = input(prompt)
        try:
            return Decimal(value)
        except InvalidOperation:
            print('Invalid number!')


def read_nonempty_text(prompt):
    while True:
        txt = input(prompt)
        if txt.strip() != '':
            return txt
        print(f'{prompt} cannot be empty!')
\end{py}
%
\subsection*{နိဂုံး}
အခုဖော်ပြခဲ့တဲ့ ပရိုဂရမ်ကို လေ့လာခြင်းအားဖြင့် အော့ဘ်ဂျက်နဲ့ ကလပ်စ်တွေကို ပရိုဂရမ်တစ်ခု တည်\allowbreak ဆောက်ရာမှာ ဘယ်လိုအသုံးချလို့ရလဲ သဘောပေါက် နားလည်မယ်လို့ မျှော်လင့်ပါတယ်။ ဒီပရိုဂရမ်မှာ အခြားဖီချာတွေ ထပ်ဖြည့်ကြည့်ပါ။ ဥပမာ လက်ကျန်ငွေ စစ်လို့ရအောင်၊ ဘဏ်ခွဲ အသစ်ထည့်လို့ရအောင် ချဲ့ထွင်ကြည့်ရင် ပိုပြီး နားလည်လာမှာပါ။