\chapter{Inheritance and Polymorphism}

\fEnEmp{Inheritance} ဟာ ရှိပြီးသား ကလပ်စ်ကို အခြေခံပြီး အခြား ကလပ်စ်အသစ်တစ်ခု သတ်မှတ်လို့ရစေတဲ့ နည်းလမ်းဖြစ်တယ်။ \fEn{Inheritance} အသုံးပြုတဲ့အခါ နဂိုကလပ်စ်မှာ ပါဝင်တဲ့ \fEn{attribute} တွေနဲ့ မက်သဒ်တွေကို နောက်တစ်ခါ ပြန်ရေးဖို့မလိုဘဲ ကလပ်စ်အသစ်က ဆက်ခံရရှိနိုင်မဲ့အပြင် လိုအပ်သလို ထပ်မံ တိုးချဲ့ ဖြည့်စွက်တာ \fEn{(extension)} နဲ့ နဂိုရှိရင်းကို ပြင်ဆင်တာ \fEn{(modification)} လည်း  လုပ်ဆောင်လို့ရမှာဖြစ်တယ်။

 ‘ရှိသည်’ ဆိုတဲ့ \fEn{has-a relationship} နဲ့ အော့ဘ်ဂျက်တွေအကြား ဆက်စပ်နေတဲ့ ဥပမာတွေကို ရှေ့အခန်းမှာ တွေ့ခဲ့ရတယ်။ ဘဏ်တစ်ခုမှာ အကောင့်တွေ ရှိတယ်၊ အကောင့်တစ်ခုမှာ အကောင့်ပိုင်ရှင်ရှိတယ်၊ ကားမှာ အင်ဂျင်ရှိတယ် စတာတွေဟာ \fEn{has-a relationship} ဥပမာတွေပါ။ \fEn{Object-oriented programming} မှာ \fEn{has-a} အပြင် အခြားအရေးပါတဲ့ \fEn{relationship} တစ်မျိုးလည်း ရှိပါသေးတယ်။ အဲဒါကတော့  ‘ဖြစ်သည်’ ဆိုတဲ့ သဘောကိုဖော်ပြတဲ့ \fEnEmp{is-a} \fEn{relationship} ပါ။ 

ပေါ်လွင်တဲ့ ဥပမာတစ်ခု ပြပါဆိုရင် ဘတ်စ်ကားနဲ့ ကားအကြား ဆက်နွယ်မှုဟာ \fEn{is-a relationship} ပါ။ “ဘတ်စ်\allowbreak ကားသည် ကားဖြစ်သည်”။ ဒါကြောင့် ဘတ်စ်ကားမှာ ကားရဲ့ လက္ခဏာရပ်တွေ အားလုံးရှိရမှာပါ။ ထိုနည်းတူစွာ လော်ရီကားသည်လည်း  ကားဖြစ်တဲ့အတွက် သူ့မှာလည်း ကားရဲ့ လက္ခဏာရပ်တွေ ရှိရမှာ ဖြစ်တယ်။ ဘီးတွေပါမယ်၊ မောင်းလို့ရတယ်၊ ဘရိတ်အုပ်လို့ ရမယ် စသည်ဖြင့်ပေါ့။ ဒီလို ယေဘုယျ ကားဖြစ်ခြင်း ဂုဏ်အင်္ဂါရပ်တွေအပြင် ဘတ်စ်ကားသီးသန့် ကားဂုဏ်အင်္ဂါရပ်တွေ၊ လော်ရီကားသီးသန့် ကားဂုဏ်အင်္ဂါရပ်တွေလည်း ပါရှိရမယ်။ ဒါမှလည်း ဘတ်စ်ကား (သို့) လော်ရီကားလို့ ခေါ်လို့ရမယ်။ ဥပမာ ဘတ်စ်ကားဆိုရင် ခရီးသည်အတွက် ထိုင်ခုံတွေ များများပါမယ်။  ခရီးသည် အများဆုံး ဘယ်လောက်ဆံ့လဲ သတ်မှတ်ချက်ရှိတယ်။ အတက်အဆင်းအတွက် တံခါးတွေရှိမယ်။ လော်ရီကားဆိုရင်တော့ ကုန်အတွက်သီးသန့်ပဲ။ ကုန်တန်ချိန် အများဆုံးဘယ်လောက် တင်လို့ရလဲ စတာတွေရှိမယ်။ 

ကလပ်စ်တွေအကြား ဆက်စပ်မှုဟာ \fEn{is-a relationship} ဆိုရင်  \fEn{inheritance} နဲ့  ထင်ဟပ်ဖော်တာကို တွေ့ကြရမှာပါ။  \fEn{Is-a relationship} အတွက်ပဲ \fEn{inheritance} ကို သုံးလို့ရတာတော့ မဟုတ်ပါဘူး။ ပရိုဂရမ် စထရက်ချာအတွက် အသုံးပြုတာကိုလည်း တွေ့ရပါတယ်။  ဥပမာတချို့ကို တွေ့ပြီးတဲ့အခါ နောက်ပိုင်းမှာ သေချာ နားလည်လာမှာပါ။ 

\section{Inheritance ဥပမာ (၁) ‘\fSecCodeBf{Car} Class Hierarchy’}


\section{Inheritance ဥပမာ (၂) ‘\fSecCodeBf{Account} Class Hierarchy’}
ဘဏ်အကောင့် အမျိုးအစား အမျိုးမျိုး ရှိပါတယ်။ အကောင့် အမျိုးအစားအလိုက် ရရှိတဲ့ အတိုးနှုန်း၊ အချိန်ကာလတစ်ခုအတွင်း ပြုလုပ်နိုင်တဲ့ \fEn{transaction} အကြိမ်အရေအတွက်၊ အနည်းဆုံး ထားရှိရမဲ့ လက်ကျန်ငွေပမာဏ၊ လက်ရှိ လက်ကျန်ငွေထက် ပိုထုတ်လို့ ရ/မရ \fEn{(overdraft)} စတဲ့အချက်တွေ မတူကြပါဘူး။ ဥပမာ လူများစု ဖွင့်လေ့ရှိတဲ့ \fEn{savings account} ဟာ ပုံမှန် နေ့စဉ်သုံး ငွေကြေး ကိစ္စတွေအတွက် ရည်ရွယ်ပြီး အတိုး \fEn{(interest)} ရတဲ့ အကောင့် အမျိုးအစားဖြစ်တယ်။ တစ်နေ့တာ လုပ်ဆောင်နိုင်တဲ့ \fEn{transaction} အကြိမ်အရေအတွက် အများကြီးမရဘူး။  \fEn{Current account} ကတော့ \fEn{transaction} လုပ်ဖို့ အဓိက ရည်ရွယ်တာ။ \fEn{Transaction} များများ လိုတဲ့သူတွေ ဖွင့်လေ့ရှိပြီး \fEn{overdraft} လို့ခေါ်တဲ့ အကောင့် လက်ကျန်ငွေထက် ပိုထုတ်လို့ရတဲ့ ဘဏ်ဝန်ဆောင်မှုမျိုးကိုလည်း ရရှိနိုင်မှာဖြစ်တယ်။  \fEn{Current account} ဆိုရင် ဘဏ်တွေက အတိုးပေးလေ့မရှိဘူး။ နောက်ထပ် အကောင့်တစ်မျိုးကတော့ \fEn{fixed deposit account} လို့ခေါ်တဲ့ သတ်မှတ် ကာလတစ်ခုအထိ ငွေမထုတ်ဘဲ ဘဏ်မှာစုထားရတဲ့ အကောင့်ပါ။ ဘဏ်က ပုံမှန်ထက် အတိုးနှုန်း ပိုပေးပါတယ်။ သတ်မှတ်ကာလ မပြည့်မီ ပြန်ထုတ်ချင်ရင်တော့ ကျသင့် ဒဏ်ကြေးပေးဆောင်ရမှာပါ။


 
 
 
 %နှလုံးအထူးကုသည် ဆရာဝန်ဖြစ်သည်၊ ဦးနှောက်နှင့် အာရုံကြောအထူးကုသည် ဆရာဝန်ဖြစ်သည်၊ တယောသည် တူရိယာ ပစ္စည်းဖြစ်သလို စန္ဒယားသည် တူရိယာဖြစ်သည် စတာတွေဟာ \fEn{is-a relationship} ဥပမာတွေဖြစ်တယ်။ 