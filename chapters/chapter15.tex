\chapter{Basic Concurrency}\label{ch:concurrency}

$\big\llbracket \fEn{Not completed yet!} \big\rrbracket$
အမှန်တကယ် အလုပ်တစ်ခုမက တစ်ပြိုင်နက်တည်း၊ တစ်ချိန်တည်း ကိုင်တွယ်ဆောင်ရွက်နိုင်တဲ့ စွမ်းရည်ကို \fEnEmp{parallelism} ခေါ်တယ်။ \fEn{Concurrency} ကတော့ အလုပ်တွေကို သဘောတရားအားဖြင့် တစ်\allowbreak ပြိုင်တည်း လုပ်နေတယ် ထင်ရအောင် ဆောင်ရွက်ပေးတဲ့ နည်းစနစ်လို့ ဆိုရမှာပါ။ 

\fEn{Multiple CPU (Central Processing Unit)} သို့မဟုတ် \fEn{Multi-core CPU} ကွန်ပျူတာတွေမှာ  \fEn{Single-core CPU} တစ်ခုပဲဆိုရင်တော့ အလှည့်ကျစနစ်ပဲ ဖြစ်နိုင်မယ်။